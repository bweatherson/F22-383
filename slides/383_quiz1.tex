% Options for packages loaded elsewhere
\PassOptionsToPackage{unicode}{hyperref}
\PassOptionsToPackage{hyphens}{url}
%
\documentclass[
  17pt,
  letterpaper,
  ignorenonframetext,
  aspectratio=169,
]{beamer}
\usepackage{pgfpages}
\setbeamertemplate{caption}[numbered]
\setbeamertemplate{caption label separator}{: }
\setbeamercolor{caption name}{fg=normal text.fg}
\beamertemplatenavigationsymbolsempty
% Prevent slide breaks in the middle of a paragraph
\widowpenalties 1 10000
\raggedbottom
\setbeamertemplate{part page}{
  \centering
  \begin{beamercolorbox}[sep=16pt,center]{part title}
    \usebeamerfont{part title}\insertpart\par
  \end{beamercolorbox}
}
\setbeamertemplate{section page}{
  \centering
  \begin{beamercolorbox}[sep=12pt,center]{part title}
    \usebeamerfont{section title}\insertsection\par
  \end{beamercolorbox}
}
\setbeamertemplate{subsection page}{
  \centering
  \begin{beamercolorbox}[sep=8pt,center]{part title}
    \usebeamerfont{subsection title}\insertsubsection\par
  \end{beamercolorbox}
}
\AtBeginPart{
  \frame{\partpage}
}
\AtBeginSection{
  \ifbibliography
  \else
    \frame{\sectionpage}
  \fi
}
\AtBeginSubsection{
  \frame{\subsectionpage}
}

\usepackage{amsmath,amssymb}
\usepackage{lmodern}
\usepackage{iftex}
\ifPDFTeX
  \usepackage[T1]{fontenc}
  \usepackage[utf8]{inputenc}
  \usepackage{textcomp} % provide euro and other symbols
\else % if luatex or xetex
  \usepackage{unicode-math}
  \defaultfontfeatures{Scale=MatchLowercase}
  \defaultfontfeatures[\rmfamily]{Ligatures=TeX,Scale=1}
  \setmainfont[BoldFont = SF Pro Text Semibold, Scale =
MatchLowercase]{SF Pro Text Light}
\fi
\usecolortheme{wolverine}
\usefonttheme{serif} % use mainfont rather than sansfont for slide text
\useinnertheme{default}
\useoutertheme{miniframes}
% Use upquote if available, for straight quotes in verbatim environments
\IfFileExists{upquote.sty}{\usepackage{upquote}}{}
\IfFileExists{microtype.sty}{% use microtype if available
  \usepackage[]{microtype}
  \UseMicrotypeSet[protrusion]{basicmath} % disable protrusion for tt fonts
}{}
\makeatletter
\@ifundefined{KOMAClassName}{% if non-KOMA class
  \IfFileExists{parskip.sty}{%
    \usepackage{parskip}
  }{% else
    \setlength{\parindent}{0pt}
    \setlength{\parskip}{6pt plus 2pt minus 1pt}}
}{% if KOMA class
  \KOMAoptions{parskip=half}}
\makeatother
\usepackage{xcolor}
\newif\ifbibliography
\setlength{\emergencystretch}{3em} % prevent overfull lines
\setcounter{secnumdepth}{-\maxdimen} % remove section numbering


\providecommand{\tightlist}{%
  \setlength{\itemsep}{0pt}\setlength{\parskip}{0pt}}\usepackage{longtable,booktabs,array}
\usepackage{calc} % for calculating minipage widths
\usepackage{caption}
% Make caption package work with longtable
\makeatletter
\def\fnum@table{\tablename~\thetable}
\makeatother
\usepackage{graphicx}
\makeatletter
\def\maxwidth{\ifdim\Gin@nat@width>\linewidth\linewidth\else\Gin@nat@width\fi}
\def\maxheight{\ifdim\Gin@nat@height>\textheight\textheight\else\Gin@nat@height\fi}
\makeatother
% Scale images if necessary, so that they will not overflow the page
% margins by default, and it is still possible to overwrite the defaults
% using explicit options in \includegraphics[width, height, ...]{}
\setkeys{Gin}{width=\maxwidth,height=\maxheight,keepaspectratio}
% Set default figure placement to htbp
\makeatletter
\def\fps@figure{htbp}
\makeatother

\captionsetup[figure]{labelformat=empty}
\usepackage{pgfpages}
\setbeamertemplate{itemize item}[circle]
\setbeamertemplate{footline}[frame number]{}
\mode<handout>{\pgfpagesuselayout{6 on 1}[letterpaper, border shrink=8mm]}
\AtBeginSection{%
   \begin{frame}
       \tableofcontents[currentsection]
   \end{frame}
}
\makeatletter
\makeatother
\makeatletter
\makeatother
\makeatletter
\@ifpackageloaded{caption}{}{\usepackage{caption}}
\AtBeginDocument{%
\ifdefined\contentsname
  \renewcommand*\contentsname{Table of contents}
\else
  \newcommand\contentsname{Table of contents}
\fi
\ifdefined\listfigurename
  \renewcommand*\listfigurename{List of Figures}
\else
  \newcommand\listfigurename{List of Figures}
\fi
\ifdefined\listtablename
  \renewcommand*\listtablename{List of Tables}
\else
  \newcommand\listtablename{List of Tables}
\fi
\ifdefined\figurename
  \renewcommand*\figurename{Figure}
\else
  \newcommand\figurename{Figure}
\fi
\ifdefined\tablename
  \renewcommand*\tablename{Table}
\else
  \newcommand\tablename{Table}
\fi
}
\@ifpackageloaded{float}{}{\usepackage{float}}
\floatstyle{ruled}
\@ifundefined{c@chapter}{\newfloat{codelisting}{h}{lop}}{\newfloat{codelisting}{h}{lop}[chapter]}
\floatname{codelisting}{Listing}
\newcommand*\listoflistings{\listof{codelisting}{List of Listings}}
\makeatother
\makeatletter
\@ifpackageloaded{caption}{}{\usepackage{caption}}
\@ifpackageloaded{subcaption}{}{\usepackage{subcaption}}
\makeatother
\makeatletter
\@ifpackageloaded{tcolorbox}{}{\usepackage[many]{tcolorbox}}
\makeatother
\makeatletter
\@ifundefined{shadecolor}{\definecolor{shadecolor}{rgb}{.97, .97, .97}}
\makeatother
\makeatletter
\makeatother
\ifLuaTeX
  \usepackage{selnolig}  % disable illegal ligatures
\fi
\IfFileExists{bookmark.sty}{\usepackage{bookmark}}{\usepackage{hyperref}}
\IfFileExists{xurl.sty}{\usepackage{xurl}}{} % add URL line breaks if available
\urlstyle{same} % disable monospaced font for URLs
\hypersetup{
  pdftitle={Knowledge and Reality, Quiz 1},
  pdfauthor={Brian Weatherson},
  hidelinks,
  pdfcreator={LaTeX via pandoc}}

\title{Knowledge and Reality, Quiz 1}
\author{Brian Weatherson}
\date{2022-10-03}

\begin{document}
\frame{\titlepage}
\ifdefined\Shaded\renewenvironment{Shaded}{\begin{tcolorbox}[sharp corners, borderline west={3pt}{0pt}{shadecolor}, breakable, frame hidden, enhanced, boxrule=0pt, interior hidden]}{\end{tcolorbox}}\fi

\begin{frame}{Question 1}
\protect\hypertarget{question-1}{}
Datta is arguing that testimony is a method of knowledge. One of the
objections to this view that he considers is that testimonial knowledge
is dependent on other sources, such as perception of the sounds of the
speaker. What is his primary reply to this objection?
\end{frame}

\begin{frame}{True Answer}
\protect\hypertarget{true-answer}{}
This argument overgenerates since, for example, inference is also
dependent on other sources. We don't just gain inferential knowledge
from inferences with zero premises, but from inferences from things
learned by testimony or perception, and these are properly classed as
inferential.
\end{frame}

\begin{frame}{False Answer}
\protect\hypertarget{false-answer}{}
This argument overgenerates since, for example, perception is also
dependent on other sources such as inference. We can't get perceptual
knowledge without relying on our background knowledge that are
perceptual faculties are reliable, and then inferring that they are
working on this occasion.
\end{frame}

\begin{frame}{Why is False Answer False}
\protect\hypertarget{why-is-false-answer-false}{}
\begin{itemize}[<+->]
\tightlist
\item
  Datta does not think perception relies on inference.
\item
  And he certainly doesn't say this in the paper.
\item
  If he thought it did, he arguably wouldn't think that it is a method
  of knowing in the relevant sense.
\end{itemize}
\end{frame}

\begin{frame}{Question 3}
\protect\hypertarget{question-3}{}
What is the ``Approved List'' response to BonJour's example of Norman
the clairvoyant, which is meant to be a counterexample to reliabilism.
\end{frame}

\begin{frame}{True Answer}
\protect\hypertarget{true-answer-1}{}
Although Norman is justified in his beliefs, people mistakenly think he
is not, since they usually judge justification by asking whether the
belief was formed by a process that they think is reliable in the real
world.
\end{frame}

\begin{frame}{False Answer}
\protect\hypertarget{false-answer-1}{}
Norman is not a counterexample since reliabilism says that justified
beliefs are those obtained by a process known to be reliable, and Norman
doesn't know that his powers are reliable.
\end{frame}

\begin{frame}{Why is False Answer False}
\protect\hypertarget{why-is-false-answer-false-1}{}
\begin{itemize}[<+->]
\tightlist
\item
  Reliabilism does not require that processes are known to be reliable.
\item
  Everyone (more or less) thinks you can get knowledge from processes
  known to be reliable.
\item
  What's distinctive about reliabilism is that you don't need this
  knowldge; fact of reliability can be enough.
\end{itemize}
\end{frame}

\begin{frame}{Question 4}
\protect\hypertarget{question-4}{}
Why do they {[}Ichikawa and Steup{]} say that the ``No False Lemmas''
analysis fails?
\end{frame}

\begin{frame}{True Answer}
\protect\hypertarget{true-answer-2}{}
Because in cases like the robot dog, there is no knowledge, but no
reasoning from a false lemma.
\end{frame}

\begin{frame}{False Answer}
\protect\hypertarget{false-answer-2}{}
Because in cases like the robot dog, there is knowledge even though
there is reasoning from a false lemma.

\begin{itemize}[<+->]
\tightlist
\item
  Here it's just a matter of checking whether there is knowledge in the
  robot dog case, and they say there isn't.
\end{itemize}
\end{frame}

\begin{frame}{Essay}
\protect\hypertarget{essay}{}
A good 300-level philosophy essay does three things.

\begin{enumerate}[<+->]
\tightlist
\item
  Say what question the person you're writing about is asking.
\item
  Say what their answer is.
\item
  Evaluate that answer.
\end{enumerate}
\end{frame}

\begin{frame}{Essay}
\protect\hypertarget{essay-1}{}
The evaluation need not be enormous, and can't be in the length
involved.

\begin{itemize}[<+->]
\tightlist
\item
  An example where the answer seems to go wrong would do.
\item
  An application of the answer to a separate problem, to either bring
  out its strengths or weaknesses, would also do.
\end{itemize}
\end{frame}

\begin{frame}{Essay}
\protect\hypertarget{essay-2}{}
For 1500-2000 words, I don't expect you to be writing a
\textbf{research} paper.

\begin{itemize}[<+->]
\tightlist
\item
  You can get top grades without reading anything that's not on the
  syllabus.
\item
  But you do have to read the stuff that is on the syllabus, including
  the recommended readings for the topics you're writing on,
  \textbf{closely}.
\end{itemize}
\end{frame}



\end{document}
