% Options for packages loaded elsewhere
\PassOptionsToPackage{unicode}{hyperref}
\PassOptionsToPackage{hyphens}{url}
%
\documentclass[
  17pt,
  letterpaper,
  ignorenonframetext,
  aspectratio=169,
  handout,
  xcolor={dvipsnames}]{beamer}
\usepackage{pgfpages}
\setbeamertemplate{caption}[numbered]
\setbeamertemplate{caption label separator}{: }
\setbeamercolor{caption name}{fg=normal text.fg}
\beamertemplatenavigationsymbolsempty
% Prevent slide breaks in the middle of a paragraph
\widowpenalties 1 10000
\raggedbottom
\setbeamertemplate{part page}{
  \centering
  \begin{beamercolorbox}[sep=16pt,center]{part title}
    \usebeamerfont{part title}\insertpart\par
  \end{beamercolorbox}
}
\setbeamertemplate{section page}{
  \centering
  \begin{beamercolorbox}[sep=12pt,center]{part title}
    \usebeamerfont{section title}\insertsection\par
  \end{beamercolorbox}
}
\setbeamertemplate{subsection page}{
  \centering
  \begin{beamercolorbox}[sep=8pt,center]{part title}
    \usebeamerfont{subsection title}\insertsubsection\par
  \end{beamercolorbox}
}
\AtBeginPart{
  \frame{\partpage}
}
\AtBeginSection{
  \ifbibliography
  \else
    \frame{\sectionpage}
  \fi
}
\AtBeginSubsection{
  \frame{\subsectionpage}
}

\usepackage{amsmath,amssymb}
\usepackage{lmodern}
\usepackage{iftex}
\ifPDFTeX
  \usepackage[T1]{fontenc}
  \usepackage[utf8]{inputenc}
  \usepackage{textcomp} % provide euro and other symbols
\else % if luatex or xetex
  \usepackage{unicode-math}
  \defaultfontfeatures{Scale=MatchLowercase}
  \defaultfontfeatures[\rmfamily]{Ligatures=TeX,Scale=1}
  \setmainfont[BoldFont = SF Pro Text Semibold, Scale =
MatchLowercase]{SF Pro Text Light}
\fi
\usecolortheme{wolverine}
\usefonttheme{serif} % use mainfont rather than sansfont for slide text
\useinnertheme{default}
\useoutertheme{miniframes}
% Use upquote if available, for straight quotes in verbatim environments
\IfFileExists{upquote.sty}{\usepackage{upquote}}{}
\IfFileExists{microtype.sty}{% use microtype if available
  \usepackage[]{microtype}
  \UseMicrotypeSet[protrusion]{basicmath} % disable protrusion for tt fonts
}{}
\makeatletter
\@ifundefined{KOMAClassName}{% if non-KOMA class
  \IfFileExists{parskip.sty}{%
    \usepackage{parskip}
  }{% else
    \setlength{\parindent}{0pt}
    \setlength{\parskip}{6pt plus 2pt minus 1pt}}
}{% if KOMA class
  \KOMAoptions{parskip=half}}
\makeatother
\usepackage{xcolor}
\newif\ifbibliography
\setlength{\emergencystretch}{3em} % prevent overfull lines
\setcounter{secnumdepth}{-\maxdimen} % remove section numbering


\providecommand{\tightlist}{%
  \setlength{\itemsep}{0pt}\setlength{\parskip}{0pt}}\usepackage{longtable,booktabs,array}
\usepackage{calc} % for calculating minipage widths
\usepackage{caption}
% Make caption package work with longtable
\makeatletter
\def\fnum@table{\tablename~\thetable}
\makeatother
\usepackage{graphicx}
\makeatletter
\def\maxwidth{\ifdim\Gin@nat@width>\linewidth\linewidth\else\Gin@nat@width\fi}
\def\maxheight{\ifdim\Gin@nat@height>\textheight\textheight\else\Gin@nat@height\fi}
\makeatother
% Scale images if necessary, so that they will not overflow the page
% margins by default, and it is still possible to overwrite the defaults
% using explicit options in \includegraphics[width, height, ...]{}
\setkeys{Gin}{width=\maxwidth,height=\maxheight,keepaspectratio}
% Set default figure placement to htbp
\makeatletter
\def\fps@figure{htbp}
\makeatother

\captionsetup[figure]{labelformat=empty}
\usepackage{pgfpages}
\setbeamertemplate{itemize item}[circle]
\setbeamertemplate{footline}[frame number]{}
\mode<handout>{\pgfpagesuselayout{6 on 1}[letterpaper, border shrink=8mm]}
\AtBeginSection{%
   \begin{frame}
       \tableofcontents[currentsection]
   \end{frame}
}
\makeatletter
\makeatother
\makeatletter
\makeatother
\makeatletter
\@ifpackageloaded{caption}{}{\usepackage{caption}}
\AtBeginDocument{%
\ifdefined\contentsname
  \renewcommand*\contentsname{Table of contents}
\else
  \newcommand\contentsname{Table of contents}
\fi
\ifdefined\listfigurename
  \renewcommand*\listfigurename{List of Figures}
\else
  \newcommand\listfigurename{List of Figures}
\fi
\ifdefined\listtablename
  \renewcommand*\listtablename{List of Tables}
\else
  \newcommand\listtablename{List of Tables}
\fi
\ifdefined\figurename
  \renewcommand*\figurename{Figure}
\else
  \newcommand\figurename{Figure}
\fi
\ifdefined\tablename
  \renewcommand*\tablename{Table}
\else
  \newcommand\tablename{Table}
\fi
}
\@ifpackageloaded{float}{}{\usepackage{float}}
\floatstyle{ruled}
\@ifundefined{c@chapter}{\newfloat{codelisting}{h}{lop}}{\newfloat{codelisting}{h}{lop}[chapter]}
\floatname{codelisting}{Listing}
\newcommand*\listoflistings{\listof{codelisting}{List of Listings}}
\makeatother
\makeatletter
\@ifpackageloaded{caption}{}{\usepackage{caption}}
\@ifpackageloaded{subcaption}{}{\usepackage{subcaption}}
\makeatother
\makeatletter
\@ifpackageloaded{tcolorbox}{}{\usepackage[many]{tcolorbox}}
\makeatother
\makeatletter
\@ifundefined{shadecolor}{\definecolor{shadecolor}{rgb}{.97, .97, .97}}
\makeatother
\makeatletter
\makeatother
\ifLuaTeX
  \usepackage{selnolig}  % disable illegal ligatures
\fi
\IfFileExists{bookmark.sty}{\usepackage{bookmark}}{\usepackage{hyperref}}
\IfFileExists{xurl.sty}{\usepackage{xurl}}{} % add URL line breaks if available
\urlstyle{same} % disable monospaced font for URLs
\hypersetup{
  pdftitle={Knowledge and Reality, Lecture 22},
  pdfauthor={Brian Weatherson},
  hidelinks,
  pdfcreator={LaTeX via pandoc}}

\title{Knowledge and Reality, Lecture 22}
\author{Brian Weatherson}
\date{11/16/22}

\begin{document}
\frame{\titlepage}
\ifdefined\Shaded\renewenvironment{Shaded}{\begin{tcolorbox}[sharp corners, boxrule=0pt, interior hidden, breakable, enhanced, borderline west={3pt}{0pt}{shadecolor}, frame hidden]}{\end{tcolorbox}}\fi

\begin{frame}{Plan}
\protect\hypertarget{plan}{}
\begin{enumerate}[<+->]
\tightlist
\item
  Ways Perception can be Irrational.
\item
  Circularity
\item
  Circularity and Perception
\end{enumerate}
\end{frame}

\hypertarget{perception-and-rationality}{%
\section{Perception and Rationality}\label{perception-and-rationality}}

\begin{frame}{Two Failure Cases}
\protect\hypertarget{two-failure-cases}{}
\begin{enumerate}[<+->]
\tightlist
\item
  Backward looking
\item
  Forward looking
\end{enumerate}
\end{frame}

\begin{frame}{Backward looking failure}
\protect\hypertarget{backward-looking-failure}{}
\begin{itemize}[<+->]
\tightlist
\item
  A perception might be partially constituted by an inference from
  earlier beliefs.
\item
  Those beliefs might be irrational.
\end{itemize}
\end{frame}

\begin{frame}{Notes}
\protect\hypertarget{notes}{}
\begin{enumerate}[<+->]
\tightlist
\item
  That it's a misperception doesn't on its own make it irrational.
\item
  That the earlier belief is false doesn't on its own make it
  irrational.
\item
  That the earlier belief is morally dubious doesn't on its own make it
  irrational.
\end{enumerate}
\end{frame}

\begin{frame}{An Irrational Example}
\protect\hypertarget{an-irrational-example}{}
\begin{itemize}[<+->]
\tightlist
\item
  Hero learns that there are as many guns as people in America.
\item
  Being unable to believe that someone could have more than one gun,
  they infer that every American has a gun.
\item
  This isn't just wrong, it's leaping to a conclusion.
\end{itemize}
\end{frame}

\begin{frame}{An Irrational Example}
\protect\hypertarget{an-irrational-example-1}{}
Because Hero has this belief, every time they see someone in America
(and leap to the conclusion that the person is an American) with any
kind of bulge in their clothing, they see it as a gun.
\end{frame}

\begin{frame}{An Irrational Example?}
\protect\hypertarget{an-irrational-example-2}{}
There is still a bit of philosophical work to do to get this to be
irrational.

\begin{itemize}[<+->]
\tightlist
\item
  Have to argue that inference really is involved. Chapter 5 started
  this argument, and it will continue in chapter 8.
\item
  Have to argue that the inference is so central that it makes the
  resuling perception irrational.
\end{itemize}
\end{frame}

\begin{frame}{Stereotypes}
\protect\hypertarget{stereotypes}{}
This belief is I guess a stereotype, but even that is perhaps avoidable.

\begin{itemize}[<+->]
\tightlist
\item
  Imagine Hero makes the same inference/perception when in Canada, from
  the belief that number of guns = number of people, so every Canadian
  has a gun.
\item
  That's not a stereotype (at least not a familiar one) but still
  irrational.
\end{itemize}
\end{frame}

\begin{frame}{Forward Looking}
\protect\hypertarget{forward-looking}{}
But what I want to spend most time on today is that there is another way
in which experiences can, for Siegel, be epistemically downgraded.

\begin{itemize}[<+->]
\tightlist
\item
  They might not support as many things going forward.
\end{itemize}
\end{frame}

\begin{frame}{Endorsement}
\protect\hypertarget{endorsement}{}
What a state supports in general is this incredibly tricky question, and
one which we'll make more tricky as today rolls along.

\begin{itemize}[<+->]
\tightlist
\item
  But note that for experiences there is one kind of support that is
  very important.
\end{itemize}
\end{frame}

\begin{frame}{Endorsement}
\protect\hypertarget{endorsement-1}{}
Someone \textbf{endorses} an experience iff they form a belief with the
same content (or at least part of the same content) on the basis of the
experience.

\begin{itemize}[<+->]
\tightlist
\item
  Here's one big question about experiences: would it be rational to
  endorse them?
\item
  And Siegel thinks that often it wouldn't be, in virtue of their
  inferential nature.
\end{itemize}
\end{frame}

\hypertarget{circularity}{%
\section{Circularity}\label{circularity}}

\begin{frame}{A Puzzle}
\protect\hypertarget{a-puzzle}{}
I'll start by taking several steps back, and going over a very hard
puzzle, which I don't think anyone has a good solution to.

\begin{itemize}[<+->]
\tightlist
\item
  The puzzle concerns what you can do with information you receive in a
  pretty reliable book.
\end{itemize}
\end{frame}

\begin{frame}{Background}
\protect\hypertarget{background}{}
\begin{itemize}[<+->]
\tightlist
\item
  Hero is reading a book about the origins of WWI.
\item
  The book says that Henriette Caillaux was acquitted of the murder of
  Gaston Calmette. (This was a big deal in France in July 1914.) Call
  this \emph{p}.
\item
  The book is from a reputable author and press, was well reviewed, and
  Hero believes that \emph{p} because the book says so.
\end{itemize}
\end{frame}

\begin{frame}{Cross-Checking}
\protect\hypertarget{cross-checking}{}
Later, Hero starts to worry that maybe the book made some mistakes. They
decide to investigate whether the book is accurate.

\begin{itemize}[<+->]
\tightlist
\item
  In the course of this check, they do the following reasoning.
\end{itemize}
\end{frame}

\begin{frame}{Cross-checking}
\protect\hypertarget{cross-checking-1}{}
\begin{enumerate}[<+->]
\tightlist
\item
  \emph{p} is true (as I remember from my reading).
\item
  The book says that \emph{p} (as I know from looking back at the
  pages).
\item
  So that's something the book got right (from 1, 2)
\item
  So that's evidence in favor of the book's reliability.
\end{enumerate}
\end{frame}

\begin{frame}{Circularity}
\protect\hypertarget{circularity-1}{}
And something has gone horribly badly wrong.

\begin{itemize}[<+->]
\tightlist
\item
  You can't get evidence that a book is reliable by opening it up,
  reading it, and believing what it says.
\item
  But what exactly has gone wrong?
\end{itemize}
\end{frame}

\begin{frame}{A Puzzle about Chains}
\protect\hypertarget{a-puzzle-about-chains}{}
Hero makes two inferential steps.

\begin{enumerate}[<+->]
\tightlist
\item
  From reading \emph{p} to believing \emph{p}.
\item
  From believing \emph{p} and reading \emph{p} to believing that the
  book was right about whether \emph{p}.
\end{enumerate}
\end{frame}

\begin{frame}{A Puzzle about Chains}
\protect\hypertarget{a-puzzle-about-chains-1}{}
Four options

\begin{enumerate}[<+->]
[A.]
\tightlist
\item
  Step 1 is bad.
\item
  Step 2 is bad.
\item
  Neither step alone is bad, but the pair is bad.
\item
  Actually, Hero did nothing wrong.
\end{enumerate}
\end{frame}

\begin{frame}{A Puzzle About Chains}
\protect\hypertarget{a-puzzle-about-chains-2}{}
It isn't too hard to argue for C.

\begin{itemize}[<+->]
\tightlist
\item
  If you don't think you can believe what you read in well reviewed
  books by reputable authors/presses, you're well down the road to a
  nasty kind of scepticism.
\item
  And inference B is practically a matter of logic.
\item
  But Hero did something wrong, so D is out.
\end{itemize}
\end{frame}

\begin{frame}{Siegel's Position}
\protect\hypertarget{siegels-position}{}
As I understand her, I think she says that B is the problem.

\begin{itemize}[<+->]
\tightlist
\item
  Which is striking because it's literally just logic.
\item
  But beliefs come with powers, and this belief does not have the power
  to support claims about its own source's reliability.
\item
  That's true even with the support goes via logic.
\end{itemize}
\end{frame}

\hypertarget{perception-and-circularity}{%
\section{Perception and Circularity}\label{perception-and-circularity}}

\begin{frame}{Wright's Example}
\protect\hypertarget{wrights-example}{}
Hero is walking past a school, let's say Pioneer High School on State
St, and sees some people playing on a soccer field.

\begin{itemize}[<+->]
\tightlist
\item
  It looks just like a soccer game, and in fact it is.
\item
  But if they were pretending to play soccer to film a TV show, Hero
  wouldn't have been able to tell.
\end{itemize}
\end{frame}

\begin{frame}{Wright's Example}
\protect\hypertarget{wrights-example-1}{}
Hero sees the ball go into the net and the referee give the signal for a
goal.

\begin{itemize}[<+->]
\tightlist
\item
  Hero believes (and maybe sees?) that a goal was scored.
\end{itemize}
\end{frame}

\begin{frame}{Wright's Example}
\protect\hypertarget{wrights-example-2}{}
Something would go wrong if Hero reasoned as follows

\begin{enumerate}[<+->]
\tightlist
\item
  That was a goal (as opposed to a pretend goal).
\item
  If that was a goal, they are really playing soccer (as opposed to
  filming a TV show).
\item
  So, they are really playing soccer.
\end{enumerate}
\end{frame}

\begin{frame}{Perception}
\protect\hypertarget{perception}{}
Arguably, Hero \textbf{sees} that a goal was scored.

\begin{itemize}[<+->]
\tightlist
\item
  If that's true, and Wright's account of the case is also true, then
  it's a case where perception is epistemically downgraded.
\item
  The perception doesn't have the power to support the conclusion that
  there's a game going on, even though its content entails that.
\end{itemize}
\end{frame}

\begin{frame}{Kinds of Downgrade}
\protect\hypertarget{kinds-of-downgrade}{}
But note what a weird case this is. (This is not to say Siegel's wrong;
it is actually a tricky weird case.)

\begin{itemize}[<+->]
\tightlist
\item
  Hero has excellent reason to believe that a goal was scored. The
  perception is reason-giving.
\end{itemize}
\end{frame}

\begin{frame}{Kinds of Downgrade}
\protect\hypertarget{kinds-of-downgrade-1}{}
But note what a weird case this is. (This is not to say Siegel's wrong;
it is actually a tricky weird case.)

\begin{itemize}[<+->]
\tightlist
\item
  Hero cannot infer that there is a game going on, even though that uses
  just logic.
\item
  The belief that a goal was scored loses some forward-power.
\end{itemize}
\end{frame}

\begin{frame}{Kinds of Downgrade}
\protect\hypertarget{kinds-of-downgrade-2}{}
Part of the solution here is that it must have been antecedently
reasonable for Hero to believe there was a game going on.

\begin{itemize}[<+->]
\tightlist
\item
  Part of what's happened here is that her perception does not give her
  \textbf{additional} reason to believe that there is game.
\end{itemize}
\end{frame}

\begin{frame}{Generalising}
\protect\hypertarget{generalising}{}
This might make us think of other things we have background reason to
believe.

\begin{itemize}[<+->]
\tightlist
\item
  Looking around I see some chairs.
\item
  The existence of chairs implies the existence of things external to
  me.
\item
  So do my perceptual experiences give me reason to believe in the
  existence of things external to me?
\end{itemize}
\end{frame}

\begin{frame}{Generalising}
\protect\hypertarget{generalising-1}{}
It seems like they don't.

\begin{itemize}[<+->]
\tightlist
\item
  But does that mean every perception lacks a kind of forward-power?
\item
  And if so, does that mean that in Wright's case there isn't really a
  downgrade?
\end{itemize}
\end{frame}

\begin{frame}{Endorsement}
\protect\hypertarget{endorsement-2}{}
No, and this is where the notion of endorsement becomes important.

\begin{itemize}[<+->]
\tightlist
\item
  A perception that doesn't well-found the belief that is a simple
  endorsement of it loses a particularly important kind of
  forward-looking power.
\end{itemize}
\end{frame}

\begin{frame}{Bananas}
\protect\hypertarget{bananas}{}
That's possibly what's going on in the banana case.

\begin{itemize}[<+->]
\tightlist
\item
  If there is a kind of circularity that leads to it being bad to infer
  this banana is yellow, then that's a special kind of badness.
\end{itemize}
\end{frame}

\begin{frame}{For Next Time}
\protect\hypertarget{for-next-time}{}
We'll jump ahead to chapter 8, looking at whether it's plausible that
perceptions really do involve inference.
\end{frame}



\end{document}
