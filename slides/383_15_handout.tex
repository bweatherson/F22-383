% Options for packages loaded elsewhere
\PassOptionsToPackage{unicode}{hyperref}
\PassOptionsToPackage{hyphens}{url}
%
\documentclass[
  17pt,
  letterpaper,
  ignorenonframetext,
  aspectratio=169,
  handout]{beamer}
\usepackage{pgfpages}
\setbeamertemplate{caption}[numbered]
\setbeamertemplate{caption label separator}{: }
\setbeamercolor{caption name}{fg=normal text.fg}
\beamertemplatenavigationsymbolsempty
% Prevent slide breaks in the middle of a paragraph
\widowpenalties 1 10000
\raggedbottom
\setbeamertemplate{part page}{
  \centering
  \begin{beamercolorbox}[sep=16pt,center]{part title}
    \usebeamerfont{part title}\insertpart\par
  \end{beamercolorbox}
}
\setbeamertemplate{section page}{
  \centering
  \begin{beamercolorbox}[sep=12pt,center]{part title}
    \usebeamerfont{section title}\insertsection\par
  \end{beamercolorbox}
}
\setbeamertemplate{subsection page}{
  \centering
  \begin{beamercolorbox}[sep=8pt,center]{part title}
    \usebeamerfont{subsection title}\insertsubsection\par
  \end{beamercolorbox}
}
\AtBeginPart{
  \frame{\partpage}
}
\AtBeginSection{
  \ifbibliography
  \else
    \frame{\sectionpage}
  \fi
}
\AtBeginSubsection{
  \frame{\subsectionpage}
}

\usepackage{amsmath,amssymb}
\usepackage{lmodern}
\usepackage{iftex}
\ifPDFTeX
  \usepackage[T1]{fontenc}
  \usepackage[utf8]{inputenc}
  \usepackage{textcomp} % provide euro and other symbols
\else % if luatex or xetex
  \usepackage{unicode-math}
  \defaultfontfeatures{Scale=MatchLowercase}
  \defaultfontfeatures[\rmfamily]{Ligatures=TeX,Scale=1}
  \setmainfont[BoldFont = SF Pro Text Semibold, Scale =
MatchLowercase]{SF Pro Text Light}
\fi
\usecolortheme{wolverine}
\usefonttheme{serif} % use mainfont rather than sansfont for slide text
\useinnertheme{default}
\useoutertheme{miniframes}
% Use upquote if available, for straight quotes in verbatim environments
\IfFileExists{upquote.sty}{\usepackage{upquote}}{}
\IfFileExists{microtype.sty}{% use microtype if available
  \usepackage[]{microtype}
  \UseMicrotypeSet[protrusion]{basicmath} % disable protrusion for tt fonts
}{}
\makeatletter
\@ifundefined{KOMAClassName}{% if non-KOMA class
  \IfFileExists{parskip.sty}{%
    \usepackage{parskip}
  }{% else
    \setlength{\parindent}{0pt}
    \setlength{\parskip}{6pt plus 2pt minus 1pt}}
}{% if KOMA class
  \KOMAoptions{parskip=half}}
\makeatother
\usepackage{xcolor}
\newif\ifbibliography
\setlength{\emergencystretch}{3em} % prevent overfull lines
\setcounter{secnumdepth}{-\maxdimen} % remove section numbering


\providecommand{\tightlist}{%
  \setlength{\itemsep}{0pt}\setlength{\parskip}{0pt}}\usepackage{longtable,booktabs,array}
\usepackage{calc} % for calculating minipage widths
\usepackage{caption}
% Make caption package work with longtable
\makeatletter
\def\fnum@table{\tablename~\thetable}
\makeatother
\usepackage{graphicx}
\makeatletter
\def\maxwidth{\ifdim\Gin@nat@width>\linewidth\linewidth\else\Gin@nat@width\fi}
\def\maxheight{\ifdim\Gin@nat@height>\textheight\textheight\else\Gin@nat@height\fi}
\makeatother
% Scale images if necessary, so that they will not overflow the page
% margins by default, and it is still possible to overwrite the defaults
% using explicit options in \includegraphics[width, height, ...]{}
\setkeys{Gin}{width=\maxwidth,height=\maxheight,keepaspectratio}
% Set default figure placement to htbp
\makeatletter
\def\fps@figure{htbp}
\makeatother

\captionsetup[figure]{labelformat=empty}
\usepackage{pgfpages}
\setbeamertemplate{itemize item}[circle]
\setbeamertemplate{footline}[frame number]{}
\mode<handout>{\pgfpagesuselayout{6 on 1}[letterpaper, border shrink=8mm]}
\AtBeginSection{%
   \begin{frame}
       \tableofcontents[currentsection]
   \end{frame}
}
\makeatletter
\makeatother
\makeatletter
\makeatother
\makeatletter
\@ifpackageloaded{caption}{}{\usepackage{caption}}
\AtBeginDocument{%
\ifdefined\contentsname
  \renewcommand*\contentsname{Table of contents}
\else
  \newcommand\contentsname{Table of contents}
\fi
\ifdefined\listfigurename
  \renewcommand*\listfigurename{List of Figures}
\else
  \newcommand\listfigurename{List of Figures}
\fi
\ifdefined\listtablename
  \renewcommand*\listtablename{List of Tables}
\else
  \newcommand\listtablename{List of Tables}
\fi
\ifdefined\figurename
  \renewcommand*\figurename{Figure}
\else
  \newcommand\figurename{Figure}
\fi
\ifdefined\tablename
  \renewcommand*\tablename{Table}
\else
  \newcommand\tablename{Table}
\fi
}
\@ifpackageloaded{float}{}{\usepackage{float}}
\floatstyle{ruled}
\@ifundefined{c@chapter}{\newfloat{codelisting}{h}{lop}}{\newfloat{codelisting}{h}{lop}[chapter]}
\floatname{codelisting}{Listing}
\newcommand*\listoflistings{\listof{codelisting}{List of Listings}}
\makeatother
\makeatletter
\@ifpackageloaded{caption}{}{\usepackage{caption}}
\@ifpackageloaded{subcaption}{}{\usepackage{subcaption}}
\makeatother
\makeatletter
\@ifpackageloaded{tcolorbox}{}{\usepackage[many]{tcolorbox}}
\makeatother
\makeatletter
\@ifundefined{shadecolor}{\definecolor{shadecolor}{rgb}{.97, .97, .97}}
\makeatother
\makeatletter
\makeatother
\ifLuaTeX
  \usepackage{selnolig}  % disable illegal ligatures
\fi
\IfFileExists{bookmark.sty}{\usepackage{bookmark}}{\usepackage{hyperref}}
\IfFileExists{xurl.sty}{\usepackage{xurl}}{} % add URL line breaks if available
\urlstyle{same} % disable monospaced font for URLs
\hypersetup{
  pdftitle={Knowledge and Reality, Lecture 15},
  pdfauthor={Brian Weatherson},
  hidelinks,
  pdfcreator={LaTeX via pandoc}}

\title{Knowledge and Reality, Lecture 15}
\author{Brian Weatherson}
\date{2022-10-24}

\begin{document}
\frame{\titlepage}
\ifdefined\Shaded\renewenvironment{Shaded}{\begin{tcolorbox}[frame hidden, sharp corners, interior hidden, borderline west={3pt}{0pt}{shadecolor}, boxrule=0pt, enhanced, breakable]}{\end{tcolorbox}}\fi

\hypertarget{themes}{%
\section{Themes}\label{themes}}

\begin{frame}{Two Big Themes}
\protect\hypertarget{two-big-themes}{}
\begin{enumerate}[<+->]
\tightlist
\item
  What's the importance of the ``Anselmian glance'', the ability to hold
  an entire argument in view at once?
\item
  Is there anything epistemologically special about first-personal
  reasoning, and if so what is it?
\end{enumerate}
\end{frame}

\begin{frame}{Hypothesis}
\protect\hypertarget{hypothesis}{}
These two questions are connected.

\begin{itemize}[<+->]
\tightlist
\item
  The glance is special because it allows this special kind of
  first-personal reasoning.
\end{itemize}
\end{frame}

\begin{frame}{A Step Back}
\protect\hypertarget{a-step-back}{}
But why is that?

\begin{itemize}[<+->]
\tightlist
\item
  The end of the chapter goes over several reasons why we might not care
  about first-personal reasoning.
\item
  Here's one case where that seems like the right thing to say.
\end{itemize}
\end{frame}

\begin{frame}{Fermat's Last Theorem}
\protect\hypertarget{fermats-last-theorem}{}
A famous mathematical problem. Are there positive integers
\(x, y, z, n\), with \(n > 2\), such that

\[
x^n + y^n = z^n
\]

\begin{itemize}[<+->]
\tightlist
\item
  A few years back, Andrew Wiles (then at Princeton) proved that the
  answer is no.
\end{itemize}
\end{frame}

\begin{frame}{Fermat's Last Theorem}
\protect\hypertarget{fermats-last-theorem-1}{}
The proof is surprisingly complicated, going through some parts of math
that don't look like they have anything to do with the result.

\begin{itemize}[<+->]
\tightlist
\item
  I haven't made much effort to follow it, and I'm pretty sure I would
  fail.
\item
  But I now know the theorem is true; people I trust say the proof
  works.
\end{itemize}
\end{frame}

\begin{frame}{Fermat's Last Theorem}
\protect\hypertarget{fermats-last-theorem-2}{}
Imagine Wiles himself forgets key parts of the proof, and could only
reconstruct it by looking back over his notes.

\begin{itemize}[<+->]
\tightlist
\item
  Question: Is Wiles (in this example, not in reality!) any better
  placed with respect to Fermat's Last Theorem than the rest of us?
\end{itemize}
\end{frame}

\begin{frame}{Fermat's Last Theorem}
\protect\hypertarget{fermats-last-theorem-3}{}
Pasnau (as I read the end of Ch. 5) wants to say no.

\begin{itemize}[<+->]
\tightlist
\item
  The fact that he, long ago, did the proof is of no epistemological
  significance.
\item
  He knows that someone, his younger self, did a proof; but we know that
  too.
\item
  And there are no other relevant differences between him now, and us
  now.
\end{itemize}
\end{frame}

\begin{frame}{Mathematics and Morality}
\protect\hypertarget{mathematics-and-morality}{}
That seems kind of plausible to me at first glance, at least for this
case.

\begin{itemize}[<+->]
\tightlist
\item
  But I want to talk us through a couple of other cases where it is more
  plausible that the Anselmian glance matters.
\end{itemize}
\end{frame}

\hypertarget{moral-testimony}{%
\section{Moral Testimony}\label{moral-testimony}}

\begin{frame}{An Example}
\protect\hypertarget{an-example}{}
Alice and her work colleague Bob are setting down to dinner. Bob says
he's thinking of ordering a hambuger, but it's sure whether it is
morally ok. Alice says no, meat eating is wrong. Bob has known Alice for
a long time, and has a high opinion of her moral judgments. So he orders
a salad.
\end{frame}

\begin{frame}{An Example}
\protect\hypertarget{an-example-1}{}
\begin{itemize}[<+->]
\tightlist
\item
  Did that seem strange?
\item
  If so, what was the strangeness?
\end{itemize}
\end{frame}

\begin{frame}{One Hypothesis}
\protect\hypertarget{one-hypothesis}{}
Moral knowledge only comes from moral reasoning/experience/empathy.

\begin{itemize}[<+->]
\tightlist
\item
  The problem is that Bob can't know that meat eating is wrong this way.
\end{itemize}
\end{frame}

\begin{frame}{Problems for That Hypothesis}
\protect\hypertarget{problems-for-that-hypothesis}{}
Why couldn't Bob get knowledge that way?

\begin{itemize}[<+->]
\tightlist
\item
  Alice is a reliable testifier, and reliable testimony is a source of
  knowledge.
\item
  Maybe there is an answer here, but it isn't obvious what it is.
\end{itemize}
\end{frame}

\begin{frame}{Another Hypothesis}
\protect\hypertarget{another-hypothesis}{}
It's just because this is an extreme case.

\begin{itemize}[<+->]
\tightlist
\item
  It's weird to not have opinions about meat-eating.
\item
  But we can probably do the same thing with more obscure topics, and
  it's still a bit weird.
\end{itemize}
\end{frame}

\begin{frame}{Oysters}
\protect\hypertarget{oysters}{}
Imagine that Carla and Dave are both vegetarians, and Dave trusts
Carla's judgments. At dinner one night, Dave sees Carla eating oysters.
He asks her if it's morally ok to eat oysters, and she says yes. Without
any more details, he orders oysters.

\begin{itemize}[<+->]
\tightlist
\item
  Is that weird?
\end{itemize}
\end{frame}

\begin{frame}{Third Hypothesis: Understanding}
\protect\hypertarget{third-hypothesis-understanding}{}
An interesting idea to steer between these views was put forward in the
2000s by Oxford philosopher Alison Hills.

\begin{itemize}[<+->]
\tightlist
\item
  What goes wrong is that Bob doesn't get moral \textbf{understanding}.
\item
  Bob can know this way that meat-eating is wrong (assuming it is!), but
  can't understand why.
\end{itemize}
\end{frame}

\begin{frame}{Mathematics and Morality}
\protect\hypertarget{mathematics-and-morality-1}{}
At one level, Hills takes mathematics and morality to be alike in these
respects.

\begin{itemize}[<+->]
\tightlist
\item
  Testimony can give you mathematical knowledge (e.g., that Fermat's
  Last Theorem is true) without mathematical understanding.
\item
  And it can give you moral knowledge without moral understanding.
\end{itemize}
\end{frame}

\begin{frame}{Mathematics and Morality}
\protect\hypertarget{mathematics-and-morality-2}{}
The difference is in how much we (typically) care about these things.

\begin{itemize}[<+->]
\tightlist
\item
  There is something - though it's hard to say what - sub-optimal about
  a person who acts on moral knowledge they don't really understand.
\end{itemize}
\end{frame}

\begin{frame}{Mathematics and Morality}
\protect\hypertarget{mathematics-and-morality-3}{}
The difference is in how much we (typically) care about these things.

\begin{itemize}[<+->]
\tightlist
\item
  But unless you're a math prof, it is fine to know/use/repeat a lot of
  mathematical claims that you don't understand.
\end{itemize}
\end{frame}

\hypertarget{first-person-and-the-glance}{%
\section{First Person and the
Glance}\label{first-person-and-the-glance}}

\begin{frame}{About a Glance}
\protect\hypertarget{about-a-glance}{}
In math cases, it is a bit intuitive that the glance is linked to
understanding.

\begin{itemize}[<+->]
\tightlist
\item
  Here's a real-life case (similar to what I've been doing elsewhere in
  my job) that might help explain it.
\end{itemize}
\end{frame}

\begin{frame}{Preference Matching}
\protect\hypertarget{preference-matching}{}
Here's one thing you have to do in a university department.

\begin{itemize}[<+->]
\tightlist
\item
  Each year, you work out which courses have to be taught, which rooms
  are available to teach them in, and which people are available at
  which times to teach which courses (and how many courses they can
  teach).
\item
  There are a lot of variables, and there are too many to get a computer
  to easily scroll through all the options.
\end{itemize}
\end{frame}

\begin{frame}{Preference Matching}
\protect\hypertarget{preference-matching-1}{}
So we sit down with some combination of pen-and-paper, and computers, to
try to get everything to slot into place.

\begin{itemize}[<+->]
\tightlist
\item
  This usually involves doing something arbritrary for the first few
  slots (e.g., repeat some things from last year) and fitting the other
  pieces around like a jigsaw.
\end{itemize}
\end{frame}

\begin{frame}{Preference Matching}
\protect\hypertarget{preference-matching-2}{}
Imagine one year, we do this, and the last few choices are forced; it
looks like there is only one assignment that meets all the constraints.
(This is a good year; usually there are 0 and panic ensues.)

\begin{itemize}[<+->]
\tightlist
\item
  We try it a few different ways, and it's the same result.
\item
  At this point, we know there is only one arrangement of
  classes/people/times/rooms that works.
\end{itemize}
\end{frame}

\begin{frame}{Preference Matching}
\protect\hypertarget{preference-matching-3}{}
But that's not the same as \textbf{understanding} why there is only one
that works.

\begin{itemize}[<+->]
\tightlist
\item
  If I go back over and see ``Oh, Brian has these absurd constraints,
  and they intersect in weird ways with other constraints, and that's
  what rules most things out,'', now I might understand why there is
  only one.
\item
  This requires the Anselmian glance.
\end{itemize}
\end{frame}

\begin{frame}{Back to Pasnau}
\protect\hypertarget{back-to-pasnau}{}
Working through these two cases makes me think Pasnau is on to
something, though I think I disagree a bit about what he's on to.

\begin{itemize}[<+->]
\tightlist
\item
  Both the glance, and the first-personal privilege, are related to
  \textbf{understanding}.
\end{itemize}
\end{frame}

\begin{frame}{Understanding}
\protect\hypertarget{understanding}{}
With that in mind, let's look at his six way division of options.
\end{frame}

\hypertarget{the-six-cases}{%
\section{The Six Cases}\label{the-six-cases}}

\begin{frame}{Cases A-C}
\protect\hypertarget{cases-a-c}{}
\begin{enumerate}[<+->]
[A.]
\tightlist
\item
  Holding a whole argument in mind at once.
\item
  Holding the conclusion in mind and being able to produce the
  supporting argument at will.
\item
  Holding the conclusion in mind and being able to produce the
  supporting argument with effort.
\end{enumerate}
\end{frame}

\begin{frame}{Cases D-F}
\protect\hypertarget{cases-d-f}{}
\begin{enumerate}[<+->]
[A.]
\setcounter{enumi}{3}
\tightlist
\item
  Holding the conclusion in mind and remembering that the supporting
  argument was once grasped, but no longer being able to produce that
  argument, even with effort.
\item
  Holding the conclusion in mind without any memory of its evidential
  basis.
\item
  Having forgotten both the conclusion and its supporting argument.
\end{enumerate}
\end{frame}

\begin{frame}{Two Easy Cases}
\protect\hypertarget{two-easy-cases}{}
\begin{itemize}[<+->]
\tightlist
\item
  A is obviously understanding.
\item
  F is obviously not.
\end{itemize}
\end{frame}

\begin{frame}{Mathematics and Morality}
\protect\hypertarget{mathematics-and-morality-4}{}
I want to spend the rest of the time today going over case from both
math and ethics that are between B and E, and asking which of them are
really cases of understanding.
\end{frame}

\begin{frame}{Math Case}
\protect\hypertarget{math-case}{}
We already discussed this: Fermat's Last Theorem.

\begin{itemize}[<+->]
\tightlist
\item
  But I don't care just about Wiles here.
\item
  Anyone who ever followed the proof, and had something like the
  Anselmian glance of it, will do.
\end{itemize}
\end{frame}

\begin{frame}{Morality Case}
\protect\hypertarget{morality-case}{}
Rather than talking about meat-eating in general, imagine someone who
basically accepts the argument for ethical vegetarianism, but isn't sure
how far it goes.

\begin{itemize}[<+->]
\tightlist
\item
  In particular, they aren't sure whether oysters are animals in the
  relevant sense.
\item
  After much research, they decide oysters are not; they are plant-like
  enough to be ethically eaten.
\end{itemize}
\end{frame}

\begin{frame}{Morality Case}
\protect\hypertarget{morality-case-1}{}
And then the later states of them will involve having forgotten, either
temporarily or permanently, in whole or in part, the argument for why it
was ok to eat oysters.

\begin{itemize}[<+->]
\tightlist
\item
  When are they just like someone who learns it's ok to eat oysters from
  a reliable friend?
\end{itemize}
\end{frame}

\begin{frame}{My (Tentative) View}
\protect\hypertarget{my-tentative-view}{}
The big gap is between C and D.

C. Holding the conclusion in mind and being able to produce the
supporting argument with effort. D. Holding the conclusion in mind and
remembering that the supporting argument was once grasped, but no longer
being able to produce that argument, even with effort.
\end{frame}

\begin{frame}{Case C}
\protect\hypertarget{case-c}{}
\begin{itemize}[<+->]
\tightlist
\item
  In the math case, this person can tell us the proof, but it takes some
  time. Still, I think they more-or-less understand why the theorem is
  true.
\item
  In the morality case, this person can recall which features of oysters
  make them ok to eat, and why those features matter. That's a big
  difference from the testimony case.
\end{itemize}
\end{frame}

\begin{frame}{Case D}
\protect\hypertarget{case-d}{}
Is this enough for understanding in either case.

\begin{itemize}[<+->]
\tightlist
\item
  Doesn't seem close to it in the math case; I'd say in that case I used
  to understand it.
\item
  But maybe, maybe, the morality case is different.
\end{itemize}
\end{frame}

\begin{frame}{Two Oyster Examples}
\protect\hypertarget{two-oyster-examples}{}
\begin{itemize}[<+->]
\tightlist
\item
  Dave doesn't know anything about oysters, but trusts Carla and takes
  her word for it that it's ok to eat them.
\item
  Carla had thought about it, and decided it was ok to eat them, but now
  can't even remember if it was something unexpected about their
  biology, or a moral distinction she hadn't considered, that made her
  think that.
\item
  Are these the same?
\end{itemize}
\end{frame}

\begin{frame}{Summary}
\protect\hypertarget{summary}{}
Summary of this class:

\begin{itemize}[<+->]
\tightlist
\item
  The distinctions that Pasnau starts Ch 5 with are really interesting -
  as is (as always) the history of them.
\item
  But I think he's a bit negative about their relevance to epistemology.
\item
  They might not be relevant to \textbf{knowledge}, but they are
  relevant to \textbf{understanding}.
\end{itemize}
\end{frame}



\end{document}
