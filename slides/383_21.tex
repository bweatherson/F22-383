% Options for packages loaded elsewhere
\PassOptionsToPackage{unicode}{hyperref}
\PassOptionsToPackage{hyphens}{url}
%
\documentclass[
  17pt,
  letterpaper,
  ignorenonframetext,
  aspectratio=169,
  xcolor={dvipsnames}]{beamer}
\usepackage{pgfpages}
\setbeamertemplate{caption}[numbered]
\setbeamertemplate{caption label separator}{: }
\setbeamercolor{caption name}{fg=normal text.fg}
\beamertemplatenavigationsymbolsempty
% Prevent slide breaks in the middle of a paragraph
\widowpenalties 1 10000
\raggedbottom
\setbeamertemplate{part page}{
  \centering
  \begin{beamercolorbox}[sep=16pt,center]{part title}
    \usebeamerfont{part title}\insertpart\par
  \end{beamercolorbox}
}
\setbeamertemplate{section page}{
  \centering
  \begin{beamercolorbox}[sep=12pt,center]{part title}
    \usebeamerfont{section title}\insertsection\par
  \end{beamercolorbox}
}
\setbeamertemplate{subsection page}{
  \centering
  \begin{beamercolorbox}[sep=8pt,center]{part title}
    \usebeamerfont{subsection title}\insertsubsection\par
  \end{beamercolorbox}
}
\AtBeginPart{
  \frame{\partpage}
}
\AtBeginSection{
  \ifbibliography
  \else
    \frame{\sectionpage}
  \fi
}
\AtBeginSubsection{
  \frame{\subsectionpage}
}

\usepackage{amsmath,amssymb}
\usepackage{lmodern}
\usepackage{iftex}
\ifPDFTeX
  \usepackage[T1]{fontenc}
  \usepackage[utf8]{inputenc}
  \usepackage{textcomp} % provide euro and other symbols
\else % if luatex or xetex
  \usepackage{unicode-math}
  \defaultfontfeatures{Scale=MatchLowercase}
  \defaultfontfeatures[\rmfamily]{Ligatures=TeX,Scale=1}
  \setmainfont[BoldFont = SF Pro Text Semibold, Scale =
MatchLowercase]{SF Pro Text Light}
\fi
\usecolortheme{wolverine}
\usefonttheme{serif} % use mainfont rather than sansfont for slide text
\useinnertheme{default}
\useoutertheme{miniframes}
% Use upquote if available, for straight quotes in verbatim environments
\IfFileExists{upquote.sty}{\usepackage{upquote}}{}
\IfFileExists{microtype.sty}{% use microtype if available
  \usepackage[]{microtype}
  \UseMicrotypeSet[protrusion]{basicmath} % disable protrusion for tt fonts
}{}
\makeatletter
\@ifundefined{KOMAClassName}{% if non-KOMA class
  \IfFileExists{parskip.sty}{%
    \usepackage{parskip}
  }{% else
    \setlength{\parindent}{0pt}
    \setlength{\parskip}{6pt plus 2pt minus 1pt}}
}{% if KOMA class
  \KOMAoptions{parskip=half}}
\makeatother
\usepackage{xcolor}
\newif\ifbibliography
\setlength{\emergencystretch}{3em} % prevent overfull lines
\setcounter{secnumdepth}{-\maxdimen} % remove section numbering


\providecommand{\tightlist}{%
  \setlength{\itemsep}{0pt}\setlength{\parskip}{0pt}}\usepackage{longtable,booktabs,array}
\usepackage{calc} % for calculating minipage widths
\usepackage{caption}
% Make caption package work with longtable
\makeatletter
\def\fnum@table{\tablename~\thetable}
\makeatother
\usepackage{graphicx}
\makeatletter
\def\maxwidth{\ifdim\Gin@nat@width>\linewidth\linewidth\else\Gin@nat@width\fi}
\def\maxheight{\ifdim\Gin@nat@height>\textheight\textheight\else\Gin@nat@height\fi}
\makeatother
% Scale images if necessary, so that they will not overflow the page
% margins by default, and it is still possible to overwrite the defaults
% using explicit options in \includegraphics[width, height, ...]{}
\setkeys{Gin}{width=\maxwidth,height=\maxheight,keepaspectratio}
% Set default figure placement to htbp
\makeatletter
\def\fps@figure{htbp}
\makeatother

\captionsetup[figure]{labelformat=empty}
\usepackage{pgfpages}
\setbeamertemplate{itemize item}[circle]
\setbeamertemplate{footline}[frame number]{}
\mode<handout>{\pgfpagesuselayout{6 on 1}[letterpaper, border shrink=8mm]}
\AtBeginSection{%
   \begin{frame}
       \tableofcontents[currentsection]
   \end{frame}
}
\makeatletter
\makeatother
\makeatletter
\makeatother
\makeatletter
\@ifpackageloaded{caption}{}{\usepackage{caption}}
\AtBeginDocument{%
\ifdefined\contentsname
  \renewcommand*\contentsname{Table of contents}
\else
  \newcommand\contentsname{Table of contents}
\fi
\ifdefined\listfigurename
  \renewcommand*\listfigurename{List of Figures}
\else
  \newcommand\listfigurename{List of Figures}
\fi
\ifdefined\listtablename
  \renewcommand*\listtablename{List of Tables}
\else
  \newcommand\listtablename{List of Tables}
\fi
\ifdefined\figurename
  \renewcommand*\figurename{Figure}
\else
  \newcommand\figurename{Figure}
\fi
\ifdefined\tablename
  \renewcommand*\tablename{Table}
\else
  \newcommand\tablename{Table}
\fi
}
\@ifpackageloaded{float}{}{\usepackage{float}}
\floatstyle{ruled}
\@ifundefined{c@chapter}{\newfloat{codelisting}{h}{lop}}{\newfloat{codelisting}{h}{lop}[chapter]}
\floatname{codelisting}{Listing}
\newcommand*\listoflistings{\listof{codelisting}{List of Listings}}
\makeatother
\makeatletter
\@ifpackageloaded{caption}{}{\usepackage{caption}}
\@ifpackageloaded{subcaption}{}{\usepackage{subcaption}}
\makeatother
\makeatletter
\@ifpackageloaded{tcolorbox}{}{\usepackage[many]{tcolorbox}}
\makeatother
\makeatletter
\@ifundefined{shadecolor}{\definecolor{shadecolor}{rgb}{.97, .97, .97}}
\makeatother
\makeatletter
\makeatother
\ifLuaTeX
  \usepackage{selnolig}  % disable illegal ligatures
\fi
\IfFileExists{bookmark.sty}{\usepackage{bookmark}}{\usepackage{hyperref}}
\IfFileExists{xurl.sty}{\usepackage{xurl}}{} % add URL line breaks if available
\urlstyle{same} % disable monospaced font for URLs
\hypersetup{
  pdftitle={Knowledge and Reality, Lecture 21},
  pdfauthor={Brian Weatherson},
  hidelinks,
  pdfcreator={LaTeX via pandoc}}

\title{Knowledge and Reality, Lecture 21}
\author{Brian Weatherson}
\date{11/7/22}

\begin{document}
\frame{\titlepage}
\ifdefined\Shaded\renewenvironment{Shaded}{\begin{tcolorbox}[boxrule=0pt, interior hidden, borderline west={3pt}{0pt}{shadecolor}, frame hidden, breakable, enhanced, sharp corners]}{\end{tcolorbox}}\fi

\begin{frame}{Plan}
\protect\hypertarget{plan}{}
\begin{enumerate}[<+->]
\tightlist
\item
  Theories of perception and how Siegel's theory relates to them.
\item
  Setting up questions about inference.
\item
  Siegel's objection to taking theory
\end{enumerate}
\end{frame}

\hypertarget{four-theories-of-perceptual-justification}{%
\section{Four Theories of Perceptual
Justification}\label{four-theories-of-perceptual-justification}}

\begin{frame}{Four Theories}
\protect\hypertarget{four-theories}{}
\begin{enumerate}[<+->]
\tightlist
\item
  Disjunctivist/Naïve Realist
\item
  Reliabilist
\item
  Inferentialist
\item
  Dogmatist
\end{enumerate}
\end{frame}

\begin{frame}{Disjunctivist}
\protect\hypertarget{disjunctivist}{}
\begin{itemize}[<+->]
\tightlist
\item
  Appearances/experiences on their own have little epistemic charge.
\item
  What has power is \textbf{perception}, where this is understood as a
  success term.
\item
  This is very externalist; what is happening on an occasion, and what
  force it has, depends on external factors.
\end{itemize}
\end{frame}

\begin{frame}{Disjunctivist}
\protect\hypertarget{disjunctivist-1}{}
This is the natural theory to adopt if you have what Pasnau called a
\textbf{diaphonous} view of perception.

\begin{itemize}[<+->]
\tightlist
\item
  If what one sees is that there is a desk, not that one is having an
  experience of a desk, then it is natural to simply take the desk to be
  the inputs to later reasoning.
\end{itemize}
\end{frame}

\begin{frame}{Reliabilist}
\protect\hypertarget{reliabilist}{}
\begin{itemize}[<+->]
\tightlist
\item
  Anything can provide positive charge as long as it is reliably tied to
  reality.
\item
  Typically, experiences are reliably tied to reality.
\item
  There is nothing particularly special about perception.
\end{itemize}
\end{frame}

\begin{frame}{Inferentialist}
\protect\hypertarget{inferentialist}{}
\begin{itemize}[<+->]
\tightlist
\item
  On their own, experiences just provide positive charge for the
  proposition that one is having the experience.
\item
  Extra step needed to get to claims about the external world.
\item
  Lots of options for next step.
\end{itemize}
\end{frame}

\begin{frame}{Inferentialist}
\protect\hypertarget{inferentialist-1}{}
\begin{itemize}[<+->]
\tightlist
\item
  One choice: what is the link claim? Presumably something about
  reliable connection.
\item
  Second choice: how is the link claim grounded? IBE, Basic, something
  else?
\item
  Third choice: does the individual perceiver have to appreciate the
  ground?
\end{itemize}
\end{frame}

\begin{frame}{Dogmatist}
\protect\hypertarget{dogmatist}{}
\begin{itemize}[<+->]
\tightlist
\item
  In the first instance, experiences provide positive charge for the
  proposition that one is having the experience.
\item
  But unless something stops them, they also provide positive charge for
  external world propositions.
\item
  And the `something' has to be accessible to the perceiver.
\end{itemize}
\end{frame}

\begin{frame}{Dogmatist}
\protect\hypertarget{dogmatist-1}{}
The big difference with the inferentialist concerns presence vs absence
of reasons.

\begin{itemize}[<+->]
\tightlist
\item
  The inferentialist thinks you need a positive reason to go from Looks
  \emph{p} to \emph{p}.
\item
  The dogmatist thinks you need an absence of defeating reasons to go
  from Looks \emph{p} to \emph{p}.
\end{itemize}
\end{frame}

\begin{frame}{How They Play with Rationality of Perception}
\protect\hypertarget{how-they-play-with-rationality-of-perception}{}
What we want to do next is to see how plausible these theories are if we
think Siegel is right and perception itself can be rational or
irrational.

\begin{itemize}[<+->]
\tightlist
\item
  The argument for that is hardly complete, so this is a slightly weird
  thing to do at this stage.
\item
  And you could think any incompatibilities here are problems for
  Siegel.
\end{itemize}
\end{frame}

\begin{frame}{How They Play with Rationality of Perception}
\protect\hypertarget{how-they-play-with-rationality-of-perception-1}{}
Disjunctivism is no problem.

\begin{itemize}[<+->]
\tightlist
\item
  In the bad case you don't have real perception, just apparent
  perception.
\item
  So there isn't much charge there.
\end{itemize}
\end{frame}

\begin{frame}{How They Play with Rationality of Perception}
\protect\hypertarget{how-they-play-with-rationality-of-perception-2}{}
Reliabilism isn't much of a problem.

\begin{itemize}[<+->]
\tightlist
\item
  Provided we get the \textbf{reference class} right, the bad cases will
  be actually unreliable.
\item
  Bit of a trick here about getting the reference classes right, but not
  a big deal.
\end{itemize}
\end{frame}

\begin{frame}{How They Play with Rationality of Perception}
\protect\hypertarget{how-they-play-with-rationality-of-perception-3}{}
Inferentialism isn't much of a problem.

\begin{itemize}[<+->]
\tightlist
\item
  Provided the `link' is defeasible, and doesn't work in all cases, you
  can easily get that the support fails.
\end{itemize}
\end{frame}

\begin{frame}{How They Play with Rationality of Perception}
\protect\hypertarget{how-they-play-with-rationality-of-perception-4}{}
Dogmatism does look like a problem.

\begin{itemize}[<+->]
\tightlist
\item
  Hijacked perception lacks defeaters that are apparent to the
  perceiver.
\item
  So the dogmatist thinks they have full charge.
\item
  But they don't.
\end{itemize}
\end{frame}

\begin{frame}{Two Dogmatist Responses}
\protect\hypertarget{two-dogmatist-responses}{}
\begin{enumerate}[<+->]
\tightlist
\item
  Maybe the perceiver could tell there was a problem; this seems
  optimistic.
\item
  Maybe dogmatism just applies to a much narrower band of properties.
\end{enumerate}
\end{frame}

\begin{frame}{Dogmatism and Perception}
\protect\hypertarget{dogmatism-and-perception}{}
Most actual dogmatists don't think we really \textbf{perceive} things
like that something is a gun or a power-tool.

\begin{itemize}[<+->]
\tightlist
\item
  They think we just perceive things like shapes and colors.
\item
  And it's much harder for those to be hijacked.
\end{itemize}
\end{frame}

\hypertarget{what-is-inference}{%
\section{What is Inference}\label{what-is-inference}}

\begin{frame}{Inference}
\protect\hypertarget{inference}{}
We often form mental states by inference.

\begin{itemize}[<+->]
\tightlist
\item
  But what does this mean?
\end{itemize}
\end{frame}

\begin{frame}{Causation}
\protect\hypertarget{causation}{}
If state S2 is the result of inference from S1, then S1 must cause S2.

\begin{itemize}[<+->]
\tightlist
\item
  Inference doesn't just mean that S1 would have been a good reason to
  believe S2.
\item
  It means S2 must actually come from S1.
\end{itemize}
\end{frame}

\begin{frame}{Causation}
\protect\hypertarget{causation-1}{}
But causation is not sufficient.

\begin{itemize}[<+->]
\tightlist
\item
  Imagine I hear a song in the supermarket, and it reminds me of a beach
  holiday where we played that song a lot.
\item
  The hearing state causes the remembering state, but I don't
  \textbf{infer} anything about the holiday from the song.
\end{itemize}
\end{frame}

\begin{frame}{States Not Contents}
\protect\hypertarget{states-not-contents}{}
Informally, we can sometimes say things like this.

\begin{itemize}[<+->]
\tightlist
\item
  ``Holmes inferred that the butler did it from the fact that no one
  else had access to the murder weapon.''
\item
  This suggests that inference is a relation between facts.
\end{itemize}
\end{frame}

\begin{frame}{States Not Contents}
\protect\hypertarget{states-not-contents-1}{}
Inference can't be a relation between facts.

\begin{itemize}[<+->]
\tightlist
\item
  Sometimes we draw inferences from mistaken beliefs, and they aren't
  facts.
\item
  But there is a more subtle question that's harder to answer.
\end{itemize}
\end{frame}

\begin{frame}{States Not Contents}
\protect\hypertarget{states-not-contents-2}{}
Is Holmes's inference a relation between:

\begin{enumerate}[<+->]
\tightlist
\item
  The proposition that no one else had access to the murder weapon, and
  the proposition that the butler did it; or
\item
  Holmes's belief that no one else had access to the murder weapon, and
  his belief that the butler did it?
\end{enumerate}
\end{frame}

\begin{frame}{States Not Contents}
\protect\hypertarget{states-not-contents-3}{}
As the header might make clear, Siegel takes the answer to be 2.

\begin{itemize}[<+->]
\tightlist
\item
  And this makes sense if you care about causation.
\item
  The fact that no one else had access doesn't cause the butler's crime.
\item
  But the belief that no one else had access does cause Holmes's belief.
\end{itemize}
\end{frame}

\begin{frame}{Which States}
\protect\hypertarget{which-states}{}
If inference was tightly tied to contents, then it would be natural to
think that it primarily related \textbf{beliefs}.

\begin{itemize}[<+->]
\tightlist
\item
  This is \textbf{not} Siegel's view.
\item
  She thinks that any number of states can be related by inference.
\end{itemize}
\end{frame}

\begin{frame}{Which States}
\protect\hypertarget{which-states-1}{}
This could be an inference.

\begin{enumerate}[<+->]
\tightlist
\item
  Desire to have omelette for dinner.
\item
  Beliefs that (a) I'm out of eggs, (b) that the only nearby places to
  get eggs are Sparrow and the Co-op, and (c) roadworks make it
  unpleasant to go the Co-op.
\end{enumerate}

\begin{enumerate}[<+->]
[A.]
\setcounter{enumi}{2}
\tightlist
\item
  Intention to go to Sparrow.
\end{enumerate}
\end{frame}

\begin{frame}{A Test?}
\protect\hypertarget{a-test}{}
Here is a good heuristic for whether something is an inference.

\begin{itemize}[<+->]
\tightlist
\item
  If we can say it was \textbf{rational} or \textbf{irrational} to form
  the latter state because of the earlier state, it's an inference.
\end{itemize}
\end{frame}

\begin{frame}{A Test?}
\protect\hypertarget{a-test-1}{}
Here is a good heuristic for whether something is an inference.

\begin{itemize}[<+->]
\tightlist
\item
  It's not rational or irrational to remember a holiday on hearing a
  song; rationality doesn't care about those kinds of memories.
\end{itemize}
\end{frame}

\begin{frame}{A Test?}
\protect\hypertarget{a-test-2}{}
Siegel sometimes sounds like she is taking this to be more than a test,
but almost definitional of an inference.

\begin{itemize}[<+->]
\tightlist
\item
  I'm not sure if that's the right way to read her.
\item
  In any case, it doesn't sound that plausible to me; defining inference
  in terms of norms seems odd.
\end{itemize}
\end{frame}

\begin{frame}{The Subject Matter}
\protect\hypertarget{the-subject-matter}{}
What we really care about here is understanding whether a transition
between states is an inference, not whether it's a good or bad
inference.

\begin{itemize}[<+->]
\tightlist
\item
  But these questions can't really be 100\% separated.
\item
  Compare, understanding what a joke is can't be wholly separated from
  sorting jokes into good and bad.
\item
  A bad enough joke is not a joke.
\end{itemize}
\end{frame}

\hypertarget{the-taking-condition}{%
\section{The Taking Condition}\label{the-taking-condition}}

\begin{frame}{The Taking Condition}
\protect\hypertarget{the-taking-condition-1}{}
A lot of the contemporary literature on inferences revolves around this
theory.

\begin{itemize}[<+->]
\tightlist
\item
  A transition from S1 to S2 is an inference if and only if the thinker
  takes S1 to support S2.
\end{itemize}
\end{frame}

\begin{frame}{Virtues of This Theory}
\protect\hypertarget{virtues-of-this-theory}{}
It gets our canonical cases right.

\begin{itemize}[<+->]
\tightlist
\item
  Holmes does take his belief about weapons to support his belief about
  the butler's guilt.
\item
  The person hearing a song doesn't take this to support reminiscing
  about the beach holiday.
\end{itemize}
\end{frame}

\begin{frame}{Virtues of This Theory}
\protect\hypertarget{virtues-of-this-theory-1}{}
It allows for bad inferences.

\begin{itemize}[<+->]
\tightlist
\item
  This is often a challenge.
\item
  But as long as the person thinks they are making a good inference, the
  taking theory can say they are inferring.
\end{itemize}
\end{frame}

\begin{frame}{Why Siegel Has To Oppose It}
\protect\hypertarget{why-siegel-has-to-oppose-it}{}
\begin{itemize}[<+->]
\tightlist
\item
  She wants to say that hijacked experiences involve bad inferring.
\item
  But the taking condition clearly does not apply to experiences.
\item
  Vivek does not take his vanity to provide a reason to believe that
  everyone is happy!
\end{itemize}
\end{frame}

\begin{frame}{Two Objections}
\protect\hypertarget{two-objections}{}
\begin{enumerate}[<+->]
\tightlist
\item
  Regress
\item
  Over-intellectualisation
\end{enumerate}
\end{frame}

\begin{frame}{Regress}
\protect\hypertarget{regress}{}
Assume that to infer S2 from S1, another state T - the taking S1 to
provide a reason for S2 - is required.

\begin{itemize}[<+->]
\tightlist
\item
  Then S2 is inferred from S1 and T.
\item
  But then we need a T', taking S1 and T to support S2.
\item
  And then we'll need a T'`,taking S1, T and T' to support S2, etc.
\end{itemize}
\end{frame}

\begin{frame}{Regress}
\protect\hypertarget{regress-1}{}
Siegel does \textbf{not} endorse this objection.

\begin{itemize}[<+->]
\tightlist
\item
  And the first step on the previous slide does look a bit dodgy.
\item
  That's not to say the regress objection fails; it's an interesting
  challenge.
\end{itemize}
\end{frame}

\begin{frame}{Over-Intellectualisation (Toddlers version)}
\protect\hypertarget{over-intellectualisation-toddlers-version}{}
\begin{enumerate}[<+->]
\tightlist
\item
  Toddlers can make inferences.
\item
  Toddlers don't take things to provide reasons for other states.
\end{enumerate}

\begin{enumerate}[<+->]
[A.]
\setcounter{enumi}{2}
\tightlist
\item
  Inferences don't require takings.
\end{enumerate}
\end{frame}

\begin{frame}{Over-Intellectualisation (Toddlers version)}
\protect\hypertarget{over-intellectualisation-toddlers-version-1}{}
Again, this isn't quite Siegel's objection, though it's getting closer.

\begin{itemize}[<+->]
\tightlist
\item
  And I think it's fairly plausible.
\end{itemize}
\end{frame}

\begin{frame}{Four Cases from section 5.5}
\protect\hypertarget{four-cases-from-section-5.5}{}
\begin{enumerate}[<+->]
\tightlist
\item
  Kindness (categorisation)
\item
  Pepperoni (aggregated factors)
\item
  Too far north (spatio-temporal calculation)
\item
  Rockmouth (search for information)
\end{enumerate}
\end{frame}

\begin{frame}{Two Questions on Each}
\protect\hypertarget{two-questions-on-each}{}
\begin{enumerate}[<+->]
\tightlist
\item
  Are they actually examples of inferences?
\item
  Are they cases where the person does not, even tacitly, take states to
  provide reasons.
\end{enumerate}
\end{frame}

\begin{frame}{For Next Time}
\protect\hypertarget{for-next-time}{}
We'll look at the argument that hijacked experiences are inferential.
\end{frame}



\end{document}
