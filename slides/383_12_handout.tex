% Options for packages loaded elsewhere
\PassOptionsToPackage{unicode}{hyperref}
\PassOptionsToPackage{hyphens}{url}
%
\documentclass[
  17pt,
  letterpaper,
  ignorenonframetext,
  aspectratio=169,
  handout]{beamer}
\usepackage{pgfpages}
\setbeamertemplate{caption}[numbered]
\setbeamertemplate{caption label separator}{: }
\setbeamercolor{caption name}{fg=normal text.fg}
\beamertemplatenavigationsymbolsempty
% Prevent slide breaks in the middle of a paragraph
\widowpenalties 1 10000
\raggedbottom
\setbeamertemplate{part page}{
  \centering
  \begin{beamercolorbox}[sep=16pt,center]{part title}
    \usebeamerfont{part title}\insertpart\par
  \end{beamercolorbox}
}
\setbeamertemplate{section page}{
  \centering
  \begin{beamercolorbox}[sep=12pt,center]{part title}
    \usebeamerfont{section title}\insertsection\par
  \end{beamercolorbox}
}
\setbeamertemplate{subsection page}{
  \centering
  \begin{beamercolorbox}[sep=8pt,center]{part title}
    \usebeamerfont{subsection title}\insertsubsection\par
  \end{beamercolorbox}
}
\AtBeginPart{
  \frame{\partpage}
}
\AtBeginSection{
  \ifbibliography
  \else
    \frame{\sectionpage}
  \fi
}
\AtBeginSubsection{
  \frame{\subsectionpage}
}

\usepackage{amsmath,amssymb}
\usepackage{lmodern}
\usepackage{iftex}
\ifPDFTeX
  \usepackage[T1]{fontenc}
  \usepackage[utf8]{inputenc}
  \usepackage{textcomp} % provide euro and other symbols
\else % if luatex or xetex
  \usepackage{unicode-math}
  \defaultfontfeatures{Scale=MatchLowercase}
  \defaultfontfeatures[\rmfamily]{Ligatures=TeX,Scale=1}
  \setmainfont[BoldFont = SF Pro Text Semibold, Scale =
MatchLowercase]{SF Pro Text Light}
\fi
\usecolortheme{wolverine}
\usefonttheme{serif} % use mainfont rather than sansfont for slide text
\useinnertheme{default}
\useoutertheme{miniframes}
% Use upquote if available, for straight quotes in verbatim environments
\IfFileExists{upquote.sty}{\usepackage{upquote}}{}
\IfFileExists{microtype.sty}{% use microtype if available
  \usepackage[]{microtype}
  \UseMicrotypeSet[protrusion]{basicmath} % disable protrusion for tt fonts
}{}
\makeatletter
\@ifundefined{KOMAClassName}{% if non-KOMA class
  \IfFileExists{parskip.sty}{%
    \usepackage{parskip}
  }{% else
    \setlength{\parindent}{0pt}
    \setlength{\parskip}{6pt plus 2pt minus 1pt}}
}{% if KOMA class
  \KOMAoptions{parskip=half}}
\makeatother
\usepackage{xcolor}
\newif\ifbibliography
\setlength{\emergencystretch}{3em} % prevent overfull lines
\setcounter{secnumdepth}{-\maxdimen} % remove section numbering


\providecommand{\tightlist}{%
  \setlength{\itemsep}{0pt}\setlength{\parskip}{0pt}}\usepackage{longtable,booktabs,array}
\usepackage{calc} % for calculating minipage widths
\usepackage{caption}
% Make caption package work with longtable
\makeatletter
\def\fnum@table{\tablename~\thetable}
\makeatother
\usepackage{graphicx}
\makeatletter
\def\maxwidth{\ifdim\Gin@nat@width>\linewidth\linewidth\else\Gin@nat@width\fi}
\def\maxheight{\ifdim\Gin@nat@height>\textheight\textheight\else\Gin@nat@height\fi}
\makeatother
% Scale images if necessary, so that they will not overflow the page
% margins by default, and it is still possible to overwrite the defaults
% using explicit options in \includegraphics[width, height, ...]{}
\setkeys{Gin}{width=\maxwidth,height=\maxheight,keepaspectratio}
% Set default figure placement to htbp
\makeatletter
\def\fps@figure{htbp}
\makeatother

\captionsetup[figure]{labelformat=empty}
\usepackage{pgfpages}
\setbeamertemplate{itemize item}[circle]
\setbeamertemplate{footline}[frame number]{}
\mode<handout>{\pgfpagesuselayout{6 on 1}[letterpaper, border shrink=8mm]}
\AtBeginSection{%
   \begin{frame}
       \tableofcontents[currentsection]
   \end{frame}
}
\makeatletter
\makeatother
\makeatletter
\makeatother
\makeatletter
\@ifpackageloaded{caption}{}{\usepackage{caption}}
\AtBeginDocument{%
\ifdefined\contentsname
  \renewcommand*\contentsname{Table of contents}
\else
  \newcommand\contentsname{Table of contents}
\fi
\ifdefined\listfigurename
  \renewcommand*\listfigurename{List of Figures}
\else
  \newcommand\listfigurename{List of Figures}
\fi
\ifdefined\listtablename
  \renewcommand*\listtablename{List of Tables}
\else
  \newcommand\listtablename{List of Tables}
\fi
\ifdefined\figurename
  \renewcommand*\figurename{Figure}
\else
  \newcommand\figurename{Figure}
\fi
\ifdefined\tablename
  \renewcommand*\tablename{Table}
\else
  \newcommand\tablename{Table}
\fi
}
\@ifpackageloaded{float}{}{\usepackage{float}}
\floatstyle{ruled}
\@ifundefined{c@chapter}{\newfloat{codelisting}{h}{lop}}{\newfloat{codelisting}{h}{lop}[chapter]}
\floatname{codelisting}{Listing}
\newcommand*\listoflistings{\listof{codelisting}{List of Listings}}
\makeatother
\makeatletter
\@ifpackageloaded{caption}{}{\usepackage{caption}}
\@ifpackageloaded{subcaption}{}{\usepackage{subcaption}}
\makeatother
\makeatletter
\@ifpackageloaded{tcolorbox}{}{\usepackage[many]{tcolorbox}}
\makeatother
\makeatletter
\@ifundefined{shadecolor}{\definecolor{shadecolor}{rgb}{.97, .97, .97}}
\makeatother
\makeatletter
\makeatother
\ifLuaTeX
  \usepackage{selnolig}  % disable illegal ligatures
\fi
\IfFileExists{bookmark.sty}{\usepackage{bookmark}}{\usepackage{hyperref}}
\IfFileExists{xurl.sty}{\usepackage{xurl}}{} % add URL line breaks if available
\urlstyle{same} % disable monospaced font for URLs
\hypersetup{
  pdftitle={Knowledge and Reality, Lecture 12},
  pdfauthor={Brian Weatherson},
  hidelinks,
  pdfcreator={LaTeX via pandoc}}

\title{Knowledge and Reality, Lecture 12}
\author{Brian Weatherson}
\date{2022-10-10}

\begin{document}
\frame{\titlepage}
\ifdefined\Shaded\renewenvironment{Shaded}{\begin{tcolorbox}[interior hidden, sharp corners, borderline west={3pt}{0pt}{shadecolor}, frame hidden, breakable, enhanced, boxrule=0pt]}{\end{tcolorbox}}\fi

\hypertarget{compulsion}{%
\section{Compulsion}\label{compulsion}}

\begin{frame}{Compulsion}
\protect\hypertarget{compulsion-1}{}
Especially in the first half of chapter 2, it was striking how much
focus there was on compulsion.

\begin{itemize}[<+->]
\tightlist
\item
  There was for a long time a real focus on things one can't help
  believe, or inferences one can't help but follow.
\end{itemize}
\end{frame}

\begin{frame}{Compulsion}
\protect\hypertarget{compulsion-2}{}
Sometimes this is the goal of a theory.

\begin{itemize}[<+->]
\tightlist
\item
  See, for example, the discussion of how the existence of disagreement
  is a sign things are bad, because you haven't come up with
  \textbf{compelling} reasons.
\end{itemize}
\end{frame}

\begin{frame}{Compulsion}
\protect\hypertarget{compulsion-3}{}
But it's always taken to be something good.

\begin{itemize}[<+->]
\tightlist
\item
  Having \textbf{indubitable}, literally, cannot be doubted, reasons is
  taken to be a good thing.
\end{itemize}
\end{frame}

\begin{frame}{Epistemology as Normative}
\protect\hypertarget{epistemology-as-normative}{}
This is an interesting contrast with a view of epistemology where it is
something that people do better or worse at.

\begin{itemize}[<+->]
\tightlist
\item
  If the good steps are ones that literally everyone will be compelled
  to do, that picture that some people are good at reasoning feels
  mistaken.
\end{itemize}
\end{frame}

\begin{frame}{Compelling Inferences}
\protect\hypertarget{compelling-inferences}{}
But there is something to do the idea that really strong inferences are
compelling. Consider this inference.

\begin{enumerate}[<+->]
\tightlist
\item
  \emph{x} equals 22 times 18.
\item
  22 times 18 equals 396.
\item
  Therefore, \emph{x} equals 396.
\end{enumerate}

\begin{itemize}[<+->]
\tightlist
\item
  It is really hard to doubt this!
\end{itemize}
\end{frame}

\begin{frame}{Compulsion and Necessity}
\protect\hypertarget{compulsion-and-necessity}{}
Note that this inference is dubitable, even though given the premise,
the conclusion has to be true.

\begin{enumerate}[<+->]
\tightlist
\item
  \emph{x} equals 22 times 18.
\item
  Therefore, \emph{x} equals 396.
\end{enumerate}
\end{frame}

\begin{frame}{Compulsion}
\protect\hypertarget{compulsion-4}{}
\begin{itemize}[<+->]
\tightlist
\item
  Maybe there is something to the idea that we want all our reasoning to
  be like the first one.
\end{itemize}
\end{frame}

\hypertarget{perfection}{%
\section{Perfection}\label{perfection}}

\begin{frame}{Two Notions of Perfection}
\protect\hypertarget{two-notions-of-perfection}{}
\begin{enumerate}[<+->]
\tightlist
\item
  Global perfection. Being an epistemic god.
\item
  Local perfection. Having the same relation to a particular proposition
  that a god does.
\end{enumerate}
\end{frame}

\begin{frame}{Local Traffic Only?}
\protect\hypertarget{local-traffic-only}{}
Question

\begin{itemize}[<+->]
\tightlist
\item
  Is it possible to be locally perfect without being globally perfect?
\end{itemize}
\end{frame}

\begin{frame}{Local Traffic Only?}
\protect\hypertarget{local-traffic-only-1}{}
Some traditions say yes.

\begin{itemize}[<+->]
\tightlist
\item
  A pramana, after all, is a proof.
\item
  Whoever believes \emph{p} on the basis of a pramana is, locally,
  perfect.
\item
  Though note this isn't something you'd expect all Indian philosophers
  to accept.
\end{itemize}
\end{frame}

\begin{frame}{Global Priority}
\protect\hypertarget{global-priority}{}
But other traditions say no.

\begin{itemize}[<+->]
\tightlist
\item
  If a person believes anything on the basis of less than perfect
  reasons, that shows they are unreliable.
\item
  Both Aristotle and Descartes, in very different ways, made it hard to
  be locally but not globally perfect.
\end{itemize}
\end{frame}

\begin{frame}{Global Priority}
\protect\hypertarget{global-priority-1}{}
And note that whatever is common ground to Aristotle and Descartes ends
up being very important to the kind of tradition we're mostly in.

\begin{itemize}[<+->]
\tightlist
\item
  The things they agree on can seem not even up for debate sometimes.
\item
  But whether one can be locally perfect without being globally perfect
  should be up for debate.
\end{itemize}
\end{frame}

\hypertarget{moral-certainty}{%
\section{Moral Certainty}\label{moral-certainty}}

\begin{frame}{Moral Certainty}
\protect\hypertarget{moral-certainty-1}{}
There is a lot to say here, and I could spend literally weeks going over
just this notion.

\begin{itemize}[<+->]
\tightlist
\item
  But that wouldn't be fun for anyone, so I'll just note two points
  about it.
\end{itemize}
\end{frame}

\begin{frame}{Variable Standards}
\protect\hypertarget{variable-standards}{}
How much evidence do you need for moral certainty?

\begin{itemize}[<+->]
\tightlist
\item
  Answer: It depends on the question.
\item
  To have moral certainty that someone is guilty of murder, a huge
  amount.
\item
  To have moral certainty that it is about to rain, not so much.
\end{itemize}
\end{frame}

\begin{frame}{Practicality}
\protect\hypertarget{practicality}{}
\begin{itemize}[<+->]
\item
  What is moral certainty for?
\item
  What is certainty for?
\item
  What is knowledge for?
\item
  If you start epistemology with moral certainty, it naturally becomes a
  very practical subject. That's very different to what happens if you
  start with Aristotelian episteme.
\end{itemize}
\end{frame}

\hypertarget{the-trilemma}{%
\section{The Trilemma}\label{the-trilemma}}

\begin{frame}{Stating the Trilemma}
\protect\hypertarget{stating-the-trilemma}{}
\begin{enumerate}[<+->]
\tightlist
\item
  Proportionality
\item
  Pessimism
\item
  Absolute Belief
\end{enumerate}
\end{frame}

\begin{frame}{Proportionality}
\protect\hypertarget{proportionality}{}
The strength of one's belief should be proportional to the evidence.

\begin{itemize}[<+->]
\tightlist
\item
  So if one gets better evidence, one's belief should be stronger.
\end{itemize}
\end{frame}

\begin{frame}{Pessimism}
\protect\hypertarget{pessimism}{}
It is never possible to get certainty.

\begin{itemize}[<+->]
\tightlist
\item
  So it is always possible to get evidence that puts us in a better
  position, i.e., closer to certainty.
\end{itemize}
\end{frame}

\begin{frame}{Absolute Belief}
\protect\hypertarget{absolute-belief}{}
We often have absolute belief, or full belief, in propositions.

\begin{itemize}[<+->]
\tightlist
\item
  We don't just think that it's very likely October right now, we simply
  take it as a fixed point in our reasoning that it is.
\item
  Even when we have probabilistic beliefs, these have to be based on
  something, and things like \emph{It's October} are among those things.
\end{itemize}
\end{frame}

\begin{frame}{The Challenge}
\protect\hypertarget{the-challenge}{}
\begin{itemize}[<+->]
\tightlist
\item
  Start with something we have absolute belief in. (By \textbf{Absolute
  belief} such a thing exists.)
\item
  By \textbf{pessimism} we could get better evidence for it. So imagine
  we do.
\item
  We can't strengthen our belief in it, because it was already absolute.
\item
  So we'll violate \textbf{proportionality}.
\end{itemize}
\end{frame}

\begin{frame}{Another Challenge}
\protect\hypertarget{another-challenge}{}
\begin{itemize}[<+->]
\tightlist
\item
  Weaken \textbf{pessimism} so it doesn't say we never get certainty,
  but that we rarely do.
\item
  Strengthen \textbf{absolute belief} so it says that there are more
  than a few things we believe absolutely.
\item
  The contradiction still goes through.
\end{itemize}
\end{frame}

\begin{frame}{Two Virtues}
\protect\hypertarget{two-virtues}{}
I think this is a great framing of a key problem.

\begin{itemize}[<+->]
\tightlist
\item
  It's really interesting to think through how different thinkers over
  time navigated it. (Even if they didn't put it this way.)
\item
  And it's really interesting to think how we should navigate it.
\end{itemize}
\end{frame}

\begin{frame}{Anti-Proportionists}
\protect\hypertarget{anti-proportionists}{}
\begin{itemize}[<+->]
\tightlist
\item
  Orthodox 20C Anglophone epistemologists.
\item
  It's fine to absolutely believe that it's raining in downtown AA iff
  we can see rain from the window here.
\item
  But we could get even better evidence for that.
\item
  Is this a new view in 17C Western Europe? Maybe! (Though I'd want to
  know more about Chinese traditions to be sure.)
\end{itemize}
\end{frame}

\begin{frame}{Anti-Pessimists}
\protect\hypertarget{anti-pessimists}{}
\begin{itemize}[<+->]
\tightlist
\item
  Descartes! (Eventually, but only for people who have read and accepted
  Descartes.)
\item
  Most classic Indian philosophers; a pramana is an absolute proof.
\item
  Some contemporary western philosophers, especially about direct
  perception.
\end{itemize}
\end{frame}

\begin{frame}{Anti-Absolutists}
\protect\hypertarget{anti-absolutists}{}
\begin{itemize}[<+->]
\tightlist
\item
  Bayesians!
\item
  And maybe, though they didn't have the math to make it rigorous, most
  Western pre-modern epistemologists.
\end{itemize}
\end{frame}

\hypertarget{anti-proportionality}{%
\section{Anti-Proportionality}\label{anti-proportionality}}

\begin{frame}{The Big Challenges}
\protect\hypertarget{the-big-challenges}{}
\begin{itemize}[<+->]
\tightlist
\item
  When is it ok to fully believe?
\item
  How can the line be anything more than an arbitrary boundary?
\item
  Should be boundary be relevant to practical concerns, like with moral
  certainty, or not.
\end{itemize}
\end{frame}

\begin{frame}{Bernoulli}
\protect\hypertarget{bernoulli}{}
There is something absurd about the idea that absolute belief is
warranted at 99 out of 100.

\begin{itemize}[<+->]
\tightlist
\item
  Problem one: lotteries.
\item
  Problem two: long-shot dangers. Don't cross roads you have a 199 in
  200 chance of crossing; you'll be dead within a month.
\item
  Problem three: not practically sensitive.
\end{itemize}
\end{frame}

\hypertarget{anti-absolute}{%
\section{Anti-Absolute}\label{anti-absolute}}

\begin{frame}{Challenge One}
\protect\hypertarget{challenge-one}{}
It would be good to have a mathematical model of what belief looks like
on this picture.

\begin{itemize}[<+->]
\tightlist
\item
  Happily we now have one: the probability calculus.
\item
  Is this a good enough model? Eh, it's not bad.
\end{itemize}
\end{frame}

\begin{frame}{Challenge Two}
\protect\hypertarget{challenge-two}{}
What are conversations like on this picture?

\begin{itemize}[<+->]
\tightlist
\item
  A asks ``Where is the cat?''
\item
  B says ``Probability 0.98 that she's on the mat, probability 0.01 that
  she's run under the couch, probability 0.09 that she's run downstairs,
  probability something that she's vanished into thin air, \ldots{}''
\end{itemize}
\end{frame}

\begin{frame}{Challenge Two}
\protect\hypertarget{challenge-two-1}{}
You need something that licences ``She's on the mat''.

\begin{itemize}[<+->]
\tightlist
\item
  And that will recreate all the problems from anti-proportionality.
\end{itemize}
\end{frame}

\begin{frame}{Challenge Three}
\protect\hypertarget{challenge-three}{}
How do you update?

\begin{itemize}[<+->]
\tightlist
\item
  The Bayesians have a mathematical theory of what to do when you get
  evidence \emph{E}.
\item
  But what does that even mean?
\item
  If we can't get certainty, why think we can get evidence?
\end{itemize}
\end{frame}

\begin{frame}{Compulsion}
\protect\hypertarget{compulsion-5}{}
One possible answer.

\begin{itemize}[<+->]
\tightlist
\item
  We get evidence \emph{E} when we are compelled to treat it as fixed.
\end{itemize}
\end{frame}

\begin{frame}{For Next Time}
\protect\hypertarget{for-next-time}{}
We'll come back to that last question, so I'll leave it with that
dangling thought for now.

\begin{itemize}[<+->]
\tightlist
\item
  Next time, chapter 3.
\end{itemize}

Pasnau, Robert. After Certainty . OUP Oxford. Kindle Edition.

The trilemma - the three things to focus on - Describe - Why are they at
issue - Examples of violations of each

Full belief - when is it ok - Problem of arbitrariness - Practical or
impractical - See Bernoulli on probabilities

No full belief - need a mathematical model - We now have one the
probability calculus - How do you talk? - How do you update? - One
option: compulsion
\end{frame}



\end{document}
