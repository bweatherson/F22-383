% Options for packages loaded elsewhere
\PassOptionsToPackage{unicode}{hyperref}
\PassOptionsToPackage{hyphens}{url}
%
\documentclass[
  17pt,
  letterpaper,
  ignorenonframetext,
  aspectratio=169,
  xcolor={dvipsnames}]{beamer}
\usepackage{pgfpages}
\setbeamertemplate{caption}[numbered]
\setbeamertemplate{caption label separator}{: }
\setbeamercolor{caption name}{fg=normal text.fg}
\beamertemplatenavigationsymbolsempty
% Prevent slide breaks in the middle of a paragraph
\widowpenalties 1 10000
\raggedbottom
\setbeamertemplate{part page}{
  \centering
  \begin{beamercolorbox}[sep=16pt,center]{part title}
    \usebeamerfont{part title}\insertpart\par
  \end{beamercolorbox}
}
\setbeamertemplate{section page}{
  \centering
  \begin{beamercolorbox}[sep=12pt,center]{part title}
    \usebeamerfont{section title}\insertsection\par
  \end{beamercolorbox}
}
\setbeamertemplate{subsection page}{
  \centering
  \begin{beamercolorbox}[sep=8pt,center]{part title}
    \usebeamerfont{subsection title}\insertsubsection\par
  \end{beamercolorbox}
}
\AtBeginPart{
  \frame{\partpage}
}
\AtBeginSection{
  \ifbibliography
  \else
    \frame{\sectionpage}
  \fi
}
\AtBeginSubsection{
  \frame{\subsectionpage}
}

\usepackage{amsmath,amssymb}
\usepackage{lmodern}
\usepackage{iftex}
\ifPDFTeX
  \usepackage[T1]{fontenc}
  \usepackage[utf8]{inputenc}
  \usepackage{textcomp} % provide euro and other symbols
\else % if luatex or xetex
  \usepackage{unicode-math}
  \defaultfontfeatures{Scale=MatchLowercase}
  \defaultfontfeatures[\rmfamily]{Ligatures=TeX,Scale=1}
  \setmainfont[BoldFont = SF Pro Text Semibold, Scale =
MatchLowercase]{SF Pro Text Light}
\fi
\usecolortheme{wolverine}
\usefonttheme{serif} % use mainfont rather than sansfont for slide text
\useinnertheme{default}
\useoutertheme{miniframes}
% Use upquote if available, for straight quotes in verbatim environments
\IfFileExists{upquote.sty}{\usepackage{upquote}}{}
\IfFileExists{microtype.sty}{% use microtype if available
  \usepackage[]{microtype}
  \UseMicrotypeSet[protrusion]{basicmath} % disable protrusion for tt fonts
}{}
\makeatletter
\@ifundefined{KOMAClassName}{% if non-KOMA class
  \IfFileExists{parskip.sty}{%
    \usepackage{parskip}
  }{% else
    \setlength{\parindent}{0pt}
    \setlength{\parskip}{6pt plus 2pt minus 1pt}}
}{% if KOMA class
  \KOMAoptions{parskip=half}}
\makeatother
\usepackage{xcolor}
\newif\ifbibliography
\setlength{\emergencystretch}{3em} % prevent overfull lines
\setcounter{secnumdepth}{-\maxdimen} % remove section numbering


\providecommand{\tightlist}{%
  \setlength{\itemsep}{0pt}\setlength{\parskip}{0pt}}\usepackage{longtable,booktabs,array}
\usepackage{calc} % for calculating minipage widths
\usepackage{caption}
% Make caption package work with longtable
\makeatletter
\def\fnum@table{\tablename~\thetable}
\makeatother
\usepackage{graphicx}
\makeatletter
\def\maxwidth{\ifdim\Gin@nat@width>\linewidth\linewidth\else\Gin@nat@width\fi}
\def\maxheight{\ifdim\Gin@nat@height>\textheight\textheight\else\Gin@nat@height\fi}
\makeatother
% Scale images if necessary, so that they will not overflow the page
% margins by default, and it is still possible to overwrite the defaults
% using explicit options in \includegraphics[width, height, ...]{}
\setkeys{Gin}{width=\maxwidth,height=\maxheight,keepaspectratio}
% Set default figure placement to htbp
\makeatletter
\def\fps@figure{htbp}
\makeatother

\captionsetup[figure]{labelformat=empty}
\usepackage{pgfpages}
\setbeamertemplate{itemize item}[circle]
\setbeamertemplate{footline}[frame number]{}
\mode<handout>{\pgfpagesuselayout{6 on 1}[letterpaper, border shrink=8mm]}
\AtBeginSection{%
   \begin{frame}
       \tableofcontents[currentsection]
   \end{frame}
}
\makeatletter
\makeatother
\makeatletter
\makeatother
\makeatletter
\@ifpackageloaded{caption}{}{\usepackage{caption}}
\AtBeginDocument{%
\ifdefined\contentsname
  \renewcommand*\contentsname{Table of contents}
\else
  \newcommand\contentsname{Table of contents}
\fi
\ifdefined\listfigurename
  \renewcommand*\listfigurename{List of Figures}
\else
  \newcommand\listfigurename{List of Figures}
\fi
\ifdefined\listtablename
  \renewcommand*\listtablename{List of Tables}
\else
  \newcommand\listtablename{List of Tables}
\fi
\ifdefined\figurename
  \renewcommand*\figurename{Figure}
\else
  \newcommand\figurename{Figure}
\fi
\ifdefined\tablename
  \renewcommand*\tablename{Table}
\else
  \newcommand\tablename{Table}
\fi
}
\@ifpackageloaded{float}{}{\usepackage{float}}
\floatstyle{ruled}
\@ifundefined{c@chapter}{\newfloat{codelisting}{h}{lop}}{\newfloat{codelisting}{h}{lop}[chapter]}
\floatname{codelisting}{Listing}
\newcommand*\listoflistings{\listof{codelisting}{List of Listings}}
\makeatother
\makeatletter
\@ifpackageloaded{caption}{}{\usepackage{caption}}
\@ifpackageloaded{subcaption}{}{\usepackage{subcaption}}
\makeatother
\makeatletter
\@ifpackageloaded{tcolorbox}{}{\usepackage[many]{tcolorbox}}
\makeatother
\makeatletter
\@ifundefined{shadecolor}{\definecolor{shadecolor}{rgb}{.97, .97, .97}}
\makeatother
\makeatletter
\makeatother
\ifLuaTeX
  \usepackage{selnolig}  % disable illegal ligatures
\fi
\IfFileExists{bookmark.sty}{\usepackage{bookmark}}{\usepackage{hyperref}}
\IfFileExists{xurl.sty}{\usepackage{xurl}}{} % add URL line breaks if available
\urlstyle{same} % disable monospaced font for URLs
\hypersetup{
  pdftitle={Knowledge and Reality, Lecture 25},
  pdfauthor={Brian Weatherson},
  hidelinks,
  pdfcreator={LaTeX via pandoc}}

\title{Knowledge and Reality, Lecture 25}
\author{Brian Weatherson}
\date{11/30/22}

\begin{document}
\frame{\titlepage}
\ifdefined\Shaded\renewenvironment{Shaded}{\begin{tcolorbox}[enhanced, frame hidden, breakable, borderline west={3pt}{0pt}{shadecolor}, sharp corners, interior hidden, boxrule=0pt]}{\end{tcolorbox}}\fi

\hypertarget{racial-misperceptions}{%
\section{Racial Misperceptions}\label{racial-misperceptions}}

\begin{frame}{Big Picture}
\protect\hypertarget{big-picture}{}
\begin{itemize}[<+->]
\tightlist
\item
  There are a lot of racial misperceptions.
\item
  These are downstream of widely shared racial (well, racist) beliefs
  that are widely shared in the community.
\item
  At least in many (most?) cases, there is enough individual culpability
  in acquiring or maintaining these background beliefs that the
  perceptions are not well-founded.
\end{itemize}
\end{frame}

\begin{frame}{Racial Misperceptions}
\protect\hypertarget{racial-misperceptions-1}{}
Note how many, and how varied, the misperceptions at the start of the
chapter are.

\begin{itemize}[<+->]
\tightlist
\item
  They include seemingly innocuous things like age.
\end{itemize}
\end{frame}

\begin{frame}{Why So Many}
\protect\hypertarget{why-so-many}{}
Siegel is responding to a (possible) criticism that some of these
misperceptions might be grounded in accurate beliefs about racial
disparities.
\end{frame}

\begin{frame}{Why So Many}
\protect\hypertarget{why-so-many-1}{}
\begin{itemize}[<+->]
\tightlist
\item
  Some of the explaining away here seems to rely on bad inferences,
  e.g., from Most Fs are Gs to Most Gs are Fs.
\item
  But in some of the experiments, it's hard to see what beliefs about
  the world could make the beliefs about ages justified.
\end{itemize}
\end{frame}

\begin{frame}{Minimal Connections}
\protect\hypertarget{minimal-connections}{}
The range of experiments also helps respond to another kind of concern.

\begin{itemize}[<+->]
\tightlist
\item
  Imagine that a person has developed a kind of association between
  `black' and `dangerous' like the association between `salt' and
  `pepper'.
\end{itemize}
\end{frame}

\begin{frame}{Minimal Connections}
\protect\hypertarget{minimal-connections-1}{}
The range of experiments also helps respond to another kind of concern.

\begin{itemize}[<+->]
\tightlist
\item
  That looks pretty dubious morally, arguably worse than on Siegel's
  positive view, but it's not obviously within the range of epistemic
  evaluation.
\end{itemize}
\end{frame}

\begin{frame}{Minimal Connections}
\protect\hypertarget{minimal-connections-2}{}
Siegel's theory is that perceptions are irrational because they are bad
inferences.

\begin{itemize}[<+->]
\tightlist
\item
  Whatever inferences are, they are richer than the connection between
  `salt' and `pepper'.
\item
  So she needs to rule out the possibility that there is the same kind
  of connection here.
\end{itemize}
\end{frame}

\begin{frame}{Minimal Connections}
\protect\hypertarget{minimal-connections-3}{}
I was a little unsure why the association picture would fail to explain
most of these experimental results.

\begin{itemize}[<+->]
\tightlist
\item
  But it's really hard to see how it helps with the age test.
\item
  And more generally, having a broader range of data helps to make it
  harder for an opponent.
\end{itemize}
\end{frame}

\begin{frame}{Testimony}
\protect\hypertarget{testimony}{}
But the opponent Siegel spends the most time on concedes that the
background beliefs are false.

\begin{itemize}[<+->]
\tightlist
\item
  They argue that the inferences are nonetheless rational (or
  well-founded) because the false beliefs were formed in a reasonable
  way.
\item
  And that reasonable way is testimony from trusted sources.
\end{itemize}
\end{frame}

\begin{frame}{Testimony}
\protect\hypertarget{testimony-1}{}
There is one defence Siegel could offer here, but does not. In fact, she
expressly rejects it.

\begin{itemize}[<+->]
\tightlist
\item
  It is what we might call the transfer model of testimony.
\item
  Testimony only ever transfers the
  rationality/reasonableness/well-foundedness of a belief from
  speaker(s) to hearer(s).
\end{itemize}
\end{frame}

\begin{frame}{Testimony}
\protect\hypertarget{testimony-2}{}
There is one defence Siegel could offer here, but does not. In fact, she
expressly rejects it.

\begin{itemize}[<+->]
\tightlist
\item
  So if the initial beliefs are ill-founded, as they are here, so will
  the subsequent beliefs be.
\end{itemize}
\end{frame}

\begin{frame}{Testimony}
\protect\hypertarget{testimony-3}{}
Siegel rejects the transfer model because of the example of the
well-meaning mother.

\begin{itemize}[<+->]
\tightlist
\item
  A child is entitled to trust their mother's safety advice, even if it
  turns out to be false and ill-founded.
\end{itemize}
\end{frame}

\begin{frame}{Contrasts}
\protect\hypertarget{contrasts}{}
What are the contrasts between this case and casually absorbing racist
beliefs?

\begin{enumerate}[<+->]
\tightlist
\item
  The mother is well-meaning and has the child's best interests at
  heart.
\item
  The racist beliefs require poor maintenance to be sustained.
\end{enumerate}
\end{frame}

\begin{frame}{Contrasts}
\protect\hypertarget{contrasts-1}{}
The first of these is not particularly compelling.

\begin{itemize}[<+->]
\tightlist
\item
  For one thing, what really matters is whether the person seems
  well-meaning, not whether they are.
\item
  For another, it isn't clear that racists spreading racist beliefs to
  people like them are not well-meaning, in the sense of trying to
  improve (by their lights) the well-being of their audience.
\end{itemize}
\end{frame}

\begin{frame}{Contrasts}
\protect\hypertarget{contrasts-2}{}
What about the second of these?

\begin{itemize}[<+->]
\tightlist
\item
  It seems fairly contingent at best that the belief requires poor
  maintenance.
\item
  Someone who grows up in a very homogenous (and racist) community won't
  have much opportunity to do any useful maintenance.
\end{itemize}
\end{frame}

\begin{frame}{Vigilance}
\protect\hypertarget{vigilance}{}
A better model might be that the person who simply absorbs racist
beliefs is (in most realistic situations) not going to be particularly
vigilant.

\begin{itemize}[<+->]
\tightlist
\item
  And that might be a difference with the case of the misleading mother.
\end{itemize}
\end{frame}

\hypertarget{pautzs-criticisms}{%
\section{Pautz's Criticisms}\label{pautzs-criticisms}}

\begin{frame}
Case 1

\begin{itemize}[<+->]
\tightlist
\item
  Person expects to see a red round thing when they enter the room, for
  no good reason at all.
\item
  When they enter the room, they hallucinate a tomato, because of this
  hallucination.
\item
  Despite knowing they had this expectation, they believe there is a
  tomato there.
\end{itemize}
\end{frame}

\begin{frame}
Case 2 (Pautz's Case)

\begin{itemize}[<+->]
\tightlist
\item
  Just like case one, except person forgets that they had this
  expectation.
\item
  They have no reason at all to think they are hallucinating.
\item
  They also believe there is a tomato in the room.
\end{itemize}
\end{frame}

\begin{frame}{Pautz's Arguments}
\protect\hypertarget{pautzs-arguments}{}
Two claims really, that are worth separating.

\begin{enumerate}[<+->]
\tightlist
\item
  It is intuitively obvious that the belief is reasonable in this case.
\item
  If you have a reason to believe something, and no reason to reject it,
  you should believe it.
\end{enumerate}
\end{frame}

\begin{frame}{Forgetting}
\protect\hypertarget{forgetting}{}
The second argument seems too strong as it stands. Example: A bill (call
it Bill) is being debated in Congress.

\begin{itemize}[<+->]
\tightlist
\item
  Person believes that Democrats support it on very weak evidence.
\item
  Next day person has forgotten their evidence, but not their belief,
  and learns Republicans support Bill.
\end{itemize}
\end{frame}

\begin{frame}{Forgetting}
\protect\hypertarget{forgetting-1}{}
The second argument seems too strong as it stands. Example: A bill (call
it Bill) is being debated in Congress.

\begin{itemize}[<+->]
\tightlist
\item
  Person concludes Democrats and Republicans have same view on Bill.
\end{itemize}
\end{frame}

\begin{frame}{Forgetting}
\protect\hypertarget{forgetting-2}{}
This seems maybe not particularly reasonable.

\begin{itemize}[<+->]
\tightlist
\item
  It would be something like epistemic laundering if it were reasonable.
\item
  But also there's nothing internal to Person \emph{now} that would tell
  them they are being unreasonable.
\end{itemize}
\end{frame}

\begin{frame}{Forgetting}
\protect\hypertarget{forgetting-3}{}
A lot of what Pautz will argue turns, I think, on this kind of
principle.

\begin{itemize}[<+->]
\tightlist
\item
  If someone is being unreasonable, they could figure out that they are
  being unreasonable by careful introspection.
\item
  In Siegel's cases, that's impossible.
\item
  But maybe it's impossible in forgetting cases.
\end{itemize}
\end{frame}

\begin{frame}{Evil Demons}
\protect\hypertarget{evil-demons}{}
This case is, I think, the core of Pautz's criticism of Siegel.

\begin{itemize}[<+->]
\tightlist
\item
  Start with a case that Siegel would describe as hijacked experience.
\item
  After 30 seconds, without the phenomenal feel changing, it becomes a
  case of evil demon deception.
\item
  Does it become more reasonable to trust appearances when that happens?
\end{itemize}
\end{frame}

\begin{frame}{Evil Demons}
\protect\hypertarget{evil-demons-1}{}
Imagine that the person does trust the appearances all along.

\begin{itemize}[<+->]
\tightlist
\item
  Siegel says that they are being rational before the demon turns up,
  but irrational afterwards.
\item
  But it seems from the inside they are doing the same thing.
\end{itemize}
\end{frame}

\begin{frame}{Evidence}
\protect\hypertarget{evidence}{}
There is one other big point that runs through Pautz's piece that I
think is worth having on the table.

\begin{itemize}[<+->]
\tightlist
\item
  What is perceptual evidence?
\end{itemize}
\end{frame}

\begin{frame}{Evidence for Pautz}
\protect\hypertarget{evidence-for-pautz}{}
It's clear from Pautz's arguments that he thinks the evidence is
something like the phenomenal apperances.

\begin{itemize}[<+->]
\tightlist
\item
  He often describes cases like the demon case as one where the evidence
  does not change between the `hijacked' case and the evil demon case.
\end{itemize}
\end{frame}

\begin{frame}{Evidence for Siegel}
\protect\hypertarget{evidence-for-siegel}{}
I think that to make sense of Siegel's position, you have to understand
evidence as kicking in at a much earlier stage.

\begin{itemize}[<+->]
\tightlist
\item
  Evidence is what we're meant to fundamentally base our reasons on.
\item
  And for Siegel, phenomenal appearances are the result of inference.
\end{itemize}
\end{frame}

\begin{frame}{Evidence for Siegel}
\protect\hypertarget{evidence-for-siegel-1}{}
So for Siegel, evidence must be something earlier than phenomenology.

\begin{itemize}[<+->]
\tightlist
\item
  If that's right, then I'm not sure a lot of Pautz's arguments go
  through.
\item
  On the other hand, that's a really striking position!
\item
  Evidence is really inaccessible, and that seems like the big argument.
\end{itemize}
\end{frame}

\begin{frame}{Next Week}
\protect\hypertarget{next-week}{}
We'll finish up looking at Siegel's book, with discussing some different
criticisms, plus her responses to critics.
\end{frame}



\end{document}
