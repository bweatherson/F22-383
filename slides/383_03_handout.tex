% Options for packages loaded elsewhere
\PassOptionsToPackage{unicode}{hyperref}
\PassOptionsToPackage{hyphens}{url}
%
\documentclass[
  17pt,
  letterpaper,
  ignorenonframetext,
  aspectratio=169,
  handout]{beamer}
\usepackage{pgfpages}
\setbeamertemplate{caption}[numbered]
\setbeamertemplate{caption label separator}{: }
\setbeamercolor{caption name}{fg=normal text.fg}
\beamertemplatenavigationsymbolsempty
% Prevent slide breaks in the middle of a paragraph
\widowpenalties 1 10000
\raggedbottom
\setbeamertemplate{part page}{
  \centering
  \begin{beamercolorbox}[sep=16pt,center]{part title}
    \usebeamerfont{part title}\insertpart\par
  \end{beamercolorbox}
}
\setbeamertemplate{section page}{
  \centering
  \begin{beamercolorbox}[sep=12pt,center]{part title}
    \usebeamerfont{section title}\insertsection\par
  \end{beamercolorbox}
}
\setbeamertemplate{subsection page}{
  \centering
  \begin{beamercolorbox}[sep=8pt,center]{part title}
    \usebeamerfont{subsection title}\insertsubsection\par
  \end{beamercolorbox}
}
\AtBeginPart{
  \frame{\partpage}
}
\AtBeginSection{
  \ifbibliography
  \else
    \frame{\sectionpage}
  \fi
}
\AtBeginSubsection{
  \frame{\subsectionpage}
}

\usepackage{amsmath,amssymb}
\usepackage{lmodern}
\usepackage{iftex}
\ifPDFTeX
  \usepackage[T1]{fontenc}
  \usepackage[utf8]{inputenc}
  \usepackage{textcomp} % provide euro and other symbols
\else % if luatex or xetex
  \usepackage{unicode-math}
  \defaultfontfeatures{Scale=MatchLowercase}
  \defaultfontfeatures[\rmfamily]{Ligatures=TeX,Scale=1}
  \setmainfont[BoldFont = SF Pro Text Semibold, Scale =
MatchLowercase]{SF Pro Text Light}
\fi
\usecolortheme{wolverine}
\usefonttheme{serif} % use mainfont rather than sansfont for slide text
\useinnertheme{default}
\useoutertheme{miniframes}
% Use upquote if available, for straight quotes in verbatim environments
\IfFileExists{upquote.sty}{\usepackage{upquote}}{}
\IfFileExists{microtype.sty}{% use microtype if available
  \usepackage[]{microtype}
  \UseMicrotypeSet[protrusion]{basicmath} % disable protrusion for tt fonts
}{}
\makeatletter
\@ifundefined{KOMAClassName}{% if non-KOMA class
  \IfFileExists{parskip.sty}{%
    \usepackage{parskip}
  }{% else
    \setlength{\parindent}{0pt}
    \setlength{\parskip}{6pt plus 2pt minus 1pt}}
}{% if KOMA class
  \KOMAoptions{parskip=half}}
\makeatother
\usepackage{xcolor}
\newif\ifbibliography
\setlength{\emergencystretch}{3em} % prevent overfull lines
\setcounter{secnumdepth}{-\maxdimen} % remove section numbering


\providecommand{\tightlist}{%
  \setlength{\itemsep}{0pt}\setlength{\parskip}{0pt}}\usepackage{longtable,booktabs,array}
\usepackage{calc} % for calculating minipage widths
\usepackage{caption}
% Make caption package work with longtable
\makeatletter
\def\fnum@table{\tablename~\thetable}
\makeatother
\usepackage{graphicx}
\makeatletter
\def\maxwidth{\ifdim\Gin@nat@width>\linewidth\linewidth\else\Gin@nat@width\fi}
\def\maxheight{\ifdim\Gin@nat@height>\textheight\textheight\else\Gin@nat@height\fi}
\makeatother
% Scale images if necessary, so that they will not overflow the page
% margins by default, and it is still possible to overwrite the defaults
% using explicit options in \includegraphics[width, height, ...]{}
\setkeys{Gin}{width=\maxwidth,height=\maxheight,keepaspectratio}
% Set default figure placement to htbp
\makeatletter
\def\fps@figure{htbp}
\makeatother

\captionsetup[figure]{labelformat=empty}
\usepackage{pgfpages}
\setbeamertemplate{itemize item}[circle]
\setbeamertemplate{footline}[frame number]{}
\mode<handout>{\pgfpagesuselayout{6 on 1}[letterpaper, border shrink=8mm]}
\AtBeginSection{%
   \begin{frame}
       \tableofcontents[currentsection]
   \end{frame}
}
\makeatletter
\makeatother
\makeatletter
\makeatother
\makeatletter
\@ifpackageloaded{caption}{}{\usepackage{caption}}
\AtBeginDocument{%
\ifdefined\contentsname
  \renewcommand*\contentsname{Table of contents}
\else
  \newcommand\contentsname{Table of contents}
\fi
\ifdefined\listfigurename
  \renewcommand*\listfigurename{List of Figures}
\else
  \newcommand\listfigurename{List of Figures}
\fi
\ifdefined\listtablename
  \renewcommand*\listtablename{List of Tables}
\else
  \newcommand\listtablename{List of Tables}
\fi
\ifdefined\figurename
  \renewcommand*\figurename{Figure}
\else
  \newcommand\figurename{Figure}
\fi
\ifdefined\tablename
  \renewcommand*\tablename{Table}
\else
  \newcommand\tablename{Table}
\fi
}
\@ifpackageloaded{float}{}{\usepackage{float}}
\floatstyle{ruled}
\@ifundefined{c@chapter}{\newfloat{codelisting}{h}{lop}}{\newfloat{codelisting}{h}{lop}[chapter]}
\floatname{codelisting}{Listing}
\newcommand*\listoflistings{\listof{codelisting}{List of Listings}}
\makeatother
\makeatletter
\@ifpackageloaded{caption}{}{\usepackage{caption}}
\@ifpackageloaded{subcaption}{}{\usepackage{subcaption}}
\makeatother
\makeatletter
\@ifpackageloaded{tcolorbox}{}{\usepackage[many]{tcolorbox}}
\makeatother
\makeatletter
\@ifundefined{shadecolor}{\definecolor{shadecolor}{rgb}{.97, .97, .97}}
\makeatother
\makeatletter
\makeatother
\ifLuaTeX
  \usepackage{selnolig}  % disable illegal ligatures
\fi
\IfFileExists{bookmark.sty}{\usepackage{bookmark}}{\usepackage{hyperref}}
\IfFileExists{xurl.sty}{\usepackage{xurl}}{} % add URL line breaks if available
\urlstyle{same} % disable monospaced font for URLs
\hypersetup{
  pdftitle={Knowledge and Reality, Lecture 03},
  pdfauthor={Brian Weatherson},
  hidelinks,
  pdfcreator={LaTeX via pandoc}}

\title{Knowledge and Reality, Lecture 03}
\author{Brian Weatherson}
\date{2022-09-07}

\begin{document}
\frame{\titlepage}
\ifdefined\Shaded\renewenvironment{Shaded}{\begin{tcolorbox}[borderline west={3pt}{0pt}{shadecolor}, interior hidden, enhanced, breakable, sharp corners, frame hidden, boxrule=0pt]}{\end{tcolorbox}}\fi

\hypertarget{review}{%
\section{Review}\label{review}}

\begin{frame}{Why Start With Indian Philosophy?}
\protect\hypertarget{why-start-with-indian-philosophy}{}
\begin{enumerate}[<+->]
\tightlist
\item
  On this topic, it's the earliest instance of work that feels like
  contemporary epistemology.
\item
  Get to see how many of questions arise across multiple traditions.
\end{enumerate}
\end{frame}

\begin{frame}{Pramāṇa}
\protect\hypertarget{pramux101ux1e47a}{}
\begin{itemize}[<+->]
\tightlist
\item
  All knowledge comes from some knowledge-generating method, or pramāṇa.
\item
  There only a handful of pramāṇas, and it's a central question to
  identify and explain them.
\end{itemize}
\end{frame}

\begin{frame}{Pramāṇa and Proof}
\protect\hypertarget{pramux101ux1e47a-and-proof}{}
\begin{itemize}[<+->]
\tightlist
\item
  The word `pramāṇa' literally means proof. So if you know something,
  you have a proof of it.
\item
  This makes them sound infallibilist, and I think that's basically the
  right way to read them.
\item
  But it's a very distinctive kind of infallibilism.
\end{itemize}
\end{frame}

\begin{frame}{Scepticism}
\protect\hypertarget{scepticism}{}
\begin{itemize}[<+->]
\tightlist
\item
  Main response is pragmatic.
\item
  We know we know stuff, because we know we act sensibly, and sensible
  action requires knowledge.
\end{itemize}
\end{frame}

\begin{frame}{Academic Scepticism}
\protect\hypertarget{academic-scepticism}{}
\begin{itemize}[<+->]
\tightlist
\item
  But you could do the same thing, and have a false belief.
\item
  Response: No, you wouldn't do the same thing.
\item
  No false believer follows a pramāṇa.
\end{itemize}
\end{frame}

\begin{frame}{Pyrrhonian Scepticism}
\protect\hypertarget{pyrrhonian-scepticism}{}
\begin{itemize}[<+->]
\tightlist
\item
  This leads to a regress.
\item
  Response 1: Pramāṇa are self-certifying.
\item
  Response 2: Don't need to know that you're using a pramāṇa, just that
  you are using one.
\end{itemize}
\end{frame}

\hypertarget{perception}{%
\section{Perception}\label{perception}}

\begin{frame}{Perception}
\protect\hypertarget{perception-1}{}
I'm going to mostly break off the history here.

\begin{itemize}[<+->]
\tightlist
\item
  What I want is to note some questions about perception.
\item
  And note that each of these were live questions in Classical Indian
  philosophy, without getting into who was on what side.
\end{itemize}
\end{frame}

\begin{frame}{Question One: Content}
\protect\hypertarget{question-one-content}{}
Do we see \textbf{that} things are true?
\end{frame}

\begin{frame}{Two Answers}
\protect\hypertarget{two-answers}{}
\begin{enumerate}[<+->]
\tightlist
\item
  Yes, and perceptual knowledge is when one simply accepts these
  contents.
\item
  No, and that's why illusions are so prevalent; all `perceptual' belief
  involves cognition, which is always fallible.
\end{enumerate}
\end{frame}

\begin{frame}{A (C17-C20) Western Answer}
\protect\hypertarget{a-c17-c20-western-answer}{}
\begin{itemize}[<+->]
\tightlist
\item
  Yes, and that gives us an analysis of what illusion is.
\item
  It's when the content is false.
\item
  I don't think that's available to most of the schools, since it would
  be very close to a false pramāṇa.
\end{itemize}
\end{frame}

\begin{frame}{Question Two}
\protect\hypertarget{question-two}{}
Assume the yes answer from now on, though we'll still talk about the
no's.

\begin{itemize}[<+->]
\tightlist
\item
  Does the content include individuals?
\item
  Or is it just properties, which we might (cognitively) use to identify
  individuals?
\end{itemize}
\end{frame}

\begin{frame}{Question Three}
\protect\hypertarget{question-three}{}
Does the content include properties?

\begin{itemize}[<+->]
\tightlist
\item
  I think this probably has to be yes if you think there's content, but
  I'm including it here for completeness.
\end{itemize}
\end{frame}

\begin{frame}{Question Four}
\protect\hypertarget{question-four}{}
Which properties can be contents of perception?

\begin{itemize}[<+->]
\tightlist
\item
  Presumably I can't simply \textbf{see} that someone is \emph{honest},
  or \emph{the grand-nephew of a prince}.
\item
  Is perception really thin, just shapes and colors?
\item
  Or does it include things like \emph{being a policeman} or, for that
  matter, \emph{being male}.
\end{itemize}
\end{frame}

\begin{frame}{Cognition and Perception}
\protect\hypertarget{cognition-and-perception}{}
\begin{itemize}[<+->]
\tightlist
\item
  Of course we can believe that someone is honest, a policeman, and the
  grand-nephew of a prince.
\item
  But do we need to use cognition to form those beliefs, or can we just
  take perception at face-value.
\end{itemize}
\end{frame}

\begin{frame}{Question Five}
\protect\hypertarget{question-five}{}
How do we acquire concepts for these properties?

\begin{enumerate}[<+->]
\tightlist
\item
  Innate
\item
  Cognition
\item
  Perception
\end{enumerate}
\end{frame}

\hypertarget{dattas-paper}{%
\section{Datta's Paper}\label{dattas-paper}}

\begin{frame}{Dhirendra Mohan Datta}
\protect\hypertarget{dhirendra-mohan-datta}{}
\begin{itemize}[<+->]
\tightlist
\item
  His name is, I'm fairly sure, misspelled in the \emph{Mind} article
  you're reading.
\item
  Lots of reasons to think the author of this piece is the prominent 20C
  Indian philosopher Dhirendra Mohan Datta, not ``Dhirendron Mohan
  Datta''.
\end{itemize}
\end{frame}

\begin{frame}{Dhirendra Mohan Datta}
\protect\hypertarget{dhirendra-mohan-datta-1}{}
\begin{itemize}[<+->]
\tightlist
\item
  Has lots of books you can find through
  \textless archive.org\textgreater.
\item
  These include a textbook on Indian philosophy, and a book on
  epistemology that came out in two editions.
\item
  The first was in 1933, the second in 1960.
\end{itemize}
\end{frame}

\begin{frame}{Dhirendra Mohan Datta}
\protect\hypertarget{dhirendra-mohan-datta-2}{}
That's a big gap. What happened?

\begin{itemize}[<+->]
\tightlist
\item
  Indian independence.
\item
  Datta took a long break from academic philosophy to work closely with
  Ghandi from a fairly early stage in the movement.
\item
  And after it was done, he wrote a book on Ghandi's philosophy.
\end{itemize}
\end{frame}

\begin{frame}{Festschrift}
\protect\hypertarget{festschrift}{}
I'm getting a bunch of this info from a festschrift for him that was
published in 1960 as \emph{World Perspectives In Philosophy Religion And
Culture}.

\begin{itemize}[<+->]
\tightlist
\item
  The volume includes a lot of prominent figures in English language
  philosophy, including William Frankena, one of the most prominent
  members of UM's philosophy department in the 20th century.
\end{itemize}
\end{frame}

\begin{frame}{Dhirendra Mohan Datta}
\protect\hypertarget{dhirendra-mohan-datta-3}{}
\begin{itemize}[<+->]
\tightlist
\item
  Datta was well known among people who worked on comparative philosophy
  because he was so interested in connecting Western and Indian
  philosophy.
\item
  But as far as I can tell, the results of going into comparative
  philosophy was that he ended up more interested in Indian/Chinese work
  than Indian/Western work.
\end{itemize}
\end{frame}

\begin{frame}{Testimony as a Method of Knowledge}
\protect\hypertarget{testimony-as-a-method-of-knowledge}{}
Obviously the title is not literally ``Testimony is a pramāṇa''.

\begin{itemize}[<+->]
\tightlist
\item
  But it kind of means that.
\end{itemize}
\end{frame}

\begin{frame}{Testimony as a Method of Knowledge}
\protect\hypertarget{testimony-as-a-method-of-knowledge-1}{}
One of my pandemic projects was using big data tools to build a model of
what happened over time in leading philosophy journals.

\begin{itemize}[<+->]
\tightlist
\item
  You can see the results at \url{http://lda.weatherson.org}.
\item
  And I was particularly interested in the history (since the 1870s) of
  work in theory of knowledge.
\end{itemize}
\end{frame}

\begin{frame}{Testimony as a Method of Knowledge}
\protect\hypertarget{testimony-as-a-method-of-knowledge-2}{}
\begin{itemize}[<+->]
\tightlist
\item
  The model said there was precisely one (1) pre-WWII article (out of
  6000) that it had real confidence (probability greater than 0.4) that
  it should be put with modern work on knowledge.
\item
  It was Datta's.
\end{itemize}
\end{frame}

\hypertarget{testimony}{%
\section{Testimony}\label{testimony}}

\begin{frame}{Is Testimony a Pramāṇa?}
\protect\hypertarget{is-testimony-a-pramux101ux1e47a}{}
Structure of the paper.

\begin{itemize}[<+->]
\tightlist
\item
  Float arguments for no.
\item
  Offer replies.
\end{itemize}
\end{frame}

\begin{frame}{Two Thoughts}
\protect\hypertarget{two-thoughts}{}
\begin{itemize}[<+->]
\tightlist
\item
  Not sure I see much of a positive argument here, but that's probably
  ok.
\item
  Lots of appeal to \textbf{overgeneration} arguments.
\end{itemize}
\end{frame}

\begin{frame}{First Objection}
\protect\hypertarget{first-objection}{}
\begin{itemize}[<+->]
\tightlist
\item
  Need to double check what people say.
\end{itemize}
\end{frame}

\begin{frame}{First Reply (to First Objection)}
\protect\hypertarget{first-reply-to-first-objection}{}
\begin{itemize}[<+->]
\tightlist
\item
  We don't in fact double check.
\item
  But this isn't much good as a reply, since arguably we should.
\end{itemize}
\end{frame}

\begin{frame}{Second Reply (to First Objection)}
\protect\hypertarget{second-reply-to-first-objection}{}
\begin{itemize}[<+->]
\tightlist
\item
  Anything might need double checking.
\item
  If this worked, perception, inference, etc would not be methods.
\end{itemize}
\end{frame}

\begin{frame}{Third Reply}
\protect\hypertarget{third-reply}{}
I'm not sure I quite understand the move on page 2 (i.e., 355).

\begin{itemize}[<+->]
\tightlist
\item
  Datta makes a distinction between knowledge of a fact and knowledge of
  validity.
\item
  What exactly is that distinction?
\end{itemize}
\end{frame}

\begin{frame}{Third Reply}
\protect\hypertarget{third-reply-1}{}
\begin{itemize}[<+->]
\tightlist
\item
  At times it seems like the difference between knowledge, and knowing
  that one has knowledge.
\item
  At other times it seems like the `difference' between knowing
  something, and knowing that thing is true.
\item
  And that seems bad to rely on; knowledge is knowledge of truth.
\end{itemize}
\end{frame}

\begin{frame}{Second Objection}
\protect\hypertarget{second-objection}{}
\begin{itemize}[<+->]
\tightlist
\item
  Testimony isn't an independent source because it relies on some other
  method for the speaker to get knowledge.
\end{itemize}
\end{frame}

\begin{frame}{Reply to Second Objection}
\protect\hypertarget{reply-to-second-objection}{}
\begin{itemize}[<+->]
\tightlist
\item
  It might still be independent for the hearer.
\end{itemize}
\end{frame}

\begin{frame}{Third Objection}
\protect\hypertarget{third-objection}{}
\begin{itemize}[<+->]
\tightlist
\item
  Testimony requires perception, since you have to use perception to
  know what words are spoken.
\end{itemize}
\end{frame}

\begin{frame}{Reply to Third Objection}
\protect\hypertarget{reply-to-third-objection}{}
\begin{itemize}[<+->]
\tightlist
\item
  All knowledge is holistic.
\item
  When we say something is a method, we mean it can be the last step.
\item
  Relatedly, no one denies inference is a method though by definition it
  has other inputs.
\end{itemize}
\end{frame}

\begin{frame}{Evaluation}
\protect\hypertarget{evaluation}{}
\begin{itemize}[<+->]
\tightlist
\item
  This is a good thing to worry about, and sometimes gets ignored in the
  recent discussion.
\item
  But it does make me worry that the whole talk about methods is on
  shakier footing than it appears.
\end{itemize}
\end{frame}

\begin{frame}{Fourth Objection}
\protect\hypertarget{fourth-objection}{}
\begin{itemize}[<+->]
\tightlist
\item
  Testimony can't be ultimate, because sources sometimes conflict.
\item
  That is, different people will tell you different, and sometimes
  inconsistent, things.
\end{itemize}
\end{frame}

\begin{frame}{First Reply to Fourth Objection}
\protect\hypertarget{first-reply-to-fourth-objection}{}
\begin{itemize}[<+->]
\tightlist
\item
  Any source may involve conflict.
\item
  Sometimes the same thing looks different from two angles.
\item
  So this also overgenerates.
\end{itemize}
\end{frame}

\begin{frame}{Second Reply to Fourth Objection}
\protect\hypertarget{second-reply-to-fourth-objection}{}
\begin{itemize}[<+->]
\tightlist
\item
  Some knowledge from testimony is beyond dispute.
\item
  E.g., that a command was given.
\item
  I'm not sure this should count as genuinely testimonial though; feels
  more perceptual.
\end{itemize}
\end{frame}

\begin{frame}{For Next Time}
\protect\hypertarget{for-next-time}{}
\begin{itemize}[<+->]
\tightlist
\item
  We'll look at a (very) modern form of what Datta calls ``the ordinary
  answer''.
\end{itemize}
\end{frame}



\end{document}
