% Options for packages loaded elsewhere
\PassOptionsToPackage{unicode}{hyperref}
\PassOptionsToPackage{hyphens}{url}
%
\documentclass[
  17pt,
  letterpaper,
  ignorenonframetext,
  aspectratio=169,
  handout]{beamer}
\usepackage{pgfpages}
\setbeamertemplate{caption}[numbered]
\setbeamertemplate{caption label separator}{: }
\setbeamercolor{caption name}{fg=normal text.fg}
\beamertemplatenavigationsymbolsempty
% Prevent slide breaks in the middle of a paragraph
\widowpenalties 1 10000
\raggedbottom
\setbeamertemplate{part page}{
  \centering
  \begin{beamercolorbox}[sep=16pt,center]{part title}
    \usebeamerfont{part title}\insertpart\par
  \end{beamercolorbox}
}
\setbeamertemplate{section page}{
  \centering
  \begin{beamercolorbox}[sep=12pt,center]{part title}
    \usebeamerfont{section title}\insertsection\par
  \end{beamercolorbox}
}
\setbeamertemplate{subsection page}{
  \centering
  \begin{beamercolorbox}[sep=8pt,center]{part title}
    \usebeamerfont{subsection title}\insertsubsection\par
  \end{beamercolorbox}
}
\AtBeginPart{
  \frame{\partpage}
}
\AtBeginSection{
  \ifbibliography
  \else
    \frame{\sectionpage}
  \fi
}
\AtBeginSubsection{
  \frame{\subsectionpage}
}

\usepackage{amsmath,amssymb}
\usepackage{lmodern}
\usepackage{iftex}
\ifPDFTeX
  \usepackage[T1]{fontenc}
  \usepackage[utf8]{inputenc}
  \usepackage{textcomp} % provide euro and other symbols
\else % if luatex or xetex
  \usepackage{unicode-math}
  \defaultfontfeatures{Scale=MatchLowercase}
  \defaultfontfeatures[\rmfamily]{Ligatures=TeX,Scale=1}
  \setmainfont[BoldFont = SF Pro Text Semibold, Scale =
MatchLowercase]{SF Pro Text Light}
\fi
\usecolortheme{wolverine}
\usefonttheme{serif} % use mainfont rather than sansfont for slide text
\useinnertheme{default}
\useoutertheme{miniframes}
% Use upquote if available, for straight quotes in verbatim environments
\IfFileExists{upquote.sty}{\usepackage{upquote}}{}
\IfFileExists{microtype.sty}{% use microtype if available
  \usepackage[]{microtype}
  \UseMicrotypeSet[protrusion]{basicmath} % disable protrusion for tt fonts
}{}
\makeatletter
\@ifundefined{KOMAClassName}{% if non-KOMA class
  \IfFileExists{parskip.sty}{%
    \usepackage{parskip}
  }{% else
    \setlength{\parindent}{0pt}
    \setlength{\parskip}{6pt plus 2pt minus 1pt}}
}{% if KOMA class
  \KOMAoptions{parskip=half}}
\makeatother
\usepackage{xcolor}
\newif\ifbibliography
\setlength{\emergencystretch}{3em} % prevent overfull lines
\setcounter{secnumdepth}{-\maxdimen} % remove section numbering


\providecommand{\tightlist}{%
  \setlength{\itemsep}{0pt}\setlength{\parskip}{0pt}}\usepackage{longtable,booktabs,array}
\usepackage{calc} % for calculating minipage widths
\usepackage{caption}
% Make caption package work with longtable
\makeatletter
\def\fnum@table{\tablename~\thetable}
\makeatother
\usepackage{graphicx}
\makeatletter
\def\maxwidth{\ifdim\Gin@nat@width>\linewidth\linewidth\else\Gin@nat@width\fi}
\def\maxheight{\ifdim\Gin@nat@height>\textheight\textheight\else\Gin@nat@height\fi}
\makeatother
% Scale images if necessary, so that they will not overflow the page
% margins by default, and it is still possible to overwrite the defaults
% using explicit options in \includegraphics[width, height, ...]{}
\setkeys{Gin}{width=\maxwidth,height=\maxheight,keepaspectratio}
% Set default figure placement to htbp
\makeatletter
\def\fps@figure{htbp}
\makeatother

\captionsetup[figure]{labelformat=empty}
\usepackage{pgfpages}
\setbeamertemplate{itemize item}[circle]
\setbeamertemplate{footline}[frame number]{}
\mode<handout>{\pgfpagesuselayout{6 on 1}[letterpaper, border shrink=8mm]}
\AtBeginSection{%
   \begin{frame}
       \tableofcontents[currentsection]
   \end{frame}
}
\makeatletter
\makeatother
\makeatletter
\makeatother
\makeatletter
\@ifpackageloaded{caption}{}{\usepackage{caption}}
\AtBeginDocument{%
\ifdefined\contentsname
  \renewcommand*\contentsname{Table of contents}
\else
  \newcommand\contentsname{Table of contents}
\fi
\ifdefined\listfigurename
  \renewcommand*\listfigurename{List of Figures}
\else
  \newcommand\listfigurename{List of Figures}
\fi
\ifdefined\listtablename
  \renewcommand*\listtablename{List of Tables}
\else
  \newcommand\listtablename{List of Tables}
\fi
\ifdefined\figurename
  \renewcommand*\figurename{Figure}
\else
  \newcommand\figurename{Figure}
\fi
\ifdefined\tablename
  \renewcommand*\tablename{Table}
\else
  \newcommand\tablename{Table}
\fi
}
\@ifpackageloaded{float}{}{\usepackage{float}}
\floatstyle{ruled}
\@ifundefined{c@chapter}{\newfloat{codelisting}{h}{lop}}{\newfloat{codelisting}{h}{lop}[chapter]}
\floatname{codelisting}{Listing}
\newcommand*\listoflistings{\listof{codelisting}{List of Listings}}
\makeatother
\makeatletter
\@ifpackageloaded{caption}{}{\usepackage{caption}}
\@ifpackageloaded{subcaption}{}{\usepackage{subcaption}}
\makeatother
\makeatletter
\@ifpackageloaded{tcolorbox}{}{\usepackage[many]{tcolorbox}}
\makeatother
\makeatletter
\@ifundefined{shadecolor}{\definecolor{shadecolor}{rgb}{.97, .97, .97}}
\makeatother
\makeatletter
\makeatother
\ifLuaTeX
  \usepackage{selnolig}  % disable illegal ligatures
\fi
\IfFileExists{bookmark.sty}{\usepackage{bookmark}}{\usepackage{hyperref}}
\IfFileExists{xurl.sty}{\usepackage{xurl}}{} % add URL line breaks if available
\urlstyle{same} % disable monospaced font for URLs
\hypersetup{
  pdftitle={Knowledge and Reality, Lecture 10},
  pdfauthor={Brian Weatherson},
  hidelinks,
  pdfcreator={LaTeX via pandoc}}

\title{Knowledge and Reality, Lecture 10}
\author{Brian Weatherson}
\date{2022-10-03}

\begin{document}
\frame{\titlepage}
\ifdefined\Shaded\renewenvironment{Shaded}{\begin{tcolorbox}[interior hidden, enhanced, borderline west={3pt}{0pt}{shadecolor}, boxrule=0pt, breakable, frame hidden, sharp corners]}{\end{tcolorbox}}\fi

\hypertarget{two-sceptical-arguments}{%
\section{Two Sceptical Arguments}\label{two-sceptical-arguments}}

\begin{frame}{Two Sceptical Arguments}
\protect\hypertarget{two-sceptical-arguments-1}{}
\begin{enumerate}[<+->]
\tightlist
\item
  Pyrrhonian scepticism
\item
  Academic scepticism
\end{enumerate}
\end{frame}

\begin{frame}{Pyrrhonian scepticism}
\protect\hypertarget{pyrrhonian-scepticism}{}
\begin{enumerate}[<+->]
\tightlist
\item
  All knowledge is by some method or other.
\item
  A method only produces knowledge if the user knows it is reliable.
\item
  So there can be no first knowledge.
\item
  So there is no knowledge.
\end{enumerate}
\end{frame}

\begin{frame}{Academic Scepticism}
\protect\hypertarget{academic-scepticism}{}
\begin{enumerate}[<+->]
\tightlist
\item
  To have ordinary knowledge, we must know that disaster scenarios can't
  obtain.
\item
  We can't know that disaster scenarios don't obtain.
\item
  So we don't have ordinary knowledge.
\end{enumerate}
\end{frame}

\begin{frame}{Three Kinds of Response}
\protect\hypertarget{three-kinds-of-response}{}
\begin{enumerate}[<+->]
\tightlist
\item
  Dismissive
\item
  Objecting to premises/steps
\item
  Supplying positive argument for knowledge
\end{enumerate}
\end{frame}

\hypertarget{dismissive}{%
\section{Dismissive}\label{dismissive}}

\begin{frame}{Dismissive}
\protect\hypertarget{dismissive-1}{}
A popular mid-20C response, often associated with Wittgenstein.

\begin{itemize}[<+->]
\tightlist
\item
  We should ignore scepticism because it is obviously ridiculous.
\end{itemize}
\end{frame}

\begin{frame}{Dismissive}
\protect\hypertarget{dismissive-2}{}
An 18C variant, associated with David Hume (though really not Hume's own
view.)

\begin{itemize}[<+->]
\tightlist
\item
  We should ignore scepticism because we are incapable of taking it
  seriously.
\end{itemize}
\end{frame}

\begin{frame}{A Reply}
\protect\hypertarget{a-reply}{}
There are things we should be sceptical about, such as claims made on
behalf of new technologies.

\begin{itemize}[<+->]
\tightlist
\item
  It is easy for that kind of scepticism to go too far.
\item
  One reason (my reason) to carefully formulate arguments for global
  scepticism is to see which premises are out of bounds in everyday
  sceptical reasoning.
\end{itemize}
\end{frame}

\begin{frame}{A Reply}
\protect\hypertarget{a-reply-1}{}
If a reason to be sceptical of a person/method/claim/theory would
generalise to be a reason to be sceptical of the external world, then
it's a bad reason.

\begin{itemize}[<+->]
\tightlist
\item
  This turns out to be a surprisingly powerful principle.
\end{itemize}
\end{frame}

\begin{frame}{Two Targets}
\protect\hypertarget{two-targets}{}
\begin{enumerate}[<+->]
\tightlist
\item
  People who argue for scepticism in a particular domain.
\item
  People who argue for positive theories on the grounds that all other
  theories would lead to scepticism in that domain.
\end{enumerate}

\begin{itemize}[<+->]
\tightlist
\item
  In both cases, asking whether their arguments would imply global
  scepticism is a worthwhile question.
\end{itemize}
\end{frame}

\hypertarget{pyrrhonian-scepticism-1}{%
\section{Pyrrhonian scepticism}\label{pyrrhonian-scepticism-1}}

\begin{frame}{Pyrrhonian scepticism - Responses}
\protect\hypertarget{pyrrhonian-scepticism---responses}{}
\begin{enumerate}[<+->]
\tightlist
\item
  Self-defeating
\item
  Deny premise 2
\item
  Deny inference from 1 and 2 to 3.
\item
  Deny inference from 3 to 4.
\end{enumerate}
\end{frame}

\begin{frame}{Pyrrhonian scepticism - Self-defeating}
\protect\hypertarget{pyrrhonian-scepticism---self-defeating}{}
\begin{itemize}[<+->]
\tightlist
\item
  The premises of the argument can't be known, since nothing can be
  known.
\item
  So we have no reason to believe what the Pyrrhonian says.
\end{itemize}
\end{frame}

\begin{frame}{Pyrrhonian scepticism - P2}
\protect\hypertarget{pyrrhonian-scepticism---p2}{}
\begin{itemize}[<+->]
\tightlist
\item
  Reliabilists say that it is enough that the method is actually
  reliable; it doesn't have to be known to be reliable.
\end{itemize}
\end{frame}

\begin{frame}{Pyrrhonian scepticism - Step 3}
\protect\hypertarget{pyrrhonian-scepticism---step-3}{}
\begin{itemize}[<+->]
\tightlist
\item
  Some Indian traditions deny the move from 1, 2 to 3.
\item
  They say that you can come to know that the method you are using is
  reliable at the same time you use it.
\item
  What they deny is that the knowledge of reliability has to come before
  the use of it; it could be simultaneous.
\item
  In those cases, the output of the method and the knowledge it is
  reliable would be `equal first' knowledge.
\end{itemize}
\end{frame}

\begin{frame}{Infinitism}
\protect\hypertarget{infinitism}{}
\begin{itemize}[<+->]
\tightlist
\item
  Maybe there can be no first knowledge, but we have an infinite amount
  of knowledge.
\item
  It's not clear to me exactly how this is supposed to work, but for
  completeness I should note that infinitism is one of the options here.
\end{itemize}
\end{frame}

\hypertarget{academic-scepticism-1}{%
\section{Academic Scepticism}\label{academic-scepticism-1}}

\begin{frame}{Limits of Academic Scepticism}
\protect\hypertarget{limits-of-academic-scepticism}{}
It can only show us that we don't know things that are false in
plausible `disaster scenarios'.

\begin{itemize}[<+->]
\tightlist
\item
  So it can't be used to defeat knowledge that 2+2=4, or that we have
  minds.
\end{itemize}
\end{frame}

\begin{frame}{Disaster Scenarios}
\protect\hypertarget{disaster-scenarios}{}
A disaster scenario is one in which

\begin{enumerate}[<+->]
\tightlist
\item
  We have the same evidence we actually do; but
\item
  Some things we ordinarily take to be true are not true.
\end{enumerate}

\begin{itemize}[<+->]
\tightlist
\item
  So it turns a lot on what counts as evidence.
\end{itemize}
\end{frame}

\begin{frame}{Evidence}
\protect\hypertarget{evidence}{}
Most sceptics assume that evidence is something like phenomenology.

\begin{itemize}[<+->]
\tightlist
\item
  This is very much up for debate.
\item
  The Indian realists (especially Nyāya) rejected it, and we'll see lots
  of reasons to reject it going forward.
\end{itemize}
\end{frame}

\begin{frame}{Disaster Scenario}
\protect\hypertarget{disaster-scenario}{}
Sceptics normally don't care about whether the scenarios they use are
realistic.

\begin{itemize}[<+->]
\tightlist
\item
  As we'll hopefully touch on at the end, anti-sceptics sometimes do
  care about that.
\end{itemize}
\end{frame}

\begin{frame}{Why Believe Premise 2}
\protect\hypertarget{why-believe-premise-2}{}
\begin{enumerate}[<+->]
\tightlist
\item
  Raw intuition
\item
  Sensitivity
\item
  Defensibility
\item
  Method
\end{enumerate}
\end{frame}

\begin{frame}{Raw Intuition}
\protect\hypertarget{raw-intuition}{}
Problems:

\begin{enumerate}[<+->]
\tightlist
\item
  Not everyone has the intuition.
\item
  The intuitions that (a) we know we have hands, and (b) if we know we
  have hands we can deduce that we are not, e.g., HBIVs, are stronger.
\end{enumerate}
\end{frame}

\begin{frame}{Sensitivity}
\protect\hypertarget{sensitivity}{}
We talked about this already.

\begin{itemize}[<+->]
\tightlist
\item
  The belief that I'm not in a disaster scenario is not sensitive.
\item
  But sensitivity leads to weird results in things like the Potemkin
  village case.
\end{itemize}
\end{frame}

\begin{frame}{Defensibility}
\protect\hypertarget{defensibility}{}
The sceptic's idea here is that we only know something if we can defend
that belief to a critic.

\begin{itemize}[<+->]
\tightlist
\item
  But this is a very strong claim.
\item
  Try defending the view that there are some reasons (for beliefs,
  actions, whatever) to a reasons sceptic.
\item
  Taken seriously, this would turn academic scepticism into the less
  plausible Pyrrhonian scepticism.
\end{itemize}
\end{frame}

\begin{frame}{Method}
\protect\hypertarget{method}{}
Lots of (western) philosophers have thought that everything we know
comes from one of two methods.

\begin{enumerate}[<+->]
\tightlist
\item
  Observation
\item
  Pure Reason
\end{enumerate}

\begin{itemize}[<+->]
\tightlist
\item
  Question: By which of these do we know we're not in a disaster
  scenario?
\end{itemize}
\end{frame}

\begin{frame}{A Better Sceptical Argument}
\protect\hypertarget{a-better-sceptical-argument}{}
\begin{enumerate}[<+->]
\tightlist
\item
  If we know we're not in a disaster scenario, we know this by
  observation or pure reason.
\item
  We can't know this by observation, since observation doesn't
  distinguish normal from disaster scenarios.
\item
  We can't know this by pure reason, since the only things we can rule
  out by pure reason are impossibilities.
\item
  So we don't know we're not in a disaster scenario.
\end{enumerate}
\end{frame}

\begin{frame}{A Better Sceptical Argument}
\protect\hypertarget{a-better-sceptical-argument-1}{}
\begin{itemize}[<+->]
\tightlist
\item
  Every premise there is debatable.
\item
  But unlike other sceptical arguments, it isn't clear just which
  premise fails.
\end{itemize}
\end{frame}

\hypertarget{positive-arguments-for-anti-scepticism}{%
\section{Positive Arguments for
Anti-Scepticism}\label{positive-arguments-for-anti-scepticism}}

\begin{frame}{Two Positive Arguments}
\protect\hypertarget{two-positive-arguments}{}
\begin{enumerate}[<+->]
\tightlist
\item
  Inference to the Best Explanation
\item
  Reliable Observation
\end{enumerate}
\end{frame}

\begin{frame}{IBE}
\protect\hypertarget{ibe}{}
\begin{itemize}[<+->]
\tightlist
\item
  We have a bunch of evidence, let's say along with the sceptic that
  it's phenomenal.
\item
  What's the best explanation for this evidence?
\item
  Arguably, the existence of an external world.
\end{itemize}
\end{frame}

\begin{frame}{Problems}
\protect\hypertarget{problems}{}
\begin{itemize}[<+->]
\tightlist
\item
  Well, there are simpler explanations, like God created you and you
  alone.
\item
  And maybe there are more complicated but more plausible ones, like
  that you're a video game character.
\end{itemize}
\end{frame}

\begin{frame}{Reliable Observation}
\protect\hypertarget{reliable-observation}{}
\begin{itemize}[<+->]
\tightlist
\item
  As a matter of fact I know things, because as a matter of fact my
  senses are reliable.
\item
  Of course, how I know this is a hard question.
\end{itemize}
\end{frame}

\begin{frame}{Reliable Observations}
\protect\hypertarget{reliable-observations}{}
Two moves at this point:

\begin{itemize}[<+->]
\tightlist
\item
  Deny that it matters whether I know that I know; knowledge is enough.
\item
  Say that I'm also pretty good at telling reliable from unreliable
  observers apart, and using that skill I can tell that I'm one of the
  reliable ones.
\end{itemize}
\end{frame}

\begin{frame}{Problem}
\protect\hypertarget{problem}{}
I'm not actually that reliable; I dream a lot.
\end{frame}

\begin{frame}{Responses}
\protect\hypertarget{responses}{}
\begin{itemize}[<+->]
\tightlist
\item
  Dreams aren't really beliefs, so my belief forming capacities are
  reliable.
\item
  I'm reliable when I'm awake, and that's enough. Compare an athlete who
  can do one very specific thing well; they might be reliably successful
  even if they would fail were circumstances different.
\end{itemize}
\end{frame}

\begin{frame}{For Next Time}
\protect\hypertarget{for-next-time}{}
\begin{itemize}[<+->]
\tightlist
\item
  We'll move onto Pasnau's book.
\end{itemize}
\end{frame}



\end{document}
