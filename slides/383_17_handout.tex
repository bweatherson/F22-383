% Options for packages loaded elsewhere
\PassOptionsToPackage{unicode}{hyperref}
\PassOptionsToPackage{hyphens}{url}
%
\documentclass[
  17pt,
  letterpaper,
  ignorenonframetext,
  aspectratio=169,
  handout]{beamer}
\usepackage{pgfpages}
\setbeamertemplate{caption}[numbered]
\setbeamertemplate{caption label separator}{: }
\setbeamercolor{caption name}{fg=normal text.fg}
\beamertemplatenavigationsymbolsempty
% Prevent slide breaks in the middle of a paragraph
\widowpenalties 1 10000
\raggedbottom
\setbeamertemplate{part page}{
  \centering
  \begin{beamercolorbox}[sep=16pt,center]{part title}
    \usebeamerfont{part title}\insertpart\par
  \end{beamercolorbox}
}
\setbeamertemplate{section page}{
  \centering
  \begin{beamercolorbox}[sep=12pt,center]{part title}
    \usebeamerfont{section title}\insertsection\par
  \end{beamercolorbox}
}
\setbeamertemplate{subsection page}{
  \centering
  \begin{beamercolorbox}[sep=8pt,center]{part title}
    \usebeamerfont{subsection title}\insertsubsection\par
  \end{beamercolorbox}
}
\AtBeginPart{
  \frame{\partpage}
}
\AtBeginSection{
  \ifbibliography
  \else
    \frame{\sectionpage}
  \fi
}
\AtBeginSubsection{
  \frame{\subsectionpage}
}

\usepackage{amsmath,amssymb}
\usepackage{lmodern}
\usepackage{iftex}
\ifPDFTeX
  \usepackage[T1]{fontenc}
  \usepackage[utf8]{inputenc}
  \usepackage{textcomp} % provide euro and other symbols
\else % if luatex or xetex
  \usepackage{unicode-math}
  \defaultfontfeatures{Scale=MatchLowercase}
  \defaultfontfeatures[\rmfamily]{Ligatures=TeX,Scale=1}
  \setmainfont[BoldFont = SF Pro Text Semibold, Scale =
MatchLowercase]{SF Pro Text Light}
\fi
\usecolortheme{wolverine}
\usefonttheme{serif} % use mainfont rather than sansfont for slide text
\useinnertheme{default}
\useoutertheme{miniframes}
% Use upquote if available, for straight quotes in verbatim environments
\IfFileExists{upquote.sty}{\usepackage{upquote}}{}
\IfFileExists{microtype.sty}{% use microtype if available
  \usepackage[]{microtype}
  \UseMicrotypeSet[protrusion]{basicmath} % disable protrusion for tt fonts
}{}
\makeatletter
\@ifundefined{KOMAClassName}{% if non-KOMA class
  \IfFileExists{parskip.sty}{%
    \usepackage{parskip}
  }{% else
    \setlength{\parindent}{0pt}
    \setlength{\parskip}{6pt plus 2pt minus 1pt}}
}{% if KOMA class
  \KOMAoptions{parskip=half}}
\makeatother
\usepackage{xcolor}
\newif\ifbibliography
\setlength{\emergencystretch}{3em} % prevent overfull lines
\setcounter{secnumdepth}{-\maxdimen} % remove section numbering


\providecommand{\tightlist}{%
  \setlength{\itemsep}{0pt}\setlength{\parskip}{0pt}}\usepackage{longtable,booktabs,array}
\usepackage{calc} % for calculating minipage widths
\usepackage{caption}
% Make caption package work with longtable
\makeatletter
\def\fnum@table{\tablename~\thetable}
\makeatother
\usepackage{graphicx}
\makeatletter
\def\maxwidth{\ifdim\Gin@nat@width>\linewidth\linewidth\else\Gin@nat@width\fi}
\def\maxheight{\ifdim\Gin@nat@height>\textheight\textheight\else\Gin@nat@height\fi}
\makeatother
% Scale images if necessary, so that they will not overflow the page
% margins by default, and it is still possible to overwrite the defaults
% using explicit options in \includegraphics[width, height, ...]{}
\setkeys{Gin}{width=\maxwidth,height=\maxheight,keepaspectratio}
% Set default figure placement to htbp
\makeatletter
\def\fps@figure{htbp}
\makeatother

\captionsetup[figure]{labelformat=empty}
\usepackage{pgfpages}
\setbeamertemplate{itemize item}[circle]
\setbeamertemplate{footline}[frame number]{}
\mode<handout>{\pgfpagesuselayout{6 on 1}[letterpaper, border shrink=8mm]}
\AtBeginSection{%
   \begin{frame}
       \tableofcontents[currentsection]
   \end{frame}
}
\makeatletter
\makeatother
\makeatletter
\makeatother
\makeatletter
\@ifpackageloaded{caption}{}{\usepackage{caption}}
\AtBeginDocument{%
\ifdefined\contentsname
  \renewcommand*\contentsname{Table of contents}
\else
  \newcommand\contentsname{Table of contents}
\fi
\ifdefined\listfigurename
  \renewcommand*\listfigurename{List of Figures}
\else
  \newcommand\listfigurename{List of Figures}
\fi
\ifdefined\listtablename
  \renewcommand*\listtablename{List of Tables}
\else
  \newcommand\listtablename{List of Tables}
\fi
\ifdefined\figurename
  \renewcommand*\figurename{Figure}
\else
  \newcommand\figurename{Figure}
\fi
\ifdefined\tablename
  \renewcommand*\tablename{Table}
\else
  \newcommand\tablename{Table}
\fi
}
\@ifpackageloaded{float}{}{\usepackage{float}}
\floatstyle{ruled}
\@ifundefined{c@chapter}{\newfloat{codelisting}{h}{lop}}{\newfloat{codelisting}{h}{lop}[chapter]}
\floatname{codelisting}{Listing}
\newcommand*\listoflistings{\listof{codelisting}{List of Listings}}
\makeatother
\makeatletter
\@ifpackageloaded{caption}{}{\usepackage{caption}}
\@ifpackageloaded{subcaption}{}{\usepackage{subcaption}}
\makeatother
\makeatletter
\@ifpackageloaded{tcolorbox}{}{\usepackage[many]{tcolorbox}}
\makeatother
\makeatletter
\@ifundefined{shadecolor}{\definecolor{shadecolor}{rgb}{.97, .97, .97}}
\makeatother
\makeatletter
\makeatother
\ifLuaTeX
  \usepackage{selnolig}  % disable illegal ligatures
\fi
\IfFileExists{bookmark.sty}{\usepackage{bookmark}}{\usepackage{hyperref}}
\IfFileExists{xurl.sty}{\usepackage{xurl}}{} % add URL line breaks if available
\urlstyle{same} % disable monospaced font for URLs
\hypersetup{
  pdftitle={Knowledge and Reality, Lecture 17},
  pdfauthor={Brian Weatherson},
  hidelinks,
  pdfcreator={LaTeX via pandoc}}

\title{Knowledge and Reality, Lecture 17}
\author{Brian Weatherson}
\date{2022-11-01}

\begin{document}
\frame{\titlepage}
\ifdefined\Shaded\renewenvironment{Shaded}{\begin{tcolorbox}[boxrule=0pt, interior hidden, breakable, borderline west={3pt}{0pt}{shadecolor}, sharp corners, frame hidden, enhanced]}{\end{tcolorbox}}\fi

\hypertarget{retreating-from-the-ideal}{%
\section{Retreating from the Ideal}\label{retreating-from-the-ideal}}

\begin{frame}{Where Do We Go Now but Nowhere?}
\protect\hypertarget{where-do-we-go-now-but-nowhere}{}
I liked this book a lot, but didn't love the last chapter as much.

\begin{itemize}[<+->]
\tightlist
\item
  So I wanted to spend today talking about three things that I (mostly)
  disagreed with.
\item
  Hopefully giving you a couple of views will let you triangulate
  between them!
\end{itemize}
\end{frame}

\begin{frame}{Two Retreats}
\protect\hypertarget{two-retreats}{}
\begin{enumerate}[<+->]
\tightlist
\item
  To probabilism
\item
  To quietism
\end{enumerate}
\end{frame}

\begin{frame}{Probabilism}
\protect\hypertarget{probabilism}{}
\begin{itemize}[<+->]
\tightlist
\item
  Nothing is certain.
\item
  But everything has a probability.
\item
  What we can do is try to make sure the things we accept are as
  probable as possible.
\end{itemize}
\end{frame}

\begin{frame}{Two Problems (Among Many!)}
\protect\hypertarget{two-problems-among-many}{}
\begin{enumerate}[<+->]
\tightlist
\item
  What is the probability that God is a deceiver, or you're a brain in a
  vat?
\item
  How do we update these probabilities?
\end{enumerate}
\end{frame}

\begin{frame}{Probabilism}
\protect\hypertarget{probabilism-1}{}
\begin{itemize}[<+->]
\tightlist
\item
  This kind of view is very popular, and I've often used it.
\item
  But I mostly want to stress here that it isn't the only way to retreat
  from the ideal.
\end{itemize}
\end{frame}

\begin{frame}{Quietism}
\protect\hypertarget{quietism}{}
Another view, perhaps best represented by 20th century Austrian
philosopher Ludwig Wittgenstein, is that we should simply ignore
sceptical scenarios.

\begin{itemize}[<+->]
\tightlist
\item
  To know something isn't to know that it's probable.
\item
  Rather it's to know that it must be the case, given our evidence,
  unless something really weird happened.
\end{itemize}
\end{frame}

\begin{frame}{Advantages}
\protect\hypertarget{advantages}{}
\begin{enumerate}[<+->]
\tightlist
\item
  Doesn't have to say what probability sceptical scenarios get; they are
  simply ignored.
\item
  We can update when we come to know something new (where this knowledge
  means it's certain unless weird stuff happens).
\item
  Logic still works.
\end{enumerate}
\end{frame}

\begin{frame}{An Annoying Fact about Probability}
\protect\hypertarget{an-annoying-fact-about-probability}{}
You can have both of the following things true.

\begin{itemize}[<+->]
\tightlist
\item
  Each of \(p_1, p_2, \dots p_n\) is highly probable.
\item
  The conjunction \(p_1 \wedge p_2 \wedge \dots \wedge p_n\) is not very
  probable.
\end{itemize}
\end{frame}

\begin{frame}{No Weirdness}
\protect\hypertarget{no-weirdness}{}
But you can't have these things true.

\begin{itemize}[<+->]
\tightlist
\item
  Each of \(p_1, p_2, \dots p_n\) is true unless something weird
  happens.
\item
  The conjunction \(p_1 \wedge p_2 \wedge \dots \wedge p_n\) is true
  without anything weird happening.
\item
  So the ``Believe what's true unless something weird happens'' is more
  compatible with basic logic.
\end{itemize}
\end{frame}

\begin{frame}{A Problem}
\protect\hypertarget{a-problem}{}
What does it mean for something to be `weird' in the relevant sense?

\begin{itemize}[<+->]
\tightlist
\item
  There are some interesting ideas here, but I want to briefly mention
  one from British philosopher Crispin Wright.
\end{itemize}
\end{frame}

\begin{frame}{Scepticism and Inquiry}
\protect\hypertarget{scepticism-and-inquiry}{}
Here are some striking facts about things like brain-in-vat scenarios,
evil demons, etc.

\begin{itemize}[<+->]
\tightlist
\item
  We can't figure out if they obtain.
\item
  And no better inquiry will help us do this.
\item
  And we know this in advance; we know inquiry will be pointless.
\end{itemize}
\end{frame}

\begin{frame}{Don't Do Pointless Stuff}
\protect\hypertarget{dont-do-pointless-stuff}{}
This is Wright's key idea.

\begin{itemize}[<+->]
\tightlist
\item
  We are \textbf{entitled} to assume the truth of things that we know in
  advance it would be pointless to inquire into.
\item
  We want a very demanding standard of pointlessness here, so it covers
  brain-in-vat scenarios, but not too much.
\end{itemize}
\end{frame}

\begin{frame}{Further Reading}
\protect\hypertarget{further-reading}{}
I don't know if this idea works in full generality, but it's an
interesting option, and one that Pasnau slides over when he moves
quickly from the ideal case to the probabilist one.

\begin{itemize}[<+->]
\tightlist
\item
  Wright's view is set out in his 2004 Aristotelian Society paper
  ``Warrant for Nothing (And Foudnations for Free)''.
\end{itemize}
\end{frame}

\hypertarget{belief-credence-dualism}{%
\section{Belief-Credence Dualism}\label{belief-credence-dualism}}

\begin{frame}{Positive View}
\protect\hypertarget{positive-view}{}
If I'm reading him right, Pasnau's positive vision at the end is a
version of what's called \textbf{belief-credence dualism}. Three plans.

\begin{itemize}[<+->]
\tightlist
\item
  What is belief-credence dualism?
\item
  What is distinctive about Pasnau's version?
\item
  Is it a plausible view?
\end{itemize}
\end{frame}

\begin{frame}{Belief}
\protect\hypertarget{belief}{}
Beliefs are off-on; you either believe something or you don't.

\begin{itemize}[<+->]
\tightlist
\item
  Belief, in the way philosophers talk about it, is a fairly strong
  state.
\item
  To believe \emph{p} is to take \emph{p} as settled.
\end{itemize}
\end{frame}

\begin{frame}{Credence}
\protect\hypertarget{credence}{}
Credences are degrees of confidence.

\begin{itemize}[<+->]
\tightlist
\item
  There are a variety of arguments that these should behave like
  probabilities.
\item
  So to have credence 0.75 in something is to treat it as three times as
  likely as not.
\end{itemize}
\end{frame}

\begin{frame}{Credences and Bets}
\protect\hypertarget{credences-and-bets}{}
There is often taken to be a tight connection between credences and
betting behavior.

\begin{itemize}[<+->]
\tightlist
\item
  Having credence \emph{x} in \emph{p} is connected to being willing to
  pay \$x for a bet that pays \$1 if \emph{p} and pays nothing if
  \emph{p} is false.
\item
  In mid-C20 this was occasionally treated as a definition; that's
  absurd, but there's a connnection of some kind.
\end{itemize}
\end{frame}

\begin{frame}{Two Kinds of Reductionism}
\protect\hypertarget{two-kinds-of-reductionism}{}
\begin{enumerate}[<+->]
\tightlist
\item
  Reduce beliefs to credences.
\item
  Reduce credences to beliefs.
\end{enumerate}
\end{frame}

\begin{frame}{Reduce Beliefs to Credences}
\protect\hypertarget{reduce-beliefs-to-credences}{}
\begin{description}[<+->]
\tightlist
\item[Simple reduction]
To believe just is to have credence greater than 0.99 (or something like
that)
\item[Complicated reduction]
To believe just is to have credence higher than any salient alternative.
\end{description}
\end{frame}

\begin{frame}{Reduce Credence to Belief}
\protect\hypertarget{reduce-credence-to-belief}{}
To have a credence just is to have a belief about probabilities.

\begin{itemize}[<+->]
\tightlist
\item
  This isn't unheard of, but it's much less common than the previous
  reduction.
\end{itemize}
\end{frame}

\begin{frame}{Doing Without}
\protect\hypertarget{doing-without}{}
Two other views:

\begin{enumerate}[<+->]
\tightlist
\item
  Beliefs are an important part of philosophical psychology; credences
  are a myth.
\item
  Credences are an important part of philosophical psychology; beliefs
  are a myth.
\end{enumerate}
\end{frame}

\begin{frame}{Four options}
\protect\hypertarget{four-options}{}
\begin{itemize}[<+->]
\tightlist
\item
  Both are real, but beliefs are really just credences.
\item
  Both are real, but credences are really just beliefs.
\item
  Only credences are real.
\item
  Only beliefs are real.
\end{itemize}
\end{frame}

\begin{frame}{Dualism}
\protect\hypertarget{dualism}{}
None of these things are right!

\begin{itemize}[<+->]
\tightlist
\item
  Beliefs and credences are both real, and they play separate roles in
  psychology, such that neither can be reduced to the other.
\end{itemize}
\end{frame}

\begin{frame}{Historical Motivation}
\protect\hypertarget{historical-motivation}{}
Think about Descartes's idea that you should, for the sake of doxastic
hygeine, temporarily suspend all believing.

\begin{itemize}[<+->]
\tightlist
\item
  If you follow Descartes's directions, does it change any credences?
\item
  I think not; you still act the same way, including in betting
  behavior.
\end{itemize}
\end{frame}

\begin{frame}{Contemporary Motivation}
\protect\hypertarget{contemporary-motivation}{}
Both things seem important.

\begin{itemize}[<+->]
\tightlist
\item
  Credences play an important role in explaining behavior under
  uncertainty.
\item
  But it doesn't feel like we have purely probabilistic attitudes
  towards things like the existence of chairs and tables.
\end{itemize}
\end{frame}

\begin{frame}{Contemporary Motivation}
\protect\hypertarget{contemporary-motivation-1}{}
I'm not going to go over them all, but there are problems with every one
of the proposed reductions on the market.

\begin{itemize}[<+->]
\tightlist
\item
  So maybe reductionism fails.
\end{itemize}
\end{frame}

\begin{frame}{Pasnau's Motivation}
\protect\hypertarget{pasnaus-motivation}{}
Beliefs are sensitive to one's ``personality'', but credences are not.

\begin{itemize}[<+->]
\tightlist
\item
  A hopeful person can believe \emph{p} even though they can't rule all
  alternatives to \emph{p} out.
\item
  A pessimistic person won't believe \emph{p} even though it's very
  probable.
\item
  And that's fine; beliefs should be sensitive to personality like this.
\end{itemize}
\end{frame}

\begin{frame}{Pasnau's Dualism}
\protect\hypertarget{pasnaus-dualism}{}
This strikes me as similar to what I was saying about Descartes.

\begin{itemize}[<+->]
\tightlist
\item
  Descartes doesn't want people to lose confidence while they are going
  through the meditation process.
\item
  He just thinks that they should be pessimistic for a while.
\end{itemize}
\end{frame}

\begin{frame}{Pasnau's Dualism}
\protect\hypertarget{pasnaus-dualism-1}{}
But it comes at things from the opposite direction.

\begin{itemize}[<+->]
\tightlist
\item
  Since the Cartesian project fails, and we know it fails, we have to
  decide what to do next.
\item
  And the healthiest decision is to not worry about absurdly sceptical
  options.
\end{itemize}
\end{frame}

\begin{frame}{Is This Plausible}
\protect\hypertarget{is-this-plausible}{}
A couple of questions.

\begin{enumerate}[<+->]
\tightlist
\item
  Are there limits to hope? Is hoping that our senses are reliable
  within that hope?
\item
  What is the probability that we're not brains in vats? How does the
  positive theory avoid answering that question?
\end{enumerate}
\end{frame}

\begin{frame}{For Next Time}
\protect\hypertarget{for-next-time}{}
Onto Siegel's \emph{The Rationality of Perception}.
\end{frame}



\end{document}
