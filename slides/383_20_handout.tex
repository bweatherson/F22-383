% Options for packages loaded elsewhere
\PassOptionsToPackage{unicode}{hyperref}
\PassOptionsToPackage{hyphens}{url}
%
\documentclass[
  17pt,
  letterpaper,
  ignorenonframetext,
  aspectratio=169,
  handout,
  xcolor={dvipsnames}]{beamer}
\usepackage{pgfpages}
\setbeamertemplate{caption}[numbered]
\setbeamertemplate{caption label separator}{: }
\setbeamercolor{caption name}{fg=normal text.fg}
\beamertemplatenavigationsymbolsempty
% Prevent slide breaks in the middle of a paragraph
\widowpenalties 1 10000
\raggedbottom
\setbeamertemplate{part page}{
  \centering
  \begin{beamercolorbox}[sep=16pt,center]{part title}
    \usebeamerfont{part title}\insertpart\par
  \end{beamercolorbox}
}
\setbeamertemplate{section page}{
  \centering
  \begin{beamercolorbox}[sep=12pt,center]{part title}
    \usebeamerfont{section title}\insertsection\par
  \end{beamercolorbox}
}
\setbeamertemplate{subsection page}{
  \centering
  \begin{beamercolorbox}[sep=8pt,center]{part title}
    \usebeamerfont{subsection title}\insertsubsection\par
  \end{beamercolorbox}
}
\AtBeginPart{
  \frame{\partpage}
}
\AtBeginSection{
  \ifbibliography
  \else
    \frame{\sectionpage}
  \fi
}
\AtBeginSubsection{
  \frame{\subsectionpage}
}

\usepackage{amsmath,amssymb}
\usepackage{lmodern}
\usepackage{iftex}
\ifPDFTeX
  \usepackage[T1]{fontenc}
  \usepackage[utf8]{inputenc}
  \usepackage{textcomp} % provide euro and other symbols
\else % if luatex or xetex
  \usepackage{unicode-math}
  \defaultfontfeatures{Scale=MatchLowercase}
  \defaultfontfeatures[\rmfamily]{Ligatures=TeX,Scale=1}
  \setmainfont[BoldFont = SF Pro Text Semibold, Scale =
MatchLowercase]{SF Pro Text Light}
\fi
\usecolortheme{wolverine}
\usefonttheme{serif} % use mainfont rather than sansfont for slide text
\useinnertheme{default}
\useoutertheme{miniframes}
% Use upquote if available, for straight quotes in verbatim environments
\IfFileExists{upquote.sty}{\usepackage{upquote}}{}
\IfFileExists{microtype.sty}{% use microtype if available
  \usepackage[]{microtype}
  \UseMicrotypeSet[protrusion]{basicmath} % disable protrusion for tt fonts
}{}
\makeatletter
\@ifundefined{KOMAClassName}{% if non-KOMA class
  \IfFileExists{parskip.sty}{%
    \usepackage{parskip}
  }{% else
    \setlength{\parindent}{0pt}
    \setlength{\parskip}{6pt plus 2pt minus 1pt}}
}{% if KOMA class
  \KOMAoptions{parskip=half}}
\makeatother
\usepackage{xcolor}
\newif\ifbibliography
\setlength{\emergencystretch}{3em} % prevent overfull lines
\setcounter{secnumdepth}{-\maxdimen} % remove section numbering


\providecommand{\tightlist}{%
  \setlength{\itemsep}{0pt}\setlength{\parskip}{0pt}}\usepackage{longtable,booktabs,array}
\usepackage{calc} % for calculating minipage widths
\usepackage{caption}
% Make caption package work with longtable
\makeatletter
\def\fnum@table{\tablename~\thetable}
\makeatother
\usepackage{graphicx}
\makeatletter
\def\maxwidth{\ifdim\Gin@nat@width>\linewidth\linewidth\else\Gin@nat@width\fi}
\def\maxheight{\ifdim\Gin@nat@height>\textheight\textheight\else\Gin@nat@height\fi}
\makeatother
% Scale images if necessary, so that they will not overflow the page
% margins by default, and it is still possible to overwrite the defaults
% using explicit options in \includegraphics[width, height, ...]{}
\setkeys{Gin}{width=\maxwidth,height=\maxheight,keepaspectratio}
% Set default figure placement to htbp
\makeatletter
\def\fps@figure{htbp}
\makeatother

\captionsetup[figure]{labelformat=empty}
\usepackage{pgfpages}
\setbeamertemplate{itemize item}[circle]
\setbeamertemplate{footline}[frame number]{}
\mode<handout>{\pgfpagesuselayout{6 on 1}[letterpaper, border shrink=8mm]}
\AtBeginSection{%
   \begin{frame}
       \tableofcontents[currentsection]
   \end{frame}
}
\makeatletter
\makeatother
\makeatletter
\makeatother
\makeatletter
\@ifpackageloaded{caption}{}{\usepackage{caption}}
\AtBeginDocument{%
\ifdefined\contentsname
  \renewcommand*\contentsname{Table of contents}
\else
  \newcommand\contentsname{Table of contents}
\fi
\ifdefined\listfigurename
  \renewcommand*\listfigurename{List of Figures}
\else
  \newcommand\listfigurename{List of Figures}
\fi
\ifdefined\listtablename
  \renewcommand*\listtablename{List of Tables}
\else
  \newcommand\listtablename{List of Tables}
\fi
\ifdefined\figurename
  \renewcommand*\figurename{Figure}
\else
  \newcommand\figurename{Figure}
\fi
\ifdefined\tablename
  \renewcommand*\tablename{Table}
\else
  \newcommand\tablename{Table}
\fi
}
\@ifpackageloaded{float}{}{\usepackage{float}}
\floatstyle{ruled}
\@ifundefined{c@chapter}{\newfloat{codelisting}{h}{lop}}{\newfloat{codelisting}{h}{lop}[chapter]}
\floatname{codelisting}{Listing}
\newcommand*\listoflistings{\listof{codelisting}{List of Listings}}
\makeatother
\makeatletter
\@ifpackageloaded{caption}{}{\usepackage{caption}}
\@ifpackageloaded{subcaption}{}{\usepackage{subcaption}}
\makeatother
\makeatletter
\@ifpackageloaded{tcolorbox}{}{\usepackage[many]{tcolorbox}}
\makeatother
\makeatletter
\@ifundefined{shadecolor}{\definecolor{shadecolor}{rgb}{.97, .97, .97}}
\makeatother
\makeatletter
\makeatother
\ifLuaTeX
  \usepackage{selnolig}  % disable illegal ligatures
\fi
\IfFileExists{bookmark.sty}{\usepackage{bookmark}}{\usepackage{hyperref}}
\IfFileExists{xurl.sty}{\usepackage{xurl}}{} % add URL line breaks if available
\urlstyle{same} % disable monospaced font for URLs
\hypersetup{
  pdftitle={Knowledge and Reality, Lecture 20},
  pdfauthor={Brian Weatherson},
  hidelinks,
  pdfcreator={LaTeX via pandoc}}

\title{Knowledge and Reality, Lecture 20}
\author{Brian Weatherson}
\date{11/7/22}

\begin{document}
\frame{\titlepage}
\ifdefined\Shaded\renewenvironment{Shaded}{\begin{tcolorbox}[borderline west={3pt}{0pt}{shadecolor}, interior hidden, enhanced, frame hidden, breakable, boxrule=0pt, sharp corners]}{\end{tcolorbox}}\fi

\hypertarget{two-arguments-against-rationality-of-perception}{%
\section{Two Arguments against Rationality of
Perception}\label{two-arguments-against-rationality-of-perception}}

\begin{frame}{Section 3.1}
\protect\hypertarget{section-3.1}{}
I want to \textbf{start} today's lecture with two arguments from section
3.1, about why one might think Siegel's argument is wrong, and
experiences cannot be evaluated the way beliefs can.

\begin{enumerate}[<+->]
\tightlist
\item
  Backward-looking;
\item
  Forward-looking.
\end{enumerate}
\end{frame}

\begin{frame}{Backward-Looking}
\protect\hypertarget{backward-looking}{}
\begin{enumerate}[<+->]
\tightlist
\item
  Experiences are formed passively.
\item
  Beliefs are formed actively.
\item
  Only actively formed states are assessable as rational or irrational.
\end{enumerate}

\begin{enumerate}[<+->]
[A.]
\setcounter{enumi}{2}
\tightlist
\item
  So beliefs, but not experiences, are assessable as rational or
  irrational.
\end{enumerate}
\end{frame}

\begin{frame}{Siegel's Response}
\protect\hypertarget{siegels-response}{}
\begin{itemize}[<+->]
\tightlist
\item
  Premise 2 is ambiguous.
\item
  But on any plausible disambiguation, it is false.
\end{itemize}
\end{frame}

\begin{frame}{What Activity Might Be (1)}
\protect\hypertarget{what-activity-might-be-1}{}
Activity might be phenomenological; we feel ourselves forming beliefs.

\begin{itemize}[<+->]
\tightlist
\item
  But this only applies to a small fraction of our beliefs.
\item
  And the ones it doesn't apply to are still capable of being rational
  or irrational.
\end{itemize}
\end{frame}

\begin{frame}{What Activity Might Be (2)}
\protect\hypertarget{what-activity-might-be-2}{}
Activity might mean involving reasoning; our beliefs come from
reasoning.

\begin{itemize}[<+->]
\tightlist
\item
  Again, this is true for only a small fraction of our beliefs.
\item
  You didn't reason to the conclusion that there are words on the screen
  now.
\item
  But all beliefs, even the not-formed-by-reasoning ones, can be
  rational or irrational.
\end{itemize}
\end{frame}

\begin{frame}{What Activity Might Be (3)}
\protect\hypertarget{what-activity-might-be-3}{}
Activity might mean involving reflection.

\begin{itemize}[<+->]
\tightlist
\item
  Even if you didn't reason to the belief that there are words on the
  screen, or in any sense \emph{reflect} before forming that belief, you
  \emph{could have} reflected on it.
\item
  Maybe belief is active in that sense.
\end{itemize}
\end{frame}

\begin{frame}{What Activity Might Be (3)}
\protect\hypertarget{what-activity-might-be-3-1}{}
But again, not everything can be reflective.

\begin{itemize}[<+->]
\tightlist
\item
  Toddlers don't have this kind of capacity for reflection, but can have
  rational beliefs.
\end{itemize}
\end{frame}

\begin{frame}{Forward-Looking}
\protect\hypertarget{forward-looking}{}
\begin{enumerate}[<+->]
\tightlist
\item
  Experiences cannot be adjusted.
\item
  Beliefs can be adjusted.
\item
  Being adjustable is necessary for being assessable for rationality.
\end{enumerate}

\begin{enumerate}[<+->]
[A.]
\setcounter{enumi}{2}
\tightlist
\item
  So beliefs, but not experiences, are assessable as rational.
\end{enumerate}
\end{frame}

\begin{frame}{What Might Adjustable Mean Here}
\protect\hypertarget{what-might-adjustable-mean-here}{}
\begin{enumerate}[<+->]
\tightlist
\item
  Subject to deliberation
\item
  Capable of being disowned
\item
  Change by habituation
\end{enumerate}
\end{frame}

\begin{frame}{What Might Adjustable Mean (1)}
\protect\hypertarget{what-might-adjustable-mean-1}{}
If we mean that the believer can deliberate their way out of them, then
delusional beliefs are not rational or irrational.

\begin{itemize}[<+->]
\tightlist
\item
  But in fact they are irrational.
\item
  NB: I'm not so sure here; some of the cases Siegel mentions (like
  Capgras) feel almost \emph{arational}.
\end{itemize}
\end{frame}

\begin{frame}{What Might Adjustable Mean (2)}
\protect\hypertarget{what-might-adjustable-mean-2}{}
If we mean by adjustable that they can be disowned, this doesn't
distinguish experience from belief.

\begin{itemize}[<+->]
\tightlist
\item
  Experiences can be disowned.
\item
  This isn't in the sense that you don't have them (again, think of the
  checker-shadow), but that you don't act on them.
\end{itemize}
\end{frame}

\begin{frame}{What Might Adjustable Mean (2)}
\protect\hypertarget{what-might-adjustable-mean-2-1}{}
Note that this is a change from the previous 4 things we looked at.

\begin{itemize}[<+->]
\tightlist
\item
  Now we're denying that experiences lack the property in question,
  rather than that beliefs have the property.
\end{itemize}

\end{frame} \begin{frame}[plain]

In the case of belief, ceasing to rely on a belief can't come apart from
ceasing to have the belief.

\begin{itemize}[<+->]
\tightlist
\item
  This doesn't seem right to me.
\item
  A good juror can cease to rely on a belief from outside the court
  without ceasing to have it.
\item
  There are hard questions here about what it means to rely on a belief,
  but they are practically significant.
\end{itemize}
\end{frame}

\begin{frame}{What Might Adjustable Mean (3)}
\protect\hypertarget{what-might-adjustable-mean-3}{}
Maybe we can habituate ourselves into not forming beliefs a certain way.

\begin{itemize}[<+->]
\tightlist
\item
  But it's even more plausible that we can habituate ourselves into not
  experiencing things a certain way.
\item
  We can learn to hear an instrument as out of tune, to see a face as
  expressing a different emotion, and so on.
\end{itemize}
\end{frame}

\hypertarget{setting-up-chapter-4}{%
\section{Setting Up Chapter 4}\label{setting-up-chapter-4}}

\begin{frame}{The Pink Drink and the Bird}
\protect\hypertarget{the-pink-drink-and-the-bird}{}
Why are there two examples here and not just one?

\begin{itemize}[<+->]
\tightlist
\item
  What differences are there between the examples?
\item
  Why do those differences matter?
\end{itemize}
\end{frame}

\begin{frame}{The Pink Drink and the Bird}
\protect\hypertarget{the-pink-drink-and-the-bird-1}{}
I guess it's because she wants to be explicit that she cares about both
perception of objects (like the drink) and perception of events (like
the flying).

\begin{itemize}[<+->]
\tightlist
\item
  Is there any reason to think these would pattern differently with
  respect to what we care about here?
\end{itemize}
\end{frame}

\begin{frame}{Reason-Power and Forward-Looking Power}
\protect\hypertarget{reason-power-and-forward-looking-power}{}
Just what is the difference between these?

\begin{itemize}[<+->]
\tightlist
\item
  Let's start with the text.
\end{itemize}

\end{frame} \begin{frame}[plain]

\begin{itemize}[<+->]
\tightlist
\item
  Your visual experience gives you excellent reason to believe that the
  drink is pink.
\item
  Your visual experience (in which the bird looks to be moving) gives
  you excellent reason to believe that the bird is flying away.
\end{itemize}

Page 60
\end{frame}

\begin{frame}{Reason-Power}
\protect\hypertarget{reason-power}{}
Experienes give you excellent reason to belief in their contents (or
something like them).
\end{frame}

\begin{frame}{Two Kinds of Reason}
\protect\hypertarget{two-kinds-of-reason}{}
\begin{itemize}[<+->]
\tightlist
\item
  Motivating
\item
  Justifying
\item
  Siegel is, I think, interested in \textbf{justifying} reasons.
\end{itemize}
\end{frame}

\begin{frame}{Forward-Looking Power}
\protect\hypertarget{forward-looking-power}{}
Again, start with some quotes

\end{frame} \begin{frame}[plain]

\begin{itemize}[<+->]
\tightlist
\item
  If you form the belief that the drink is pink on the basis of the
  visual experience, ceteris paribus, you'll have a well-founded
  belief.\\
\item
  If you form the belief that the bird is flying away on the basis of
  your visual experience (in which the bird looks to be moving), ceteris
  paribus, you'll have a well-founded belief.
\end{itemize}

Page 60
\end{frame}

\begin{frame}{Forward-Looking Power}
\protect\hypertarget{forward-looking-power-1}{}
Experiences tend to make beliefs in their contents well-founded.
\end{frame}

\begin{frame}{What is the Difference}
\protect\hypertarget{what-is-the-difference}{}
\begin{itemize}[<+->]
\tightlist
\item
  Providing a reason for the belief vs making the belief well-founded;
\item
  This is sort of a backwards-looking vs forward-looking distinction;
\item
  Except it's looking forward to a backwards looking thing.
\end{itemize}
\end{frame}

\begin{frame}{The Powers Together}
\protect\hypertarget{the-powers-together}{}
You'd think they go together fairly tightly.

\begin{itemize}[<+->]
\tightlist
\item
  As Siegel says, experiences typically seem to have sort of both.
\item
  And the forward-looking power might be explained by the reason-power.
\end{itemize}
\end{frame}

\begin{frame}{Reason-Power without Forward-Looking Power}
\protect\hypertarget{reason-power-without-forward-looking-power}{}
Maybe if:

\begin{enumerate}[<+->]
[(a)]
\tightlist
\item
  The reason provided is very very weak; and
\item
  There is independent reason to believe otherwise.
\end{enumerate}
\end{frame}

\begin{frame}{Forward-Looking Power without Reason-Power}
\protect\hypertarget{forward-looking-power-without-reason-power}{}
Maybe if beliefs can be well-founded by things other than reasons.

\begin{itemize}[<+->]
\tightlist
\item
  We get deep into murky waters about the metaphysics of reasons here,
  and I'm not going to go firther.
\item
  Footnote 5 suggests this is why Siegel is not going further this way
  either.
\end{itemize}
\end{frame}

\hypertarget{four-theories-of-perceptual-justification}{%
\section{Four Theories of Perceptual
Justification}\label{four-theories-of-perceptual-justification}}

\begin{frame}{Four Theories}
\protect\hypertarget{four-theories}{}
\begin{enumerate}[<+->]
\tightlist
\item
  Disjunctivist/Naïve Realist
\item
  Reliabilist
\item
  Inferentialist
\item
  Dogmatist
\end{enumerate}
\end{frame}

\begin{frame}{Disjunctivist}
\protect\hypertarget{disjunctivist}{}
\begin{itemize}[<+->]
\tightlist
\item
  Appearances/experiences on their own have little epistemic charge.
\item
  What has power is \textbf{perception}, where this is understood as a
  success term.
\item
  This is very externalist; what is happening on an occasion, and what
  force it has, depends on external factors.
\end{itemize}
\end{frame}

\begin{frame}{Reliabilist}
\protect\hypertarget{reliabilist}{}
\begin{itemize}[<+->]
\tightlist
\item
  Anything can provide positive charge as long as it is reliably tied to
  reality.
\item
  Typically, experiences are reliably tied to reality.
\item
  There is nothing particularly special about perception.
\end{itemize}
\end{frame}

\begin{frame}{Inferentialist}
\protect\hypertarget{inferentialist}{}
\begin{itemize}[<+->]
\tightlist
\item
  On their own, experiences just provide positive charge for the
  proposition that one is having the experience.
\item
  Extra step needed to get to claims about the external world.
\item
  Lots of options for next step.
\end{itemize}
\end{frame}

\begin{frame}{Inferentialist}
\protect\hypertarget{inferentialist-1}{}
\begin{itemize}[<+->]
\tightlist
\item
  One choice: what is the link claim? Presumably something about
  reliable connection.
\item
  Second choice: how is the link claim grounded? IBE, Basic, something
  else?
\item
  Third choice: does the individual perceiver have to appreciate the
  ground?
\end{itemize}
\end{frame}

\begin{frame}{Dogmatist}
\protect\hypertarget{dogmatist}{}
\begin{itemize}[<+->]
\tightlist
\item
  In the first instance, experiences provide positive charge for the
  proposition that one is having the experience.
\item
  But unless something stops them, they also provide positive charge for
  external world propositions.
\item
  And the `something' has to be accessible to the perceiver.
\end{itemize}
\end{frame}

\begin{frame}{Dogmatist}
\protect\hypertarget{dogmatist-1}{}
The big difference with the inferentialist concerns presence vs absence
of reasons.

\begin{itemize}[<+->]
\tightlist
\item
  The inferentialist thinks you need a positive reason to go from Looks
  \emph{p} to \emph{p}.
\item
  The dogmatist thinks you need an absence of defeating reasons to go
  from Looks \emph{p} to \emph{p}.
\end{itemize}
\end{frame}

\begin{frame}{How They Play with Rationality of Perception}
\protect\hypertarget{how-they-play-with-rationality-of-perception}{}
Disjunctivism is no problem.

\begin{itemize}[<+->]
\tightlist
\item
  In the bad case you don't have real perception, just apparent
  perception.
\item
  So there isn't much charge there.
\end{itemize}
\end{frame}

\begin{frame}{How They Play with Rationality of Perception}
\protect\hypertarget{how-they-play-with-rationality-of-perception-1}{}
Reliabilism isn't much of a problem.

\begin{itemize}[<+->]
\tightlist
\item
  Provided we get the \textbf{reference class} right, the bad cases will
  be actually unreliable.
\item
  Bit of a trick here about getting the reference classes right, but not
  a big deal.
\end{itemize}
\end{frame}

\begin{frame}{How They Play with Rationality of Perception}
\protect\hypertarget{how-they-play-with-rationality-of-perception-2}{}
Inferentialism isn't much of a problem.

\begin{itemize}[<+->]
\tightlist
\item
  Provided the `link' is defeasible, and doesn't work in all cases, you
  can easily get that the support fails.
\end{itemize}
\end{frame}

\begin{frame}{How They Play with Rationality of Perception}
\protect\hypertarget{how-they-play-with-rationality-of-perception-3}{}
Dogmatism does look like a problem.

\begin{itemize}[<+->]
\tightlist
\item
  Hijacked perception lacks defeaters that are apparent to the
  perceiver.
\item
  So the dogmatist thinks they have full charge.
\item
  But they don't.
\end{itemize}
\end{frame}

\begin{frame}{Two Dogmatist Responses}
\protect\hypertarget{two-dogmatist-responses}{}
\begin{enumerate}[<+->]
\tightlist
\item
  Maybe the perceiver could tell there was a problem; this seems
  optimistic.
\item
  Maybe dogmatism just applies to a much narrower band of properties.
\end{enumerate}
\end{frame}

\begin{frame}{Dogmatism and Perception}
\protect\hypertarget{dogmatism-and-perception}{}
Most actual dogmatists don't think we really \textbf{perceive} things
like that something is a gun or a power-tool.

\begin{itemize}[<+->]
\tightlist
\item
  They think we just perceive things like shapes and colors.
\item
  This might be an implausible theory of perception, but it makes it
  seem more plausible that they couldn't be hijacked.
\end{itemize}
\end{frame}

\begin{frame}{For Next Time}
\protect\hypertarget{for-next-time}{}
Chapter 5
\end{frame}



\end{document}
