% Options for packages loaded elsewhere
\PassOptionsToPackage{unicode}{hyperref}
\PassOptionsToPackage{hyphens}{url}
%
\documentclass[
  17pt,
  letterpaper,
  ignorenonframetext,
  aspectratio=169,
  xcolor={dvipsnames}]{beamer}
\usepackage{pgfpages}
\setbeamertemplate{caption}[numbered]
\setbeamertemplate{caption label separator}{: }
\setbeamercolor{caption name}{fg=normal text.fg}
\beamertemplatenavigationsymbolsempty
% Prevent slide breaks in the middle of a paragraph
\widowpenalties 1 10000
\raggedbottom
\setbeamertemplate{part page}{
  \centering
  \begin{beamercolorbox}[sep=16pt,center]{part title}
    \usebeamerfont{part title}\insertpart\par
  \end{beamercolorbox}
}
\setbeamertemplate{section page}{
  \centering
  \begin{beamercolorbox}[sep=12pt,center]{part title}
    \usebeamerfont{section title}\insertsection\par
  \end{beamercolorbox}
}
\setbeamertemplate{subsection page}{
  \centering
  \begin{beamercolorbox}[sep=8pt,center]{part title}
    \usebeamerfont{subsection title}\insertsubsection\par
  \end{beamercolorbox}
}
\AtBeginPart{
  \frame{\partpage}
}
\AtBeginSection{
  \ifbibliography
  \else
    \frame{\sectionpage}
  \fi
}
\AtBeginSubsection{
  \frame{\subsectionpage}
}

\usepackage{amsmath,amssymb}
\usepackage{lmodern}
\usepackage{iftex}
\ifPDFTeX
  \usepackage[T1]{fontenc}
  \usepackage[utf8]{inputenc}
  \usepackage{textcomp} % provide euro and other symbols
\else % if luatex or xetex
  \usepackage{unicode-math}
  \defaultfontfeatures{Scale=MatchLowercase}
  \defaultfontfeatures[\rmfamily]{Ligatures=TeX,Scale=1}
  \setmainfont[BoldFont = SF Pro Text Semibold, Scale =
MatchLowercase]{SF Pro Text Light}
\fi
\usecolortheme{wolverine}
\usefonttheme{serif} % use mainfont rather than sansfont for slide text
\useinnertheme{default}
\useoutertheme{miniframes}
% Use upquote if available, for straight quotes in verbatim environments
\IfFileExists{upquote.sty}{\usepackage{upquote}}{}
\IfFileExists{microtype.sty}{% use microtype if available
  \usepackage[]{microtype}
  \UseMicrotypeSet[protrusion]{basicmath} % disable protrusion for tt fonts
}{}
\makeatletter
\@ifundefined{KOMAClassName}{% if non-KOMA class
  \IfFileExists{parskip.sty}{%
    \usepackage{parskip}
  }{% else
    \setlength{\parindent}{0pt}
    \setlength{\parskip}{6pt plus 2pt minus 1pt}}
}{% if KOMA class
  \KOMAoptions{parskip=half}}
\makeatother
\usepackage{xcolor}
\newif\ifbibliography
\setlength{\emergencystretch}{3em} % prevent overfull lines
\setcounter{secnumdepth}{-\maxdimen} % remove section numbering


\providecommand{\tightlist}{%
  \setlength{\itemsep}{0pt}\setlength{\parskip}{0pt}}\usepackage{longtable,booktabs,array}
\usepackage{calc} % for calculating minipage widths
\usepackage{caption}
% Make caption package work with longtable
\makeatletter
\def\fnum@table{\tablename~\thetable}
\makeatother
\usepackage{graphicx}
\makeatletter
\def\maxwidth{\ifdim\Gin@nat@width>\linewidth\linewidth\else\Gin@nat@width\fi}
\def\maxheight{\ifdim\Gin@nat@height>\textheight\textheight\else\Gin@nat@height\fi}
\makeatother
% Scale images if necessary, so that they will not overflow the page
% margins by default, and it is still possible to overwrite the defaults
% using explicit options in \includegraphics[width, height, ...]{}
\setkeys{Gin}{width=\maxwidth,height=\maxheight,keepaspectratio}
% Set default figure placement to htbp
\makeatletter
\def\fps@figure{htbp}
\makeatother

\captionsetup[figure]{labelformat=empty}
\usepackage{pgfpages}
\setbeamertemplate{itemize item}[circle]
\setbeamertemplate{footline}[frame number]{}
\mode<handout>{\pgfpagesuselayout{6 on 1}[letterpaper, border shrink=8mm]}
\AtBeginSection{%
   \begin{frame}
       \tableofcontents[currentsection]
   \end{frame}
}
\makeatletter
\makeatother
\makeatletter
\makeatother
\makeatletter
\@ifpackageloaded{caption}{}{\usepackage{caption}}
\AtBeginDocument{%
\ifdefined\contentsname
  \renewcommand*\contentsname{Table of contents}
\else
  \newcommand\contentsname{Table of contents}
\fi
\ifdefined\listfigurename
  \renewcommand*\listfigurename{List of Figures}
\else
  \newcommand\listfigurename{List of Figures}
\fi
\ifdefined\listtablename
  \renewcommand*\listtablename{List of Tables}
\else
  \newcommand\listtablename{List of Tables}
\fi
\ifdefined\figurename
  \renewcommand*\figurename{Figure}
\else
  \newcommand\figurename{Figure}
\fi
\ifdefined\tablename
  \renewcommand*\tablename{Table}
\else
  \newcommand\tablename{Table}
\fi
}
\@ifpackageloaded{float}{}{\usepackage{float}}
\floatstyle{ruled}
\@ifundefined{c@chapter}{\newfloat{codelisting}{h}{lop}}{\newfloat{codelisting}{h}{lop}[chapter]}
\floatname{codelisting}{Listing}
\newcommand*\listoflistings{\listof{codelisting}{List of Listings}}
\makeatother
\makeatletter
\@ifpackageloaded{caption}{}{\usepackage{caption}}
\@ifpackageloaded{subcaption}{}{\usepackage{subcaption}}
\makeatother
\makeatletter
\@ifpackageloaded{tcolorbox}{}{\usepackage[many]{tcolorbox}}
\makeatother
\makeatletter
\@ifundefined{shadecolor}{\definecolor{shadecolor}{rgb}{.97, .97, .97}}
\makeatother
\makeatletter
\makeatother
\ifLuaTeX
  \usepackage{selnolig}  % disable illegal ligatures
\fi
\IfFileExists{bookmark.sty}{\usepackage{bookmark}}{\usepackage{hyperref}}
\IfFileExists{xurl.sty}{\usepackage{xurl}}{} % add URL line breaks if available
\urlstyle{same} % disable monospaced font for URLs
\hypersetup{
  pdftitle={Knowledge and Reality, Lecture 23},
  pdfauthor={Brian Weatherson},
  hidelinks,
  pdfcreator={LaTeX via pandoc}}

\title{Knowledge and Reality, Lecture 23}
\author{Brian Weatherson}
\date{11/21/22}

\begin{document}
\frame{\titlepage}
\ifdefined\Shaded\renewenvironment{Shaded}{\begin{tcolorbox}[frame hidden, boxrule=0pt, breakable, borderline west={3pt}{0pt}{shadecolor}, enhanced, interior hidden, sharp corners]}{\end{tcolorbox}}\fi

\begin{frame}{Plan}
\protect\hypertarget{plan}{}
\begin{enumerate}[<+->]
\tightlist
\item
  Selection Effects
\item
  Fear and Loathing and Inference
\end{enumerate}
\end{frame}

\hypertarget{selection-effects}{%
\section{Selection Effects}\label{selection-effects}}

\begin{frame}{A Kind of Irrationality}
\protect\hypertarget{a-kind-of-irrationality}{}
\begin{itemize}[<+->]
\tightlist
\item
  Observer focuses on a particular kind of feature.
\item
  From the things they observe, they draw unbalanced conclusions.
\item
  Siegel argues that this can be irrational perception.
\end{itemize}
\end{frame}

\begin{frame}{A Contrast Case}
\protect\hypertarget{a-contrast-case}{}
I've described cases like this earlier, but in order to better
understand Siegel's argument, I think it's good to have a contrast case.

\begin{itemize}[<+->]
\tightlist
\item
  A is trying to predict an upcoming election, so decides to conduct a
  little opinion poll.
\item
  Unfortunately, a lot of things go wrong.
\end{itemize}
\end{frame}

\begin{frame}{Polling Example}
\protect\hypertarget{polling-example}{}
A decides to conduct a poll of students on the UM quad.

\begin{itemize}[<+->]
\tightlist
\item
  Unsurprisingly, they find a lot of Democrats.
\item
  So they conclude that the Democratic candidate will win the election.
\end{itemize}
\end{frame}

\begin{frame}{Three Steps}
\protect\hypertarget{three-steps}{}
\begin{enumerate}[<+->]
\tightlist
\item
  Deciding where to conduct the poll.
\item
  The conduct of the poll itself.
\item
  The inferences drawn from the poll
\end{enumerate}
\end{frame}

\begin{frame}{Three Steps}
\protect\hypertarget{three-steps-1}{}
\begin{enumerate}[<+->]
\tightlist
\item
  Deciding where to conduct the poll.
\end{enumerate}

\begin{itemize}[<+->]
\tightlist
\item
  This seems like an obvious mistake; it's a really really non-random
  sample.
\item
  It is what people would normally call a selection effect.
\item
  So here's one irrational step.
\end{itemize}
\end{frame}

\begin{frame}{Three Steps}
\protect\hypertarget{three-steps-2}{}
\begin{enumerate}[<+->]
\setcounter{enumi}{2}
\tightlist
\item
  The inferences drawn from the poll
\end{enumerate}

\begin{itemize}[<+->]
\tightlist
\item
  This is a really really bad mistake.
\item
  And it's uncontroversially an epistemic mistake.
\end{itemize}
\end{frame}

\begin{frame}{Three Steps}
\protect\hypertarget{three-steps-3}{}
\begin{enumerate}[<+->]
\setcounter{enumi}{1}
\tightlist
\item
  The conduct of the poll itself
\end{enumerate}

\begin{itemize}[<+->]
\tightlist
\item
  We could imagine that this step was conducted well.
\item
  Of course they could mess this up too; wishful hearing on what someone
  says, or how enthusiastic they are.
\item
  But it could be good.
\end{itemize}
\end{frame}

\begin{frame}{Irrational Perception}
\protect\hypertarget{irrational-perception}{}
I think this is pretty clearly \emph{not} a case of irrational
perception.

\begin{itemize}[<+->]
\tightlist
\item
  There are mistakes either side of the perception.
\item
  But the perceptual step, step 2, is not a mistake.
\end{itemize}
\end{frame}

\begin{frame}{Analogy}
\protect\hypertarget{analogy}{}
A critic might say that Siegel's example is just like the polling case.

\begin{itemize}[<+->]
\tightlist
\item
  Typically choosing to focus on particular features is a bad place to
  look, like polling just on the diag.
\item
  Not accounting for the biased sample is an epistemic error.
\end{itemize}
\end{frame}

\begin{frame}{Analogy}
\protect\hypertarget{analogy-1}{}
A critic might say that Siegel's example is just like the polling case.

\begin{itemize}[<+->]
\tightlist
\item
  But observing the negative features is good perception, just like
  correctly polling.
\end{itemize}
\end{frame}

\begin{frame}{Siegel's Two Responses}
\protect\hypertarget{siegels-two-responses}{}
\begin{enumerate}[<+->]
\tightlist
\item
  What Other Kind of Error is It?
\item
  What happens when steps 1 and 2 are merged?
\end{enumerate}
\end{frame}

\begin{frame}{Error Kinds}
\protect\hypertarget{error-kinds}{}
Here's a standard philosophical picture of rationality.

\begin{itemize}[<+->]
\tightlist
\item
  There are two kinds of rationality: epistemic and prudential.
\item
  Prudential rationality is a matter of getting what one wants.
\item
  Epistemic rationality is, well, everything else.
\end{itemize}
\end{frame}

\begin{frame}{Error Kinds and the Inattentive Looker}
\protect\hypertarget{error-kinds-and-the-inattentive-looker}{}
\begin{itemize}[<+->]
\tightlist
\item
  Just focusing on the negative qualities of the job applicant is an
  error.
\item
  So it's either an epistemic or a prudential error.
\item
  Might it might be in the interests of the looker, in which case it
  isn't prudential.
\item
  So it's epistemic.
\end{itemize}
\end{frame}

\begin{frame}{Error Kinds and Polling}
\protect\hypertarget{error-kinds-and-polling}{}
To understand this argument, it helps to apply it to the polling case.

\begin{itemize}[<+->]
\tightlist
\item
  Maybe it is in the interest of the poller to get a pro-Democrats
  response; that will make their bosses pleased or something.
\item
  If so, are all the mistakes epistemic?
\end{itemize}
\end{frame}

\begin{frame}{Two Replies}
\protect\hypertarget{two-replies}{}
\begin{enumerate}[<+->]
\tightlist
\item
  Mistakes about where to look feel like a distinctive kind of error,
  neither epistemic nor practical, which raise doubts for the first
  assumption of the argument.
\item
  Even if that's not right, the errors still feel like they take place
  either side of the perception, not in the perception itself.
\end{enumerate}
\end{frame}

\begin{frame}{Norms of Inquiry}
\protect\hypertarget{norms-of-inquiry}{}
There's a big question here about how to think about mistakes in
inquiry.

\begin{itemize}[<+->]
\tightlist
\item
  This is a very interesting topic in contemporary philosophy. (Jane
  Friedman at NYU is the leading figure in this debate, but it's a
  really fascinating one.)
\end{itemize}
\end{frame}

\begin{frame}{Norms of Inquiry}
\protect\hypertarget{norms-of-inquiry-1}{}
There's a big question here about how to think about mistakes in
inquiry.

\begin{itemize}[<+->]
\tightlist
\item
  Could biased perception be bad in the distinctive way biased inquiry
  is bad, which isn't exactly epistemic rationality or prudential
  rationality?
\end{itemize}
\end{frame}

\begin{frame}{Errors Either Side}
\protect\hypertarget{errors-either-side}{}
Even if we grant that the choice to focus on one set of features (or one
set of voters) is an epistemic error, it doesn't follow that the
perception of those features (or voters) is an epistemic error.

\begin{itemize}[<+->]
\tightlist
\item
  It might be that step was done well, even if either side of it was a
  mistake.
\end{itemize}
\end{frame}

\begin{frame}{Siegel's Second Argument}
\protect\hypertarget{siegels-second-argument}{}
Siegel has a second argument that addresses this point.

\begin{itemize}[<+->]
\tightlist
\item
  One disanalogy between the perception case and the polling case is
  that in the polling case, the three steps are clearly distinguished in
  time.
\item
  That might not be true here.
\end{itemize}
\end{frame}

\begin{frame}{Perceptions of Distributions}
\protect\hypertarget{perceptions-of-distributions}{}
Think about the discussion she has of seeing that there are exactly
three pens on the desk, or exactly three eggs in the carton.

\begin{itemize}[<+->]
\tightlist
\item
  This isn't just a perception of the three eggs.
\item
  It is a perception that there are no more eggs.
\end{itemize}
\end{frame}

\begin{frame}{Perceptions of Distributions}
\protect\hypertarget{perceptions-of-distributions-1}{}
Siegel really means this literally.

\begin{itemize}[<+->]
\tightlist
\item
  She thinks it can be part of the content of the perception that there
  are three eggs and no more in the carton.
\item
  This sounds plausible, but it isn't uncontentious.
\end{itemize}
\end{frame}

\begin{frame}{Perceptions of Distributions}
\protect\hypertarget{perceptions-of-distributions-2}{}
\begin{itemize}[<+->]
\tightlist
\item
  In the polling case, it would be absurd to say that one has the
  perception that these are a representative sample of Michigan voters.
\item
  Even if one believed that, it couldn't possibly be the content of a
  perception.
\end{itemize}
\end{frame}

\begin{frame}{Perceptions of Distributions}
\protect\hypertarget{perceptions-of-distributions-3}{}
\begin{itemize}[<+->]
\tightlist
\item
  But maybe in the biased observation case it could be.
\item
  The person doing the hiring might not just perceive (correctly!) the
  negative features of the candidate.
\item
  They might perceive that the candidate is a certain kind of candidate,
  with these representative features.
\end{itemize}
\end{frame}

\begin{frame}{Big Assumptions Here}
\protect\hypertarget{big-assumptions-here}{}
\begin{enumerate}[<+->]
\tightlist
\item
  This kind of biased inquiry is an epistemic failing, not some other
  kind of failing.
\item
  The perceiver doesn't just see the negative features and infer they
  are representative, the perception includes both those things at once.
\end{enumerate}

\begin{itemize}[<+->]
\tightlist
\item
  If you spot both of those features, then it seems plausible to say
  that the biased perception is irrational.
\end{itemize}
\end{frame}

\begin{frame}{Objections}
\protect\hypertarget{objections}{}
The big thing to note is that Siegel's critics here very much do not say
that the biased perceiver is rational. Instead they say one of two
things.

\begin{enumerate}[<+->]
\tightlist
\item
  The biased perceiver is making a practical error, or an inquiry error,
  or some other kind of error.
\end{enumerate}
\end{frame}

\begin{frame}{Objections}
\protect\hypertarget{objections-1}{}
The big thing to note is that Siegel's critics here very much do not say
that the biased perceiver is rational. Instead they say one of two
things.

\begin{enumerate}[<+->]
\setcounter{enumi}{1}
\tightlist
\item
  The biased perceiver is making an epistemic error in not correcting
  for the non-representative nature of their sample. This is epistemic,
  but it's an epistemic error in post-perceptual processing.
\end{enumerate}
\end{frame}

\hypertarget{fear}{%
\section{Fear}\label{fear}}

\begin{frame}{Three Uncontroversial Claims}
\protect\hypertarget{three-uncontroversial-claims}{}
\begin{enumerate}[<+->]
\tightlist
\item
  Sometimes fear can cause us to see the world a certain way.
\item
  Sometimes the fears in question can be irrational, at least in their
  intensity.
\item
  Fears involve seeing the world a certain way; they involve an outlook.
\end{enumerate}
\end{frame}

\begin{frame}{Two More Controversial Claims}
\protect\hypertarget{two-more-controversial-claims}{}
\begin{enumerate}[<+->]
\setcounter{enumi}{3}
\tightlist
\item
  The link from fear to perception is an \textbf{inference}.
\item
  The \emph{outlook} involved in fear is a belief about how the world
  is.
\end{enumerate}
\end{frame}

\begin{frame}{Fear and Outlooks}
\protect\hypertarget{fear-and-outlooks}{}
What is the outlook involved in fear?

\begin{itemize}[<+->]
\tightlist
\item
  Siegel says it is treating certain risks as particularly \emph{live}
  or \emph{salient}.
\item
  This is important, because these are risks; simply believing they are
  risks isn't irrational.
\end{itemize}
\end{frame}

\begin{frame}{Fear and Outlooks}
\protect\hypertarget{fear-and-outlooks-1}{}
For a transition to be an inference, it must be that:

\begin{enumerate}[<+->]
\tightlist
\item
  The fear is an information state;
\item
  This information includes something about liveness or salience.
\item
  The subsequent perception is caused by this information.
\end{enumerate}
\end{frame}

\begin{frame}{Alternative View}
\protect\hypertarget{alternative-view}{}
Maybe fear isn't a belief, it's a distinctive kind of attitude.

\begin{itemize}[<+->]
\tightlist
\item
  Compare wanting.
\item
  Some people think wanting X just is believing that X is
  valuable/wantable.
\item
  But a more common view is that wanting X is just an attitude towards
  X, and what makes it a desire is the kind of attitude it is.
\end{itemize}
\end{frame}

\begin{frame}{Alternative View}
\protect\hypertarget{alternative-view-1}{}
Maybe fearing snakes is just an attitude towards snakes, rather than to
the proposition that the danger posed by snakes is particularly live or
salient.

\begin{itemize}[<+->]
\tightlist
\item
  I'm not sure that Siegel's inferential view would go through on this
  account.
\end{itemize}
\end{frame}

\begin{frame}{For Next Time}
\protect\hypertarget{for-next-time}{}
We'll possibly return to the selection effects (for people who were away
for Thanksgiving!) and wrap up chapter 10.
\end{frame}



\end{document}
