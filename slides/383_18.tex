% Options for packages loaded elsewhere
\PassOptionsToPackage{unicode}{hyperref}
\PassOptionsToPackage{hyphens}{url}
%
\documentclass[
  17pt,
  letterpaper,
  ignorenonframetext,
  aspectratio=169,
]{beamer}
\usepackage{pgfpages}
\setbeamertemplate{caption}[numbered]
\setbeamertemplate{caption label separator}{: }
\setbeamercolor{caption name}{fg=normal text.fg}
\beamertemplatenavigationsymbolsempty
% Prevent slide breaks in the middle of a paragraph
\widowpenalties 1 10000
\raggedbottom
\setbeamertemplate{part page}{
  \centering
  \begin{beamercolorbox}[sep=16pt,center]{part title}
    \usebeamerfont{part title}\insertpart\par
  \end{beamercolorbox}
}
\setbeamertemplate{section page}{
  \centering
  \begin{beamercolorbox}[sep=12pt,center]{part title}
    \usebeamerfont{section title}\insertsection\par
  \end{beamercolorbox}
}
\setbeamertemplate{subsection page}{
  \centering
  \begin{beamercolorbox}[sep=8pt,center]{part title}
    \usebeamerfont{subsection title}\insertsubsection\par
  \end{beamercolorbox}
}
\AtBeginPart{
  \frame{\partpage}
}
\AtBeginSection{
  \ifbibliography
  \else
    \frame{\sectionpage}
  \fi
}
\AtBeginSubsection{
  \frame{\subsectionpage}
}

\usepackage{amsmath,amssymb}
\usepackage{lmodern}
\usepackage{iftex}
\ifPDFTeX
  \usepackage[T1]{fontenc}
  \usepackage[utf8]{inputenc}
  \usepackage{textcomp} % provide euro and other symbols
\else % if luatex or xetex
  \usepackage{unicode-math}
  \defaultfontfeatures{Scale=MatchLowercase}
  \defaultfontfeatures[\rmfamily]{Ligatures=TeX,Scale=1}
  \setmainfont[BoldFont = SF Pro Text Semibold, Scale =
MatchLowercase]{SF Pro Text Light}
\fi
\usecolortheme{wolverine}
\usefonttheme{serif} % use mainfont rather than sansfont for slide text
\useinnertheme{default}
\useoutertheme{miniframes}
% Use upquote if available, for straight quotes in verbatim environments
\IfFileExists{upquote.sty}{\usepackage{upquote}}{}
\IfFileExists{microtype.sty}{% use microtype if available
  \usepackage[]{microtype}
  \UseMicrotypeSet[protrusion]{basicmath} % disable protrusion for tt fonts
}{}
\makeatletter
\@ifundefined{KOMAClassName}{% if non-KOMA class
  \IfFileExists{parskip.sty}{%
    \usepackage{parskip}
  }{% else
    \setlength{\parindent}{0pt}
    \setlength{\parskip}{6pt plus 2pt minus 1pt}}
}{% if KOMA class
  \KOMAoptions{parskip=half}}
\makeatother
\usepackage{xcolor}
\newif\ifbibliography
\setlength{\emergencystretch}{3em} % prevent overfull lines
\setcounter{secnumdepth}{-\maxdimen} % remove section numbering


\providecommand{\tightlist}{%
  \setlength{\itemsep}{0pt}\setlength{\parskip}{0pt}}\usepackage{longtable,booktabs,array}
\usepackage{calc} % for calculating minipage widths
\usepackage{caption}
% Make caption package work with longtable
\makeatletter
\def\fnum@table{\tablename~\thetable}
\makeatother
\usepackage{graphicx}
\makeatletter
\def\maxwidth{\ifdim\Gin@nat@width>\linewidth\linewidth\else\Gin@nat@width\fi}
\def\maxheight{\ifdim\Gin@nat@height>\textheight\textheight\else\Gin@nat@height\fi}
\makeatother
% Scale images if necessary, so that they will not overflow the page
% margins by default, and it is still possible to overwrite the defaults
% using explicit options in \includegraphics[width, height, ...]{}
\setkeys{Gin}{width=\maxwidth,height=\maxheight,keepaspectratio}
% Set default figure placement to htbp
\makeatletter
\def\fps@figure{htbp}
\makeatother

\captionsetup[figure]{labelformat=empty}
\usepackage{pgfpages}
\setbeamertemplate{itemize item}[circle]
\setbeamertemplate{footline}[frame number]{}
\mode<handout>{\pgfpagesuselayout{6 on 1}[letterpaper, border shrink=8mm]}
\AtBeginSection{%
   \begin{frame}
       \tableofcontents[currentsection]
   \end{frame}
}
\makeatletter
\makeatother
\makeatletter
\makeatother
\makeatletter
\@ifpackageloaded{caption}{}{\usepackage{caption}}
\AtBeginDocument{%
\ifdefined\contentsname
  \renewcommand*\contentsname{Table of contents}
\else
  \newcommand\contentsname{Table of contents}
\fi
\ifdefined\listfigurename
  \renewcommand*\listfigurename{List of Figures}
\else
  \newcommand\listfigurename{List of Figures}
\fi
\ifdefined\listtablename
  \renewcommand*\listtablename{List of Tables}
\else
  \newcommand\listtablename{List of Tables}
\fi
\ifdefined\figurename
  \renewcommand*\figurename{Figure}
\else
  \newcommand\figurename{Figure}
\fi
\ifdefined\tablename
  \renewcommand*\tablename{Table}
\else
  \newcommand\tablename{Table}
\fi
}
\@ifpackageloaded{float}{}{\usepackage{float}}
\floatstyle{ruled}
\@ifundefined{c@chapter}{\newfloat{codelisting}{h}{lop}}{\newfloat{codelisting}{h}{lop}[chapter]}
\floatname{codelisting}{Listing}
\newcommand*\listoflistings{\listof{codelisting}{List of Listings}}
\makeatother
\makeatletter
\@ifpackageloaded{caption}{}{\usepackage{caption}}
\@ifpackageloaded{subcaption}{}{\usepackage{subcaption}}
\makeatother
\makeatletter
\@ifpackageloaded{tcolorbox}{}{\usepackage[many]{tcolorbox}}
\makeatother
\makeatletter
\@ifundefined{shadecolor}{\definecolor{shadecolor}{rgb}{.97, .97, .97}}
\makeatother
\makeatletter
\makeatother
\ifLuaTeX
  \usepackage{selnolig}  % disable illegal ligatures
\fi
\IfFileExists{bookmark.sty}{\usepackage{bookmark}}{\usepackage{hyperref}}
\IfFileExists{xurl.sty}{\usepackage{xurl}}{} % add URL line breaks if available
\urlstyle{same} % disable monospaced font for URLs
\hypersetup{
  pdftitle={Knowledge and Reality, Lecture 18},
  pdfauthor={Brian Weatherson},
  hidelinks,
  pdfcreator={LaTeX via pandoc}}

\title{Knowledge and Reality, Lecture 18}
\author{Brian Weatherson}
\date{11/2/22}

\begin{document}
\frame{\titlepage}
\ifdefined\Shaded\renewenvironment{Shaded}{\begin{tcolorbox}[boxrule=0pt, enhanced, borderline west={3pt}{0pt}{shadecolor}, breakable, sharp corners, interior hidden, frame hidden]}{\end{tcolorbox}}\fi

\begin{frame}{Reminder}
\protect\hypertarget{reminder}{}
\begin{itemize}[<+->]
\tightlist
\item
  Election day is 6 days away.
\item
  You can vote at UMMA, just outside this building
\end{itemize}
\end{frame}

\hypertarget{a-grand-theory-of-perception}{%
\section{A Grand Theory of
Perception}\label{a-grand-theory-of-perception}}

\begin{frame}{Four Steps}
\protect\hypertarget{four-steps}{}
\begin{enumerate}[<+->]
\tightlist
\item
  Physical world \(\rightarrow\) sensory irritations
\item
  Sensory irritations \(\rightarrow\) experiences
\item
  Experiences \(\rightarrow\) beliefs
\item
  Beliefs \(\rightarrow\) actions
\end{enumerate}
\end{frame}

\begin{frame}{World to Inputs}
\protect\hypertarget{world-to-inputs}{}
We mostly think this is a matter for the scientists to discover.

\begin{itemize}[<+->]
\tightlist
\item
  But note that there are two kinds of philosopher who disagree.
\end{itemize}
\end{frame}

\begin{frame}{World to Inputs}
\protect\hypertarget{world-to-inputs-1}{}
Sceptics think that there may not be an external world.

\begin{itemize}[<+->]
\tightlist
\item
  And if not, the external world can't cause sensory irritations.
\item
  But we're by and large setting them aside.
\item
  For this part of the course, assume there is an external world, and it
  is much like science says it is.
\end{itemize}
\end{frame}

\begin{frame}{World to Input}
\protect\hypertarget{world-to-input}{}
Idealists (in this sense of idealist) deny that the external world is
\emph{physical}.

\begin{itemize}[<+->]
\tightlist
\item
  They think that the world is in some sense fundamentally mental.
\item
  Perhaps the entire world is a construction out of (perhaps merely
  possible) sense-experiences.
\item
  Again, set that view to one side, though until recently it was
  \emph{very} popular in Western philosophy.
\end{itemize}
\end{frame}

\begin{frame}{Inputs to Experiences}
\protect\hypertarget{inputs-to-experiences}{}
We don't see firing rates in the optic nerve, or hear vibrations in the
inner ear.

\begin{itemize}[<+->]
\tightlist
\item
  It's only when these things are converted into experiences that we
  have perceptions.
\item
  There are a few reasons to think that we are sensitive to things that
  we don't consciously experience - e.g., blindsight, or very fast
  reactions.
\item
  But we'll mostly pay attention to things where the is a conscious
  experience.
\end{itemize}
\end{frame}

\begin{frame}{Inputs to Experiences}
\protect\hypertarget{inputs-to-experiences-1}{}
I'll sometimes call these things `appearances' or `looks', when talking
about visual experience.

\begin{itemize}[<+->]
\tightlist
\item
  These are all meant to be the same kind of thing.
\item
  As we get further into the book, we'll fuss more about some details
  here.
\end{itemize}
\end{frame}

\begin{frame}{Experiences to Beliefs}
\protect\hypertarget{experiences-to-beliefs}{}
We don't believe everything we see.

\begin{itemize}[<+->]
\tightlist
\item
  Sometimes we ignore it. Actually most of the time I guess that's true;
  I have almost no beliefs about the periphery of my vision most of the
  time.
\item
  Sometimes we overrule it, when we think something must be illusory.
\item
  But we're interested in perception because so often we do take things
  to be as they appear.
\end{itemize}
\end{frame}

\begin{frame}{Beliefs to Action}
\protect\hypertarget{beliefs-to-action}{}
And sometimes belief leads to action.

\begin{itemize}[<+->]
\tightlist
\item
  Not always in the way that would be optimal
\item
  Sometimes there is a deer in the headlights effect
\item
  Sometimes someone is committed to an action and does it even when they
  know by perception it would be wrong/pointles.
\item
  But often enough.
\end{itemize}
\end{frame}

\begin{frame}{Four Steps}
\protect\hypertarget{four-steps-1}{}
\begin{enumerate}[<+->]
\tightlist
\item
  Physical world \(\rightarrow\) sensory irritations
\item
  Sensory irritations \(\rightarrow\) experiences
\item
  Experiences \(\rightarrow\) beliefs
\item
  Beliefs \(\rightarrow\) actions
\end{enumerate}
\end{frame}

\begin{frame}{Where Does Philosophy Fit In?}
\protect\hypertarget{where-does-philosophy-fit-in}{}
A simple view:

\begin{itemize}[<+->]
\tightlist
\item
  No role for philosophy at steps 1 and 2; they are just scientific
  question.
\item
  Epistemology has a lot to say about step 3, about how beliefs are
  formed.
\item
  And practical philosophy (ethics, philosophy of action) has a lot to
  say about step 4.
\end{itemize}
\end{frame}

\begin{frame}{Where Does Philosophy Fit In?}
\protect\hypertarget{where-does-philosophy-fit-in-1}{}
Siegel's view:

\begin{itemize}[<+->]
\tightlist
\item
  Somewhat sympathetic to the simple view about step 1, though we'll
  come back to it.
\item
  But thinks that the theory of rationality should cover all of steps 2
  to 4.
\end{itemize}
\end{frame}

\hypertarget{differences-between-the-stages}{%
\section{Differences Between the
Stages}\label{differences-between-the-stages}}

\begin{frame}{Three Questions}
\protect\hypertarget{three-questions}{}
\begin{itemize}[<+->]
\tightlist
\item
  Which of the four stages is under voluntary control (or anything like
  it)?
\item
  Which of the four stages reflects skill on the part of the perceiver?
\item
  Which of the four involves (or should involve) other beliefs?
\end{itemize}
\end{frame}

\begin{frame}{Voluntary Control}
\protect\hypertarget{voluntary-control}{}
\begin{itemize}[<+->]
\tightlist
\item
  Stage 1 seems to. You can turn your head, close your eyes, shift your
  attention, or zone out.
\item
  Stage 2 is a little harder to see intuitively, though maybe focus can
  do it.
\end{itemize}
\end{frame}

\begin{frame}
\begin{figure}

{\centering \includegraphics[width=\textwidth,height=0.6\textheight]{../images/necker.png}

}

\caption{The necker cube 'illusion}

\end{figure}
\end{frame}

\begin{frame}{Voluntary}
\protect\hypertarget{voluntary}{}
\end{frame}

\hypertarget{experiences}{%
\section{Experiences}\label{experiences}}

\begin{frame}{Step 2 - Irritations \(\rightarrow\) Experiences}
\protect\hypertarget{step-2---irritations-rightarrow-experiences}{}
\end{frame}



\end{document}
