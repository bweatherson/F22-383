% Options for packages loaded elsewhere
\PassOptionsToPackage{unicode}{hyperref}
\PassOptionsToPackage{hyphens}{url}
%
\documentclass[
  17pt,
  letterpaper,
  ignorenonframetext,
  aspectratio=169,
]{beamer}
\usepackage{pgfpages}
\setbeamertemplate{caption}[numbered]
\setbeamertemplate{caption label separator}{: }
\setbeamercolor{caption name}{fg=normal text.fg}
\beamertemplatenavigationsymbolsempty
% Prevent slide breaks in the middle of a paragraph
\widowpenalties 1 10000
\raggedbottom
\setbeamertemplate{part page}{
  \centering
  \begin{beamercolorbox}[sep=16pt,center]{part title}
    \usebeamerfont{part title}\insertpart\par
  \end{beamercolorbox}
}
\setbeamertemplate{section page}{
  \centering
  \begin{beamercolorbox}[sep=12pt,center]{part title}
    \usebeamerfont{section title}\insertsection\par
  \end{beamercolorbox}
}
\setbeamertemplate{subsection page}{
  \centering
  \begin{beamercolorbox}[sep=8pt,center]{part title}
    \usebeamerfont{subsection title}\insertsubsection\par
  \end{beamercolorbox}
}
\AtBeginPart{
  \frame{\partpage}
}
\AtBeginSection{
  \ifbibliography
  \else
    \frame{\sectionpage}
  \fi
}
\AtBeginSubsection{
  \frame{\subsectionpage}
}

\usepackage{amsmath,amssymb}
\usepackage{lmodern}
\usepackage{iftex}
\ifPDFTeX
  \usepackage[T1]{fontenc}
  \usepackage[utf8]{inputenc}
  \usepackage{textcomp} % provide euro and other symbols
\else % if luatex or xetex
  \usepackage{unicode-math}
  \defaultfontfeatures{Scale=MatchLowercase}
  \defaultfontfeatures[\rmfamily]{Ligatures=TeX,Scale=1}
  \setmainfont[BoldFont = SF Pro Text Semibold, Scale =
MatchLowercase]{SF Pro Text Light}
\fi
\usecolortheme{wolverine}
\usefonttheme{serif} % use mainfont rather than sansfont for slide text
\useinnertheme{default}
\useoutertheme{miniframes}
% Use upquote if available, for straight quotes in verbatim environments
\IfFileExists{upquote.sty}{\usepackage{upquote}}{}
\IfFileExists{microtype.sty}{% use microtype if available
  \usepackage[]{microtype}
  \UseMicrotypeSet[protrusion]{basicmath} % disable protrusion for tt fonts
}{}
\makeatletter
\@ifundefined{KOMAClassName}{% if non-KOMA class
  \IfFileExists{parskip.sty}{%
    \usepackage{parskip}
  }{% else
    \setlength{\parindent}{0pt}
    \setlength{\parskip}{6pt plus 2pt minus 1pt}}
}{% if KOMA class
  \KOMAoptions{parskip=half}}
\makeatother
\usepackage{xcolor}
\newif\ifbibliography
\setlength{\emergencystretch}{3em} % prevent overfull lines
\setcounter{secnumdepth}{-\maxdimen} % remove section numbering


\providecommand{\tightlist}{%
  \setlength{\itemsep}{0pt}\setlength{\parskip}{0pt}}\usepackage{longtable,booktabs,array}
\usepackage{calc} % for calculating minipage widths
\usepackage{caption}
% Make caption package work with longtable
\makeatletter
\def\fnum@table{\tablename~\thetable}
\makeatother
\usepackage{graphicx}
\makeatletter
\def\maxwidth{\ifdim\Gin@nat@width>\linewidth\linewidth\else\Gin@nat@width\fi}
\def\maxheight{\ifdim\Gin@nat@height>\textheight\textheight\else\Gin@nat@height\fi}
\makeatother
% Scale images if necessary, so that they will not overflow the page
% margins by default, and it is still possible to overwrite the defaults
% using explicit options in \includegraphics[width, height, ...]{}
\setkeys{Gin}{width=\maxwidth,height=\maxheight,keepaspectratio}
% Set default figure placement to htbp
\makeatletter
\def\fps@figure{htbp}
\makeatother

\captionsetup[figure]{labelformat=empty}
\usepackage{pgfpages}
\setbeamertemplate{itemize item}[circle]
\setbeamertemplate{footline}[frame number]{}
\mode<handout>{\pgfpagesuselayout{6 on 1}[letterpaper, border shrink=8mm]}
\AtBeginSection{%
   \begin{frame}
       \tableofcontents[currentsection]
   \end{frame}
}
\makeatletter
\makeatother
\makeatletter
\makeatother
\makeatletter
\@ifpackageloaded{caption}{}{\usepackage{caption}}
\AtBeginDocument{%
\ifdefined\contentsname
  \renewcommand*\contentsname{Table of contents}
\else
  \newcommand\contentsname{Table of contents}
\fi
\ifdefined\listfigurename
  \renewcommand*\listfigurename{List of Figures}
\else
  \newcommand\listfigurename{List of Figures}
\fi
\ifdefined\listtablename
  \renewcommand*\listtablename{List of Tables}
\else
  \newcommand\listtablename{List of Tables}
\fi
\ifdefined\figurename
  \renewcommand*\figurename{Figure}
\else
  \newcommand\figurename{Figure}
\fi
\ifdefined\tablename
  \renewcommand*\tablename{Table}
\else
  \newcommand\tablename{Table}
\fi
}
\@ifpackageloaded{float}{}{\usepackage{float}}
\floatstyle{ruled}
\@ifundefined{c@chapter}{\newfloat{codelisting}{h}{lop}}{\newfloat{codelisting}{h}{lop}[chapter]}
\floatname{codelisting}{Listing}
\newcommand*\listoflistings{\listof{codelisting}{List of Listings}}
\makeatother
\makeatletter
\@ifpackageloaded{caption}{}{\usepackage{caption}}
\@ifpackageloaded{subcaption}{}{\usepackage{subcaption}}
\makeatother
\makeatletter
\@ifpackageloaded{tcolorbox}{}{\usepackage[many]{tcolorbox}}
\makeatother
\makeatletter
\@ifundefined{shadecolor}{\definecolor{shadecolor}{rgb}{.97, .97, .97}}
\makeatother
\makeatletter
\makeatother
\ifLuaTeX
  \usepackage{selnolig}  % disable illegal ligatures
\fi
\IfFileExists{bookmark.sty}{\usepackage{bookmark}}{\usepackage{hyperref}}
\IfFileExists{xurl.sty}{\usepackage{xurl}}{} % add URL line breaks if available
\urlstyle{same} % disable monospaced font for URLs
\hypersetup{
  pdftitle={Knowledge and Reality, Lecture 11},
  pdfauthor={Brian Weatherson},
  hidelinks,
  pdfcreator={LaTeX via pandoc}}

\title{Knowledge and Reality, Lecture 11}
\author{Brian Weatherson}
\date{2022-10-05}

\begin{document}
\frame{\titlepage}
\ifdefined\Shaded\renewenvironment{Shaded}{\begin{tcolorbox}[interior hidden, borderline west={3pt}{0pt}{shadecolor}, boxrule=0pt, frame hidden, enhanced, breakable, sharp corners]}{\end{tcolorbox}}\fi

\hypertarget{the-aim-of-epistemology}{%
\section{The Aim of Epistemology}\label{the-aim-of-epistemology}}

\begin{frame}{Three Projects}
\protect\hypertarget{three-projects}{}
\begin{enumerate}[<+->]
\tightlist
\item
  Ideal
\item
  Threshold
\item
  Improvements
\end{enumerate}
\end{frame}

\begin{frame}{Not Just Epistemology}
\protect\hypertarget{not-just-epistemology}{}
In ethics you can do all three.

\begin{enumerate}[<+->]
\tightlist
\item
  Ideal, e.g., the perfect person.
\item
  Threshold, e.g., not a moral failing.
\item
  Improvements, e.g., better ways to live.
\end{enumerate}
\end{frame}

\begin{frame}{An Important Threshold}
\protect\hypertarget{an-important-threshold}{}
In criminal law, the difference between \emph{is a crime} and \emph{is
not a crime} is very important.

\begin{itemize}[<+->]
\tightlist
\item
  This is a threshold concept.
\item
  Not being a criminal is a long way from ideal.
\item
  But it's a very important threshold.
\end{itemize}
\end{frame}

\begin{frame}{Pasnau's Big Picture}
\protect\hypertarget{pasnaus-big-picture}{}
\begin{itemize}[<+->]
\tightlist
\item
  We use to care about 1, the ideal.
\item
  We now (or at least in C20) care about 2, the threshold.
\item
  This has been a bad change, and we should go back to the old ways.
\end{itemize}
\end{frame}

\begin{frame}{Other Details}
\protect\hypertarget{other-details}{}
There are two other features of Pasnau's picture of C20 epistemology.

\begin{itemize}[<+->]
\item
  It is closely connected to language.
\item
  It is connected to standards people already internalise.
\item
  I think all of these are mistaken claims about C20 epistemology, but
  arguing for that would take us too far afield.
\end{itemize}
\end{frame}

\begin{frame}{Knowledge as a Threshold}
\protect\hypertarget{knowledge-as-a-threshold}{}
So here's one way to do epistemology.

\begin{itemize}[<+->]
\tightlist
\item
  Work out what's good enough for knowledge/rationality/justification.
\item
  Try to get above that `good enough' line as often as possible.
\item
  Pasnau thinks that's too modest a goal, we should aim higher.
\end{itemize}
\end{frame}

\begin{frame}{Missing Option 3}
\protect\hypertarget{missing-option-3}{}
But even if you think threshold epistemology (project 2) is bad, it
doesn't follow that you should do ideal epistemology (project 1).

\begin{itemize}[<+->]
\tightlist
\item
  There's the project of just trying to get better over time (project
  3).
\end{itemize}
\end{frame}

\begin{frame}{Ideals and Improvements}
\protect\hypertarget{ideals-and-improvements}{}
I think Pasnau would say that projects 1 and 3 are too connected to just
do project 3 (improvement) without project 1 (ideal). Two possible
connections.

\begin{enumerate}[<+->]
\tightlist
\item
  You can't improve unless you know what you're improving towards, the
  ideal.
\item
  Getting more like the ideal is a way of improving.
\end{enumerate}
\end{frame}

\begin{frame}{What Would Jesus Do?}
\protect\hypertarget{what-would-jesus-do}{}
There is (or at least was until recently) a kind of pop Christianity
that goes along with this picture.

\begin{itemize}[<+->]
\tightlist
\item
  Idea is to guide your action by asking \emph{What would Jesus do?}.
\item
  That is, you try to get better by trying to imitate the ideal. (Or at
  least what you take to be the ideal.)
\end{itemize}
\end{frame}

\begin{frame}{Objection}
\protect\hypertarget{objection}{}
This seems dubious as an approach to ethics, and I suspect to
epistemology as well.

\begin{itemize}[<+->]
\tightlist
\item
  The ideal person never apologises.
\item
  I'm trying to be more like the ideal.
\item
  So I'm not going to ever apologise.
\item
  Wait - that can't be right!!
\end{itemize}
\end{frame}

\begin{frame}{Ideals and Improvements}
\protect\hypertarget{ideals-and-improvements-1}{}
I think there's a deep problem here for people who think about using
ideals to generate improvements.

\begin{itemize}[<+->]
\tightlist
\item
  But I'm going to set that aside for the most part.
\end{itemize}
\end{frame}

\hypertarget{aristotle}{%
\section{Aristotle}\label{aristotle}}

\begin{frame}{Three Aristotelian Theses}
\protect\hypertarget{three-aristotelian-theses}{}
\begin{enumerate}[<+->]
\tightlist
\item
  Objects have essences.
\item
  These essences necessitate, cause, and explain, the object's actions.
\item
  Having episteme requires understanding why the thing is necessary.
\end{enumerate}
\end{frame}

\begin{frame}{Aristotle as a Sceptic}
\protect\hypertarget{aristotle-as-a-sceptic}{}
It is easy to read the third claim as implying that Aristotle is
basically a sceptic.

\begin{itemize}[<+->]
\tightlist
\item
  We can't have episteme of contingencies.
\item
  If episteme just is knowledge, then we can't have knowledge of
  contingencies.
\item
  And that's a pretty sweeping sceptical claim.
\end{itemize}
\end{frame}

\begin{frame}{Aristotle as an Ideal Epistemologist}
\protect\hypertarget{aristotle-as-an-ideal-epistemologist}{}
Pasnau wants to reject that interpretation.

\begin{itemize}[<+->]
\tightlist
\item
  I'm \emph{really} not going to get into interpretative disputes here.
\item
  Do note that a lot of the claims he makes here about interpretation
  are disputed.
\item
  Also note how often the argument is \emph{It would make more sense if
  {[}philosopher X{]} meant\ldots{}}.
\end{itemize}
\end{frame}

\begin{frame}{Aristotle as an Ideal Epistemologist}
\protect\hypertarget{aristotle-as-an-ideal-epistemologist-1}{}
Pasnau: The theory in \emph{Posterior Analytics} is a theory of what's
ideal.

\begin{itemize}[<+->]
\tightlist
\item
  It's not sceptical because knowledge doesn't require ideals.
\end{itemize}
\end{frame}

\begin{frame}{An Example}
\protect\hypertarget{an-example}{}
Imagine that I stole the cookies, but you don't know this.

\begin{itemize}[<+->]
\tightlist
\item
  You're trying to figure out who stole the cookies.
\item
  Fortunately, Sumeet saw me steal them.
\item
  Even better, he tells you that I stole them.
\end{itemize}
\end{frame}

\begin{frame}{An Example}
\protect\hypertarget{an-example-1}{}
At this point, you know I stole the cookies, but you do not have
episteme of it.

\begin{itemize}[<+->]
\tightlist
\item
  In fact, you couldn't have episteme of it, for two reasons.
\item
  Only generalisations can be episteme; not particular claims like
  \emph{Brian} stole the cookies.
\item
  Only necessitations can be episteme; and it's presumably contingent
  that I stole the cookies.
\end{itemize}
\end{frame}

\begin{frame}{Aristotle}
\protect\hypertarget{aristotle-1}{}
Why should we agree with Aristotle that this isn't episteme?

\begin{itemize}[<+->]
\tightlist
\item
  Because you could do better.
\item
  It would be better to know \emph{why} I stole the cookies.
\end{itemize}
\end{frame}

\begin{frame}{Aristotle and the Classical Indians}
\protect\hypertarget{aristotle-and-the-classical-indians}{}
It's interesting to compare Aristotle to classical Indian philosophers.

\begin{itemize}[<+->]
\tightlist
\item
  They also cared about ideals, about \emph{proofs}.
\item
  But most of them would say that you're in an ideal position.
\item
  You have a proof (via testimony) that I stole the cookies.
\end{itemize}
\end{frame}

\begin{frame}{Aristotle and the Classical Indians}
\protect\hypertarget{aristotle-and-the-classical-indians-1}{}
Question: What happens when you learn \emph{why} I stole the cookies.

\begin{itemize}[<+->]
\tightlist
\item
  For Aristotle, your position on \emph{Brian stole the cookies}
  improves; you now understand it (better).
\item
  For the Indian philosophers, you get a proof of a \textbf{different}
  question: \emph{Why did Brian steal the cookies?}.
\end{itemize}
\end{frame}

\begin{frame}{Aristotle}
\protect\hypertarget{aristotle-2}{}
Let's wrap up Pasnau's interpretation.

\begin{itemize}[<+->]
\tightlist
\item
  Episteme requires not being able to do better.
\item
  That means it requires optimal understanding.
\item
  That, given Aristotle's metaphysics, requires deriving the result from
  essences.
\end{itemize}
\end{frame}

\begin{frame}{Aristotle}
\protect\hypertarget{aristotle-3}{}
The distinctive features of episteme follow from this (plus Aristotle's
views on other parts of philosophy).

\begin{itemize}[<+->]
\tightlist
\item
  Only have episteme of necessities because essences are necessary.
\item
  Only have episteme of generalisations because derivations are via
  syllogism, and syllogisms involve generalities.
\item
  And so on, but that's enough Aristotle for now.
\end{itemize}
\end{frame}

\hypertarget{modern-science}{%
\section{Modern Science}\label{modern-science}}

\begin{frame}{Two Anti-Aristotelian Moves}
\protect\hypertarget{two-anti-aristotelian-moves}{}
\begin{enumerate}[<+->]
\tightlist
\item
  Reject that we should aim for ideals.
\item
  Reject his account of ideals.
\end{enumerate}

\begin{itemize}[<+->]
\tightlist
\item
  Sometimes it will be hard to tell whether someone is doing 1 or 2.
\end{itemize}
\end{frame}

\begin{frame}{Aristotle's Metaphysics}
\protect\hypertarget{aristotles-metaphysics}{}
Since his account of the ideal involves his distinctive metaphsyics, in
particular the account of essences, you might think that rejecting it
would involve rejecting the ideals.

\begin{itemize}[<+->]
\tightlist
\item
  But Hobbes and Locke rejected the metaphysics, accepted the ideals,
  and inferred a pessimistic conclusion.
\end{itemize}
\end{frame}

\begin{frame}{Newton}
\protect\hypertarget{newton}{}
As the last thing I'm going to have slides about, I want to see how we
think about these moves in the context of Newtonian physics.
\end{frame}

\begin{frame}{Newton}
\protect\hypertarget{newton-1}{}
Think about the theory of planetary orbits you get from Newton.

\begin{itemize}[<+->]
\tightlist
\item
  Given some very general principles (basically the principles of
  inertia and of gravitational attraction), you can derive the planetary
  orbits (more or less).
\item
  Kepler had earlier done pretty good derivations of them from his laws
  of planetary motion.
\end{itemize}
\end{frame}

\begin{frame}{Newton}
\protect\hypertarget{newton-2}{}
But Newton did a couple more things than Kepler.

\begin{itemize}[<+->]
\tightlist
\item
  He showed how to derive these Kepler's laws from even more general
  principles.
\item
  And he showed that the same principles could do other things, like
  explain the tides.
\end{itemize}
\end{frame}

\begin{frame}{Two (Related) Objections}
\protect\hypertarget{two-related-objections}{}
But there are still two reasons to be sceptical of Newton's
achievements.

\begin{enumerate}[<+->]
\tightlist
\item
  Action at a distance.
\item
  The laws themselves still aren't explained. Why do massive bodies
  attract?
\end{enumerate}
\end{frame}

\begin{frame}{A Few Responses}
\protect\hypertarget{a-few-responses}{}
\begin{enumerate}[<+->]
\tightlist
\item
  Sweeping generalisations are ideal, even if they aren't Aristotelian
  explanations. (Hempel, Kitcher)
\item
  This is ideal for humans, because it's as good as humans can do, and
  that's all that ideals should ever be (maybe Newton?).
\item
  This isn't ideal, and it's not really a step towards the ideal, and
  that's too bad because (as in 2), it's the best we can do (maybe
  Locke?)
\end{enumerate}
\end{frame}

\begin{frame}{For Next Time}
\protect\hypertarget{for-next-time}{}
We will look more at the notion of the ideal, and something that was
quite left out of today - how the epistemic ideal relates to certainty.

\begin{enumerate}[<+->]
\tightlist
\item
  To argue that the great epistemologists of
\end{enumerate}
\end{frame}



\end{document}
