% Options for packages loaded elsewhere
\PassOptionsToPackage{unicode}{hyperref}
\PassOptionsToPackage{hyphens}{url}
%
\documentclass[
  17pt,
  letterpaper,
  ignorenonframetext,
  aspectratio=169,
  handout]{beamer}
\usepackage{pgfpages}
\setbeamertemplate{caption}[numbered]
\setbeamertemplate{caption label separator}{: }
\setbeamercolor{caption name}{fg=normal text.fg}
\beamertemplatenavigationsymbolsempty
% Prevent slide breaks in the middle of a paragraph
\widowpenalties 1 10000
\raggedbottom
\setbeamertemplate{part page}{
  \centering
  \begin{beamercolorbox}[sep=16pt,center]{part title}
    \usebeamerfont{part title}\insertpart\par
  \end{beamercolorbox}
}
\setbeamertemplate{section page}{
  \centering
  \begin{beamercolorbox}[sep=12pt,center]{part title}
    \usebeamerfont{section title}\insertsection\par
  \end{beamercolorbox}
}
\setbeamertemplate{subsection page}{
  \centering
  \begin{beamercolorbox}[sep=8pt,center]{part title}
    \usebeamerfont{subsection title}\insertsubsection\par
  \end{beamercolorbox}
}
\AtBeginPart{
  \frame{\partpage}
}
\AtBeginSection{
  \ifbibliography
  \else
    \frame{\sectionpage}
  \fi
}
\AtBeginSubsection{
  \frame{\subsectionpage}
}

\usepackage{amsmath,amssymb}
\usepackage{lmodern}
\usepackage{iftex}
\ifPDFTeX
  \usepackage[T1]{fontenc}
  \usepackage[utf8]{inputenc}
  \usepackage{textcomp} % provide euro and other symbols
\else % if luatex or xetex
  \usepackage{unicode-math}
  \defaultfontfeatures{Scale=MatchLowercase}
  \defaultfontfeatures[\rmfamily]{Ligatures=TeX,Scale=1}
  \setmainfont[BoldFont = SF Pro Text Semibold, Scale =
MatchLowercase]{SF Pro Text Light}
\fi
\usecolortheme{wolverine}
\usefonttheme{serif} % use mainfont rather than sansfont for slide text
\useinnertheme{default}
\useoutertheme{miniframes}
% Use upquote if available, for straight quotes in verbatim environments
\IfFileExists{upquote.sty}{\usepackage{upquote}}{}
\IfFileExists{microtype.sty}{% use microtype if available
  \usepackage[]{microtype}
  \UseMicrotypeSet[protrusion]{basicmath} % disable protrusion for tt fonts
}{}
\makeatletter
\@ifundefined{KOMAClassName}{% if non-KOMA class
  \IfFileExists{parskip.sty}{%
    \usepackage{parskip}
  }{% else
    \setlength{\parindent}{0pt}
    \setlength{\parskip}{6pt plus 2pt minus 1pt}}
}{% if KOMA class
  \KOMAoptions{parskip=half}}
\makeatother
\usepackage{xcolor}
\newif\ifbibliography
\setlength{\emergencystretch}{3em} % prevent overfull lines
\setcounter{secnumdepth}{-\maxdimen} % remove section numbering


\providecommand{\tightlist}{%
  \setlength{\itemsep}{0pt}\setlength{\parskip}{0pt}}\usepackage{longtable,booktabs,array}
\usepackage{calc} % for calculating minipage widths
\usepackage{caption}
% Make caption package work with longtable
\makeatletter
\def\fnum@table{\tablename~\thetable}
\makeatother
\usepackage{graphicx}
\makeatletter
\def\maxwidth{\ifdim\Gin@nat@width>\linewidth\linewidth\else\Gin@nat@width\fi}
\def\maxheight{\ifdim\Gin@nat@height>\textheight\textheight\else\Gin@nat@height\fi}
\makeatother
% Scale images if necessary, so that they will not overflow the page
% margins by default, and it is still possible to overwrite the defaults
% using explicit options in \includegraphics[width, height, ...]{}
\setkeys{Gin}{width=\maxwidth,height=\maxheight,keepaspectratio}
% Set default figure placement to htbp
\makeatletter
\def\fps@figure{htbp}
\makeatother

\captionsetup[figure]{labelformat=empty}
\usepackage{pgfpages}
\setbeamertemplate{itemize item}[circle]
\setbeamertemplate{footline}[frame number]{}
\mode<handout>{\pgfpagesuselayout{6 on 1}[letterpaper, border shrink=8mm]}
\AtBeginSection{%
   \begin{frame}
       \tableofcontents[currentsection]
   \end{frame}
}
\makeatletter
\makeatother
\makeatletter
\makeatother
\makeatletter
\@ifpackageloaded{caption}{}{\usepackage{caption}}
\AtBeginDocument{%
\ifdefined\contentsname
  \renewcommand*\contentsname{Table of contents}
\else
  \newcommand\contentsname{Table of contents}
\fi
\ifdefined\listfigurename
  \renewcommand*\listfigurename{List of Figures}
\else
  \newcommand\listfigurename{List of Figures}
\fi
\ifdefined\listtablename
  \renewcommand*\listtablename{List of Tables}
\else
  \newcommand\listtablename{List of Tables}
\fi
\ifdefined\figurename
  \renewcommand*\figurename{Figure}
\else
  \newcommand\figurename{Figure}
\fi
\ifdefined\tablename
  \renewcommand*\tablename{Table}
\else
  \newcommand\tablename{Table}
\fi
}
\@ifpackageloaded{float}{}{\usepackage{float}}
\floatstyle{ruled}
\@ifundefined{c@chapter}{\newfloat{codelisting}{h}{lop}}{\newfloat{codelisting}{h}{lop}[chapter]}
\floatname{codelisting}{Listing}
\newcommand*\listoflistings{\listof{codelisting}{List of Listings}}
\makeatother
\makeatletter
\@ifpackageloaded{caption}{}{\usepackage{caption}}
\@ifpackageloaded{subcaption}{}{\usepackage{subcaption}}
\makeatother
\makeatletter
\@ifpackageloaded{tcolorbox}{}{\usepackage[many]{tcolorbox}}
\makeatother
\makeatletter
\@ifundefined{shadecolor}{\definecolor{shadecolor}{rgb}{.97, .97, .97}}
\makeatother
\makeatletter
\makeatother
\ifLuaTeX
  \usepackage{selnolig}  % disable illegal ligatures
\fi
\IfFileExists{bookmark.sty}{\usepackage{bookmark}}{\usepackage{hyperref}}
\IfFileExists{xurl.sty}{\usepackage{xurl}}{} % add URL line breaks if available
\urlstyle{same} % disable monospaced font for URLs
\hypersetup{
  pdftitle={Knowledge and Reality, Lecture 14},
  pdfauthor={Brian Weatherson},
  hidelinks,
  pdfcreator={LaTeX via pandoc}}

\title{Knowledge and Reality, Lecture 14}
\author{Brian Weatherson}
\date{2022-10-19}

\begin{document}
\frame{\titlepage}
\ifdefined\Shaded\renewenvironment{Shaded}{\begin{tcolorbox}[borderline west={3pt}{0pt}{shadecolor}, interior hidden, enhanced, boxrule=0pt, breakable, frame hidden, sharp corners]}{\end{tcolorbox}}\fi

\hypertarget{two-questions-about-perception}{%
\section{Two Questions about
Perception}\label{two-questions-about-perception}}

\begin{frame}{Pasnau's Two Questions}
\protect\hypertarget{pasnaus-two-questions}{}
\begin{enumerate}[<+->]
\tightlist
\item
  Simple or Dual?
\item
  Mediated or Diaphonus?
\end{enumerate}
\end{frame}

\begin{frame}{Simple or Dual}
\protect\hypertarget{simple-or-dual}{}
\begin{description}[<+->]
\tightlist
\item[Simple]
The perceptual act alone represents external things, without any need
for some further, distinct internal representation.
\item[Dual]
Perception should be analyzed into the act and a distinct internal
representation.
\end{description}
\end{frame}

\begin{frame}{Mediated or Diaphonus}
\protect\hypertarget{mediated-or-diaphonus}{}
\begin{description}[<+->]
\tightlist
\item[Mediated]
We perceive external objects---if we do at all---only in virtue of
perceiving something within ourselves.
\item[Diaphonus]
Whatever internal intermediaries there may be {[}in perception{]} are
not themselves the objects of perception.
\end{description}
\end{frame}

\begin{frame}{Two Contemporary Questions}
\protect\hypertarget{two-contemporary-questions}{}
\begin{enumerate}[<+->]
\tightlist
\item
  Does perception have content?
\item
  Do we perceive the world by perceiving internal states?
\end{enumerate}
\end{frame}

\begin{frame}{The Same Questions}
\protect\hypertarget{the-same-questions}{}
\begin{itemize}[<+->]
\tightlist
\item
  Not exactly; the weird `simple but mediated' view does not go the same
  way.
\item
  But close enough for current purposes.
\end{itemize}
\end{frame}

\hypertarget{object-and-content}{%
\section{Object and Content}\label{object-and-content}}

\begin{frame}{Content}
\protect\hypertarget{content}{}
We talked last time about perceptual content, with questions like.

\begin{itemize}[<+->]
\tightlist
\item
  Are individuals parts of the contents of perception, or just their
  properties?
\item
  Are the properties perceived things like colors and sounds, things
  like shape and size, or something else?
\end{itemize}
\end{frame}

\begin{frame}{Object}
\protect\hypertarget{object}{}
But that's a slightly different question from asking what it is we see
when we see.

\begin{itemize}[<+->]
\tightlist
\item
  Hopefully they are related, and Pasnau will eventually argue they are
  closely related, but they are separate.
\end{itemize}
\end{frame}

\begin{frame}{Some Examples}
\protect\hypertarget{some-examples}{}
I want to build up to the question by thinking about some cases.

\begin{itemize}[<+->]
\tightlist
\item
  Assume, for now, that when you look at a person, say Barack Obama,
  standing directly in front of you, in broad daylight, what you see is
  that person.
\end{itemize}
\end{frame}

\begin{frame}{Assumptions}
\protect\hypertarget{assumptions}{}
\begin{enumerate}[<+->]
\tightlist
\item
  I've said you see Obama, not just his qualities. If you prefer,
  substitute Obama's shape, size, appearance, etc for the object of
  perception in whatever I say going forward.
\item
  I've said you see Obama, not your mental representation of him (a la
  Descartes, Locke, etc). That's a substantive assumption, and we'll
  come back to it.
\end{enumerate}
\end{frame}

\begin{frame}{Case 1: Window}
\protect\hypertarget{case-1-window}{}
You look towards Obama, and get a clear sight of him through a (clear,
clean) window. What do you see?

\begin{enumerate}[<+->]
\tightlist
\item
  Obama?
\item
  The window?
\end{enumerate}
\end{frame}

\begin{frame}{Case 1: Window}
\protect\hypertarget{case-1-window-1}{}
I assume here the answer is Obama, not the window.

\begin{itemize}[<+->]
\tightlist
\item
  The window is causally relevant; vision works by detecting light
  bouncing off a distal object, and that light passes through (and is
  affected by) the window.
\item
  But that doesn't mean you see the window; you see Obama.
\end{itemize}
\end{frame}

\begin{frame}{Case 2: Binoculars}
\protect\hypertarget{case-2-binoculars}{}
You have a pair of binoculars, and you point them at a distant stage. On
that stage is Obama, and you can make him out clearly after adjusting
the focus of the binoculars. What do you see?

\begin{enumerate}[<+->]
\tightlist
\item
  Obama?
\item
  The image in the binoculars?
\end{enumerate}
\end{frame}

\begin{frame}{Case 2: Binoculars}
\protect\hypertarget{case-2-binoculars-1}{}
I'm still inclined to say it's Obama here, but it turns out to be a
useful case to consider.

\begin{itemize}[<+->]
\tightlist
\item
  If you say `binoculars', do people who wear glasses only ever see
  their glasses?
\end{itemize}
\end{frame}

\begin{frame}{Case 2: Binoculars}
\protect\hypertarget{case-2-binoculars-2}{}
I'm still inclined to say it's Obama here, but it turns out to be a
useful case to consider.

\begin{itemize}[<+->]
\tightlist
\item
  If you say Obama, what do you say is being seen while the focus is
  being adjusted? If it is the picture, when do you stop seeing it?
\item
  And we might want to revisit that after some other cases.
\end{itemize}
\end{frame}

\begin{frame}{Case 3: Live TV}
\protect\hypertarget{case-3-live-tv}{}
You're facing a TV that is showing a live speech by Obama, and
concentrating on him. What do you see?

\begin{enumerate}[<+->]
\tightlist
\item
  Obama?
\item
  The TV?
\end{enumerate}
\end{frame}

\begin{frame}{Case 3: Live TV}
\protect\hypertarget{case-3-live-tv-1}{}
I don't have anything useful here; it's a really tricky case.

\begin{itemize}[<+->]
\tightlist
\item
  Note that the test Pasnau gives at the end of chapter 4 says that
  Obama, not the TV, is the object of perception. I'm not sure it's so
  clear-cut.
\end{itemize}
\end{frame}

\begin{frame}{Case 4: Recorded TV}
\protect\hypertarget{case-4-recorded-tv}{}
You're facing a TV that is showing a replay of a debate Obama took part
in before the 2008 election. What do you see?

\begin{enumerate}[<+->]
\tightlist
\item
  Obama?
\item
  The TV?
\end{enumerate}
\end{frame}

\begin{frame}{Case 4: Recorded TV}
\protect\hypertarget{case-4-recorded-tv-1}{}
I guess the TV here, but who knows?
\end{frame}

\begin{frame}{Case 5: A Photograph}
\protect\hypertarget{case-5-a-photograph}{}
You're facing a photograph of Obama, taken during the 2008 election
campaign. What do you see?

\begin{enumerate}[<+->]
\tightlist
\item
  Obama?
\item
  The photo?
\end{enumerate}
\end{frame}

\begin{frame}{Case 5: A Photograph}
\protect\hypertarget{case-5-a-photograph-1}{}
Kendall Walton (a prominent philosopher of art now emiritus here at UM)
has argued that photographs involve directly seeing the thing
photographed.

\begin{itemize}[<+->]
\tightlist
\item
  This is a very unpopular view!
\item
  But maybe worth thinking about where it differs from the binoculars
  case.
\end{itemize}
\end{frame}

\begin{frame}{Case 6: An Artwork}
\protect\hypertarget{case-6-an-artwork}{}
\begin{figure}

{\centering \includegraphics[width=\textwidth,height=0.5\textheight]{../images/obama.jpg}

}

\caption{A prominent poster from the 2008 election.}

\end{figure}

Do you see Obama or the artwork?
\end{frame}

\begin{frame}{Case 6: An Artwork}
\protect\hypertarget{case-6-an-artwork-1}{}
Here, if not before, we have a case where perception is mediated.

\begin{itemize}[<+->]
\tightlist
\item
  If looking at this picture is a way of seeing Obama, it is only by
  seeing something else, namely the artwork.
\end{itemize}
\end{frame}

\begin{frame}{The Six Cases}
\protect\hypertarget{the-six-cases}{}
\begin{enumerate}[<+->]
\tightlist
\item
  Window
\item
  Binoculars
\item
  Live TV
\item
  Recorded TV
\item
  Photo
\item
  Artwork
\end{enumerate}
\end{frame}

\begin{frame}{The Six Cases}
\protect\hypertarget{the-six-cases-1}{}
\begin{itemize}[<+->]
\tightlist
\item
  All of these are cases of causal mediation; something is part of the
  causal chain connecting Obama to you.
\item
  And in each case, the causal intermediator could mess something up;
  windows could be tinted, the picture could be out of focus, etc.
\end{itemize}
\end{frame}

\begin{frame}{The Six Cases}
\protect\hypertarget{the-six-cases-2}{}
\begin{itemize}[<+->]
\tightlist
\item
  But the standard (pre-C17, post-C18) view is that somewhere on this
  list, a break occurs.
\item
  Somewhere we go from seeing an external object, to seeing a mediator.
\item
  But where?
\end{itemize}
\end{frame}

\begin{frame}{Three Reactions}
\protect\hypertarget{three-reactions}{}
\begin{enumerate}[<+->]
\tightlist
\item
  Find a line!
\item
  Say that the idea of objects of perception is a mistake; perception is
  a purely causal process.
\item
  Say that the idea of unmediated perception is a mistake; perception
  always goes via internal mediators that are themselves the objects of
  perception.
\end{enumerate}
\end{frame}

\begin{frame}{Option 2}
\protect\hypertarget{option-2}{}
We're going to mostly talk about option 3, that's the focus of the
chapter, but a brief word on option 2.

\begin{itemize}[<+->]
\tightlist
\item
  The external world can effect thoughts without going via
  representations.
\item
  If I don't eat, I get hungry, and I think/write more negative lecture
  notes.
\end{itemize}
\end{frame}

\begin{frame}{Option 2}
\protect\hypertarget{option-2-1}{}
\begin{itemize}[<+->]
\tightlist
\item
  That's not because there is any intermediating representation between
  my lack of food and my negative views; it's just a causal force.
\item
  Maybe perception no more involves representation than those kind of
  external effects on mood.
\item
  That's certainly not how it seems though.
\end{itemize}
\end{frame}

\hypertarget{why-mediators}{%
\section{Why Mediators}\label{why-mediators}}

\begin{frame}{Four Possible Reasons}
\protect\hypertarget{four-possible-reasons}{}
\begin{enumerate}[<+->]
\tightlist
\item
  Introspection.
\item
  Illusion.
\item
  Fidelity.
\item
  Slippery Slope.
\end{enumerate}

\begin{itemize}[<+->]
\tightlist
\item
  Pasnau rejects the first two as causally explatory, and doesn't
  consider the fourth (it's not historically relevant I think), to argue
  for the third.
\end{itemize}
\end{frame}

\begin{frame}{Pasnau's Positive Story}
\protect\hypertarget{pasnaus-positive-story}{}
Combines (true?) philosophical claim with (false) empirical claim.

\begin{enumerate}[<+->]
\tightlist
\item
  Fidelity Constraint: The item that is tracked with greater fidelity is
  that which is most properly the object of perception.
\item
  Empirical Claim: Perceptions are a higher fidelity representation of
  our minds than of the external world.
\end{enumerate}
\end{frame}

\begin{frame}{Pasnau's Positive Story}
\protect\hypertarget{pasnaus-positive-story-1}{}
He doesn't believe 2; it's not like our minds are splashed with color,
or are loud, or smelly.

\begin{itemize}[<+->]
\tightlist
\item
  But he claims that C17 folks did believe (something like) 1 and 2.
\item
  I couldn't tell how enthusiastic he is about 1, as opposed to saying
  it's part of what the long ago folks believed.
\end{itemize}
\end{frame}

\begin{frame}{Introspection}
\protect\hypertarget{introspection}{}
I found this a bit odd, though probably right on the history.

\begin{itemize}[<+->]
\tightlist
\item
  Question: Why motivate internal objects of perception by
  introspection?
\end{itemize}
\end{frame}

\begin{frame}{Introspection}
\protect\hypertarget{introspection-1}{}
Two part answer

\begin{enumerate}[<+->]
\tightlist
\item
  Introspection is a reliable means of understanding the mind.
\item
  It seems like we see internal states.
\end{enumerate}

\begin{itemize}[<+->]
\tightlist
\item
  On reflection, both parts seem wrong.
\end{itemize}
\end{frame}

\begin{frame}{Why Not Introspection}
\protect\hypertarget{why-not-introspection}{}
\begin{itemize}[<+->]
\tightlist
\item
  Pasnau focuses on 1, the general unreliability of introspection. And
  fair enough too; it's really unreliable.
\item
  But 2 is just as bad; it really doesn't seem like we see internal
  states.
\end{itemize}
\end{frame}

\begin{frame}{An Argument from Illusion}
\protect\hypertarget{an-argument-from-illusion}{}
\begin{enumerate}[<+->]
\tightlist
\item
  Whenever we see, we see something.
\item
  Whatever we see, we see accurately.
\item
  We don't see external objects accurately (in all cases).
\item
  We see the same kinds of things in accurate and inaccurate perception.
\end{enumerate}

\begin{enumerate}[<+->]
[A.]
\setcounter{enumi}{2}
\tightlist
\item
  Seeing involves seeing internal things.
\end{enumerate}
\end{frame}

\begin{frame}{An Argument from Illusion}
\protect\hypertarget{an-argument-from-illusion-1}{}
I guess step 3 is very plausible, but everything else seems dubious.
\end{frame}

\begin{frame}{An Argument from Hallucination}
\protect\hypertarget{an-argument-from-hallucination}{}
\begin{enumerate}[<+->]
\tightlist
\item
  Whenever we see, we see something.
\item
  Whatever we see, exists.
\item
  In hallucinations, the thing we see does not exist externally.
\item
  We see the same kinds of things in perception and hallucination.
\end{enumerate}

\begin{enumerate}[<+->]
[A.]
\setcounter{enumi}{2}
\tightlist
\item
  Seeing involves seeing internal things.
\end{enumerate}
\end{frame}

\begin{frame}{An Argument from Hallucination}
\protect\hypertarget{an-argument-from-hallucination-1}{}
Here it's trickier to say what's wrong, and the disjunctivist position -
reject 4 - has some promise.
\end{frame}

\begin{frame}{For Next Time}
\protect\hypertarget{for-next-time}{}
We'll go on to the importance of keeping a whole argument in mind at a
moment.
\end{frame}



\end{document}
