% Options for packages loaded elsewhere
\PassOptionsToPackage{unicode}{hyperref}
\PassOptionsToPackage{hyphens}{url}
%
\documentclass[
  17pt,
  letterpaper,
  ignorenonframetext,
  aspectratio=169,
  handout,
  xcolor={dvipsnames}]{beamer}
\usepackage{pgfpages}
\setbeamertemplate{caption}[numbered]
\setbeamertemplate{caption label separator}{: }
\setbeamercolor{caption name}{fg=normal text.fg}
\beamertemplatenavigationsymbolsempty
% Prevent slide breaks in the middle of a paragraph
\widowpenalties 1 10000
\raggedbottom
\setbeamertemplate{part page}{
  \centering
  \begin{beamercolorbox}[sep=16pt,center]{part title}
    \usebeamerfont{part title}\insertpart\par
  \end{beamercolorbox}
}
\setbeamertemplate{section page}{
  \centering
  \begin{beamercolorbox}[sep=12pt,center]{part title}
    \usebeamerfont{section title}\insertsection\par
  \end{beamercolorbox}
}
\setbeamertemplate{subsection page}{
  \centering
  \begin{beamercolorbox}[sep=8pt,center]{part title}
    \usebeamerfont{subsection title}\insertsubsection\par
  \end{beamercolorbox}
}
\AtBeginPart{
  \frame{\partpage}
}
\AtBeginSection{
  \ifbibliography
  \else
    \frame{\sectionpage}
  \fi
}
\AtBeginSubsection{
  \frame{\subsectionpage}
}

\usepackage{amsmath,amssymb}
\usepackage{lmodern}
\usepackage{iftex}
\ifPDFTeX
  \usepackage[T1]{fontenc}
  \usepackage[utf8]{inputenc}
  \usepackage{textcomp} % provide euro and other symbols
\else % if luatex or xetex
  \usepackage{unicode-math}
  \defaultfontfeatures{Scale=MatchLowercase}
  \defaultfontfeatures[\rmfamily]{Ligatures=TeX,Scale=1}
  \setmainfont[BoldFont = SF Pro Text Semibold, Scale =
MatchLowercase]{SF Pro Text Light}
\fi
\usecolortheme{wolverine}
\usefonttheme{serif} % use mainfont rather than sansfont for slide text
\useinnertheme{default}
\useoutertheme{miniframes}
% Use upquote if available, for straight quotes in verbatim environments
\IfFileExists{upquote.sty}{\usepackage{upquote}}{}
\IfFileExists{microtype.sty}{% use microtype if available
  \usepackage[]{microtype}
  \UseMicrotypeSet[protrusion]{basicmath} % disable protrusion for tt fonts
}{}
\makeatletter
\@ifundefined{KOMAClassName}{% if non-KOMA class
  \IfFileExists{parskip.sty}{%
    \usepackage{parskip}
  }{% else
    \setlength{\parindent}{0pt}
    \setlength{\parskip}{6pt plus 2pt minus 1pt}}
}{% if KOMA class
  \KOMAoptions{parskip=half}}
\makeatother
\usepackage{xcolor}
\newif\ifbibliography
\setlength{\emergencystretch}{3em} % prevent overfull lines
\setcounter{secnumdepth}{-\maxdimen} % remove section numbering


\providecommand{\tightlist}{%
  \setlength{\itemsep}{0pt}\setlength{\parskip}{0pt}}\usepackage{longtable,booktabs,array}
\usepackage{calc} % for calculating minipage widths
\usepackage{caption}
% Make caption package work with longtable
\makeatletter
\def\fnum@table{\tablename~\thetable}
\makeatother
\usepackage{graphicx}
\makeatletter
\def\maxwidth{\ifdim\Gin@nat@width>\linewidth\linewidth\else\Gin@nat@width\fi}
\def\maxheight{\ifdim\Gin@nat@height>\textheight\textheight\else\Gin@nat@height\fi}
\makeatother
% Scale images if necessary, so that they will not overflow the page
% margins by default, and it is still possible to overwrite the defaults
% using explicit options in \includegraphics[width, height, ...]{}
\setkeys{Gin}{width=\maxwidth,height=\maxheight,keepaspectratio}
% Set default figure placement to htbp
\makeatletter
\def\fps@figure{htbp}
\makeatother

\captionsetup[figure]{labelformat=empty}
\usepackage{pgfpages}
\setbeamertemplate{itemize item}[circle]
\setbeamertemplate{footline}[frame number]{}
\mode<handout>{\pgfpagesuselayout{6 on 1}[letterpaper, border shrink=8mm]}
\AtBeginSection{%
   \begin{frame}
       \tableofcontents[currentsection]
   \end{frame}
}
\makeatletter
\makeatother
\makeatletter
\makeatother
\makeatletter
\@ifpackageloaded{caption}{}{\usepackage{caption}}
\AtBeginDocument{%
\ifdefined\contentsname
  \renewcommand*\contentsname{Table of contents}
\else
  \newcommand\contentsname{Table of contents}
\fi
\ifdefined\listfigurename
  \renewcommand*\listfigurename{List of Figures}
\else
  \newcommand\listfigurename{List of Figures}
\fi
\ifdefined\listtablename
  \renewcommand*\listtablename{List of Tables}
\else
  \newcommand\listtablename{List of Tables}
\fi
\ifdefined\figurename
  \renewcommand*\figurename{Figure}
\else
  \newcommand\figurename{Figure}
\fi
\ifdefined\tablename
  \renewcommand*\tablename{Table}
\else
  \newcommand\tablename{Table}
\fi
}
\@ifpackageloaded{float}{}{\usepackage{float}}
\floatstyle{ruled}
\@ifundefined{c@chapter}{\newfloat{codelisting}{h}{lop}}{\newfloat{codelisting}{h}{lop}[chapter]}
\floatname{codelisting}{Listing}
\newcommand*\listoflistings{\listof{codelisting}{List of Listings}}
\makeatother
\makeatletter
\@ifpackageloaded{caption}{}{\usepackage{caption}}
\@ifpackageloaded{subcaption}{}{\usepackage{subcaption}}
\makeatother
\makeatletter
\@ifpackageloaded{tcolorbox}{}{\usepackage[many]{tcolorbox}}
\makeatother
\makeatletter
\@ifundefined{shadecolor}{\definecolor{shadecolor}{rgb}{.97, .97, .97}}
\makeatother
\makeatletter
\makeatother
\ifLuaTeX
  \usepackage{selnolig}  % disable illegal ligatures
\fi
\IfFileExists{bookmark.sty}{\usepackage{bookmark}}{\usepackage{hyperref}}
\IfFileExists{xurl.sty}{\usepackage{xurl}}{} % add URL line breaks if available
\urlstyle{same} % disable monospaced font for URLs
\hypersetup{
  pdftitle={Knowledge and Reality, Lecture 24},
  pdfauthor={Brian Weatherson},
  hidelinks,
  pdfcreator={LaTeX via pandoc}}

\title{Knowledge and Reality, Lecture 24}
\author{Brian Weatherson}
\date{11/28/22}

\begin{document}
\frame{\titlepage}
\ifdefined\Shaded\renewenvironment{Shaded}{\begin{tcolorbox}[boxrule=0pt, frame hidden, borderline west={3pt}{0pt}{shadecolor}, enhanced, breakable, interior hidden, sharp corners]}{\end{tcolorbox}}\fi

\hypertarget{overview}{%
\section{Overview}\label{overview}}

\begin{frame}{Summary}
\protect\hypertarget{summary}{}
\begin{itemize}[<+->]
\tightlist
\item
  The book doesn't have a traditional summary or conclusion.
\item
  But I think it's a good idea for us to look over what it covered.
\end{itemize}
\end{frame}

\begin{frame}{Rationality of Perception}
\protect\hypertarget{rationality-of-perception}{}
\begin{itemize}[<+->]
\tightlist
\item
  The traditional philosophical view is that perceptual states are
  arational states that ground other rational states.
\item
  Siegel thinks that the perceptual states themselves can be rational
  or, crucially, irrational.
\end{itemize}
\end{frame}

\begin{frame}{Rationality of Perception}
\protect\hypertarget{rationality-of-perception-1}{}
Two key theses:

\begin{enumerate}[<+->]
\tightlist
\item
  Perception is \textbf{rich}
\item
  Perception is \textbf{inferential}
\end{enumerate}
\end{frame}

\begin{frame}{Richness of Perception}
\protect\hypertarget{richness-of-perception}{}
Siegel thinks we can perceive things like \emph{That person is
dangerous}.

\begin{itemize}[<+->]
\tightlist
\item
  This contrasts with views on which perception doesn't have contents;
  it is just a presentation of the world (like a window)
\item
  It also contrasts with the (more common) view that we just perceive
  shapes, colors, sounds and the like.
\end{itemize}
\end{frame}

\begin{frame}{Inferential Perception}
\protect\hypertarget{inferential-perception}{}
Siegel this that:

\begin{enumerate}[<+->]
\tightlist
\item
  Perceptual states are downstream of other attitudes, including
  beliefs.
\item
  These transitions from the other states to perceptions are
  \textbf{inferences}.
\end{enumerate}
\end{frame}

\begin{frame}{Inferential Perception}
\protect\hypertarget{inferential-perception-1}{}
The first thesis contrasts with views on which perception is
\textbf{encapsulated}, i.e., insensitive to other information the world
contains.

\begin{itemize}[<+->]
\tightlist
\item
  The existence of illusions is some evidence for encapsulation.
\item
  But the existence of skilled perceivings is evidence against, and
  Siegel takes this to be stronger.
\end{itemize}
\end{frame}

\begin{frame}{Inferential Perception}
\protect\hypertarget{inferential-perception-2}{}
The second thesis contrasts with views on which the way perception
relates to background views involves something like ``mental jogging''.
\end{frame}

\begin{frame}{Inferential Perception}
\protect\hypertarget{inferential-perception-3}{}
\begin{itemize}[<+->]
\tightlist
\item
  People often believe that salt and pepper are often found together.
\item
  They also frequently transition from thoughts about \emph{salt} to
  thoughts about \emph{pepper}.
\item
  But the latter is not really an inference, it is just jogging.
\end{itemize}
\end{frame}

\hypertarget{failed-inferences}{%
\section{Failed Inferences}\label{failed-inferences}}

\begin{frame}{Five Possibilities}
\protect\hypertarget{five-possibilities}{}
\begin{enumerate}[<+->]
\tightlist
\item
  Poor support for premise
\item
  Poor transitions
\item
  Circularity
\item
  False premisess
\item
  Poor maintenance
\end{enumerate}
\end{frame}

\begin{frame}{Poor Support}
\protect\hypertarget{poor-support}{}
\begin{enumerate}[<+->]
\setcounter{enumi}{-1}
\tightlist
\item
  The largest city in a state is always the capital.
\item
  Detroit is the capital of Michigan (from 0)
\item
  So, this university is about an hour's drive from the Michigan
  capital.
\end{enumerate}

\begin{itemize}[<+->]
\tightlist
\item
  In normal circumstances, 0 is unsupported, so if it's the reason for
  1, the inferential belief 2 is not well-founded.
\end{itemize}
\end{frame}

\begin{frame}{Poor Transition}
\protect\hypertarget{poor-transition}{}
\begin{enumerate}[<+->]
\tightlist
\item
  Most Fs are Gs.
\item
  So, most Gs are Fs.
\end{enumerate}

\begin{itemize}[<+->]
\tightlist
\item
  This one might be relevant to some of the racial stereotype
  inferences.
\end{itemize}
\end{frame}

\begin{frame}{Circularity}
\protect\hypertarget{circularity}{}
\begin{enumerate}[<+->]
\tightlist
\item
  The scale says that the cup weighs 100g.
\item
  The cup weighs 100g.
\item
  So, the scale accurately measured the cup.
\end{enumerate}

\begin{itemize}[<+->]
\tightlist
\item
  If 1 is the reason for 2, this isn't a good inference. And this is
  funny because 1 is normally a good reason for 2, and 1+2 entails 3.
\end{itemize}
\end{frame}

\begin{frame}{False Premises}
\protect\hypertarget{false-premises}{}
\begin{enumerate}[<+->]
\tightlist
\item
  Mother says the water from this faucet is safe to drink.
\item
  So the water from this faucet is safe to drink.
\item
  So when I want water, I'll get it from this faucet.
\end{enumerate}

\begin{itemize}[<+->]
\tightlist
\item
  Siegel agrees with what's probably the most common view that the even
  if 2 is false, the transition from 1 to 2 to 3 could be a rational
  inference.
\end{itemize}
\end{frame}

\begin{frame}{Maintenance}
\protect\hypertarget{maintenance}{}
This is harder to get a clear example for, but note the following is a
way for a belief to be irrational.

\begin{itemize}[<+->]
\tightlist
\item
  It is originally formed by a reasonable process.
\item
  But once the belief is formed, it is not well maintained.
\end{itemize}
\end{frame}

\begin{frame}{Poor Maintenance}
\protect\hypertarget{poor-maintenance}{}
Some possible things that could go wrong.

\begin{itemize}[<+->]
\tightlist
\item
  Blocking off from sources of counter-evidence.
\item
  Ignoring counter-evidence.
\item
  Using the belief itself as a reason to reject counterevidence.
\item
  Using the belief itself as a reason to get more evidence.
\end{itemize}
\end{frame}

\hypertarget{racial-misperceptions}{%
\section{Racial Misperceptions}\label{racial-misperceptions}}

\begin{frame}{Big Picture}
\protect\hypertarget{big-picture}{}
\begin{itemize}[<+->]
\tightlist
\item
  There are a lot of racial misperceptions.
\item
  These are downstream of widely shared racial (well, racist) beliefs
  that are widely shared in the community.
\item
  At least in many (most?) cases, there is enough individual culpability
  in acquiring or maintaining these background beliefs that the
  perceptions are not well-founded.
\end{itemize}
\end{frame}

\begin{frame}{Racial Misperceptions}
\protect\hypertarget{racial-misperceptions-1}{}
Note how many, and how varied, the misperceptions at the start of the
chapter are.

\begin{itemize}[<+->]
\tightlist
\item
  They include seemingly innocuous things like age.
\end{itemize}
\end{frame}

\begin{frame}{Why So Many}
\protect\hypertarget{why-so-many}{}
Siegel is responding to a (possible) criticism that some of these
misperceptions might be grounded in accurate beliefs about racial
disparities.
\end{frame}

\begin{frame}{Why So Many}
\protect\hypertarget{why-so-many-1}{}
\begin{itemize}[<+->]
\tightlist
\item
  Some of the explaining away here seems to rely on bad inferences,
  e.g., from Most Fs are Gs to Most Gs are Fs.
\item
  But in some of the experiments, it's hard to see what beliefs about
  the world could make the beliefs about ages justified.
\end{itemize}
\end{frame}

\begin{frame}{Minimal Connections}
\protect\hypertarget{minimal-connections}{}
The range of experiments also helps respond to another kind of concern.

\begin{itemize}[<+->]
\tightlist
\item
  Imagine that a person has developed a kind of association between
  `black' and `dangerous' like the association between `salt' and
  `pepper'.
\item
  That looks pretty dubious morally, arguably worse than on Siegel's
  positive view, but it's not obviously within the range of epistemic
  evaluation.
\end{itemize}
\end{frame}

\begin{frame}{Minimal Connections}
\protect\hypertarget{minimal-connections-1}{}
Siegel's theory is that perceptions are irrational because they are bad
inferences.

\begin{itemize}[<+->]
\tightlist
\item
  Whatever inferences are, they are richer than the connection between
  `salt' and `pepper'.
\item
  So she needs to rule out the possibility that there is the same kind
  of connection here.
\end{itemize}
\end{frame}

\begin{frame}{Minimal Connections}
\protect\hypertarget{minimal-connections-2}{}
I was a little unsure why the association picture would fail to explain
most of these experimental results.

\begin{itemize}[<+->]
\tightlist
\item
  But it's really hard to see how it helps with the age test.
\item
  And more generally, having a broader range of data helps to make it
  harder for an opponent.
\end{itemize}
\end{frame}

\begin{frame}{Testimony}
\protect\hypertarget{testimony}{}
But the opponent Siegel spends the most time on concedes that the
background beliefs are false.

\begin{itemize}[<+->]
\tightlist
\item
  They argue that the inferences are nonetheless rational (or
  well-founded) because the false beliefs were formed in a reasonable
  way.
\item
  And that reasonable way is testimony from trusted sources.
\end{itemize}
\end{frame}

\begin{frame}{Testimony}
\protect\hypertarget{testimony-1}{}
There is one defence Siegel could offer here, but does not. In fact, she
expressly rejects it.

\begin{itemize}[<+->]
\tightlist
\item
  It is what we might call the transfer model of testimony.
\item
  Testimony only ever transfers the
  rationality/reasonableness/well-foundedness of a belief from
  speaker(s) to hearer(s).
\item
  So if the initial beliefs are ill-founded, as they are here, so will
  the subsequent beliefs be.
\end{itemize}
\end{frame}

\begin{frame}{Testimony}
\protect\hypertarget{testimony-2}{}
Siegel rejects the transfer model because of the example of the
well-meaning mother.

\begin{itemize}[<+->]
\tightlist
\item
  A child is entitled to trust their mother's safety advice, even if it
  turns out to be false and ill-founded.
\end{itemize}
\end{frame}

\begin{frame}{Contrasts}
\protect\hypertarget{contrasts}{}
What are the contrasts between this case and casually absorbing racist
beliefs?

\begin{enumerate}[<+->]
\tightlist
\item
  The mother is well-meaning and has the child's best interests at
  heart.
\item
  The racist beliefs require poor maintenance to be sustained.
\end{enumerate}
\end{frame}

\begin{frame}{Contrasts}
\protect\hypertarget{contrasts-1}{}
The first of these is not particularly compelling.

\begin{itemize}[<+->]
\tightlist
\item
  For one thing, what really matters is whether the person seems
  well-meaning, not whether they are.
\item
  For another, it isn't clear that racists spreading racist beliefs to
  people like them are not well-meaning, in the sense of trying to
  improve (by their lights) the well-being of their audience.
\end{itemize}
\end{frame}

\begin{frame}{Contrasts}
\protect\hypertarget{contrasts-2}{}
What about the second of these?

\begin{itemize}[<+->]
\tightlist
\item
  It seems fairly contingent at best that the belief requires poor
  maintenance.
\item
  Someone who grows up in a very homogenous (and racist) community won't
  have much opportunity to do any useful maintenance.
\end{itemize}
\end{frame}

\begin{frame}{Vigilance}
\protect\hypertarget{vigilance}{}
A better model might be that the person who simply absorbs racist
beliefs is (in most realistic situations) not going to be particularly
vigilant.

\begin{itemize}[<+->]
\tightlist
\item
  And that might be a difference with the case of the misleading mother.
\end{itemize}
\end{frame}

\begin{frame}{For Next Time}
\protect\hypertarget{for-next-time}{}
We'll look at some actual critics of Siegel's book.
\end{frame}



\end{document}
