% Options for packages loaded elsewhere
\PassOptionsToPackage{unicode}{hyperref}
\PassOptionsToPackage{hyphens}{url}
%
\documentclass[
  17pt,
  letterpaper,
  ignorenonframetext,
  aspectratio=169,
  handout]{beamer}
\usepackage{pgfpages}
\setbeamertemplate{caption}[numbered]
\setbeamertemplate{caption label separator}{: }
\setbeamercolor{caption name}{fg=normal text.fg}
\beamertemplatenavigationsymbolsempty
% Prevent slide breaks in the middle of a paragraph
\widowpenalties 1 10000
\raggedbottom
\setbeamertemplate{part page}{
  \centering
  \begin{beamercolorbox}[sep=16pt,center]{part title}
    \usebeamerfont{part title}\insertpart\par
  \end{beamercolorbox}
}
\setbeamertemplate{section page}{
  \centering
  \begin{beamercolorbox}[sep=12pt,center]{part title}
    \usebeamerfont{section title}\insertsection\par
  \end{beamercolorbox}
}
\setbeamertemplate{subsection page}{
  \centering
  \begin{beamercolorbox}[sep=8pt,center]{part title}
    \usebeamerfont{subsection title}\insertsubsection\par
  \end{beamercolorbox}
}
\AtBeginPart{
  \frame{\partpage}
}
\AtBeginSection{
  \ifbibliography
  \else
    \frame{\sectionpage}
  \fi
}
\AtBeginSubsection{
  \frame{\subsectionpage}
}

\usepackage{amsmath,amssymb}
\usepackage{lmodern}
\usepackage{iftex}
\ifPDFTeX
  \usepackage[T1]{fontenc}
  \usepackage[utf8]{inputenc}
  \usepackage{textcomp} % provide euro and other symbols
\else % if luatex or xetex
  \usepackage{unicode-math}
  \defaultfontfeatures{Scale=MatchLowercase}
  \defaultfontfeatures[\rmfamily]{Ligatures=TeX,Scale=1}
  \setmainfont[BoldFont = SF Pro Text Semibold, Scale =
MatchLowercase]{SF Pro Text Light}
\fi
\usecolortheme{wolverine}
\usefonttheme{serif} % use mainfont rather than sansfont for slide text
\useinnertheme{default}
\useoutertheme{miniframes}
% Use upquote if available, for straight quotes in verbatim environments
\IfFileExists{upquote.sty}{\usepackage{upquote}}{}
\IfFileExists{microtype.sty}{% use microtype if available
  \usepackage[]{microtype}
  \UseMicrotypeSet[protrusion]{basicmath} % disable protrusion for tt fonts
}{}
\makeatletter
\@ifundefined{KOMAClassName}{% if non-KOMA class
  \IfFileExists{parskip.sty}{%
    \usepackage{parskip}
  }{% else
    \setlength{\parindent}{0pt}
    \setlength{\parskip}{6pt plus 2pt minus 1pt}}
}{% if KOMA class
  \KOMAoptions{parskip=half}}
\makeatother
\usepackage{xcolor}
\newif\ifbibliography
\setlength{\emergencystretch}{3em} % prevent overfull lines
\setcounter{secnumdepth}{-\maxdimen} % remove section numbering


\providecommand{\tightlist}{%
  \setlength{\itemsep}{0pt}\setlength{\parskip}{0pt}}\usepackage{longtable,booktabs,array}
\usepackage{calc} % for calculating minipage widths
\usepackage{caption}
% Make caption package work with longtable
\makeatletter
\def\fnum@table{\tablename~\thetable}
\makeatother
\usepackage{graphicx}
\makeatletter
\def\maxwidth{\ifdim\Gin@nat@width>\linewidth\linewidth\else\Gin@nat@width\fi}
\def\maxheight{\ifdim\Gin@nat@height>\textheight\textheight\else\Gin@nat@height\fi}
\makeatother
% Scale images if necessary, so that they will not overflow the page
% margins by default, and it is still possible to overwrite the defaults
% using explicit options in \includegraphics[width, height, ...]{}
\setkeys{Gin}{width=\maxwidth,height=\maxheight,keepaspectratio}
% Set default figure placement to htbp
\makeatletter
\def\fps@figure{htbp}
\makeatother

\captionsetup[figure]{labelformat=empty}
\usepackage{pgfpages}
\setbeamertemplate{itemize item}[circle]
\setbeamertemplate{footline}[frame number]{}
\mode<handout>{\pgfpagesuselayout{6 on 1}[letterpaper, border shrink=8mm]}
\AtBeginSection{%
   \begin{frame}
       \tableofcontents[currentsection]
   \end{frame}
}
\makeatletter
\makeatother
\makeatletter
\makeatother
\makeatletter
\@ifpackageloaded{caption}{}{\usepackage{caption}}
\AtBeginDocument{%
\ifdefined\contentsname
  \renewcommand*\contentsname{Table of contents}
\else
  \newcommand\contentsname{Table of contents}
\fi
\ifdefined\listfigurename
  \renewcommand*\listfigurename{List of Figures}
\else
  \newcommand\listfigurename{List of Figures}
\fi
\ifdefined\listtablename
  \renewcommand*\listtablename{List of Tables}
\else
  \newcommand\listtablename{List of Tables}
\fi
\ifdefined\figurename
  \renewcommand*\figurename{Figure}
\else
  \newcommand\figurename{Figure}
\fi
\ifdefined\tablename
  \renewcommand*\tablename{Table}
\else
  \newcommand\tablename{Table}
\fi
}
\@ifpackageloaded{float}{}{\usepackage{float}}
\floatstyle{ruled}
\@ifundefined{c@chapter}{\newfloat{codelisting}{h}{lop}}{\newfloat{codelisting}{h}{lop}[chapter]}
\floatname{codelisting}{Listing}
\newcommand*\listoflistings{\listof{codelisting}{List of Listings}}
\makeatother
\makeatletter
\@ifpackageloaded{caption}{}{\usepackage{caption}}
\@ifpackageloaded{subcaption}{}{\usepackage{subcaption}}
\makeatother
\makeatletter
\@ifpackageloaded{tcolorbox}{}{\usepackage[many]{tcolorbox}}
\makeatother
\makeatletter
\@ifundefined{shadecolor}{\definecolor{shadecolor}{rgb}{.97, .97, .97}}
\makeatother
\makeatletter
\makeatother
\ifLuaTeX
  \usepackage{selnolig}  % disable illegal ligatures
\fi
\IfFileExists{bookmark.sty}{\usepackage{bookmark}}{\usepackage{hyperref}}
\IfFileExists{xurl.sty}{\usepackage{xurl}}{} % add URL line breaks if available
\urlstyle{same} % disable monospaced font for URLs
\hypersetup{
  pdftitle={Knowledge and Reality, Lecture 09},
  pdfauthor={Brian Weatherson},
  hidelinks,
  pdfcreator={LaTeX via pandoc}}

\title{Knowledge and Reality, Lecture 09}
\author{Brian Weatherson}
\date{2022-09-28}

\begin{document}
\frame{\titlepage}
\ifdefined\Shaded\renewenvironment{Shaded}{\begin{tcolorbox}[interior hidden, breakable, enhanced, borderline west={3pt}{0pt}{shadecolor}, boxrule=0pt, sharp corners, frame hidden]}{\end{tcolorbox}}\fi

\hypertarget{review}{%
\section{Review}\label{review}}

\begin{frame}{A Dilemma}
\protect\hypertarget{a-dilemma}{}
If there is some feature of true belief that suffices for knowledge, it
must either

\begin{enumerate}[<+->]
\tightlist
\item
  Not survive under logical reasoning; or
\item
  Imply truth
\end{enumerate}
\end{frame}

\begin{frame}{Option 1}
\protect\hypertarget{option-1}{}
Nozick defended a sensitivity theory that says that the good making
feature of belief - counterfactual sensitivity - did not survive under
logical reasoning.

\begin{itemize}[<+->]
\tightlist
\item
  But it gave us strange results in many cases.
\item
  And it left it hard to say what the point of logical reasoning is.
\end{itemize}
\end{frame}

\begin{frame}{Option 2}
\protect\hypertarget{option-2}{}
So a lot of people moved to option 2.

\begin{itemize}[<+->]
\tightlist
\item
  Goal: Find some good-making feature of belief that implies truth, and
  which might be enough for knowledge.
\end{itemize}
\end{frame}

\begin{frame}{Intuitive Idea}
\protect\hypertarget{intuitive-idea}{}
The good beliefs are not just justified \textbf{and} true, they are
justified \textbf{because} true.

\begin{itemize}[<+->]
\tightlist
\item
  The explanation of the success of the inquiry, i.e., that it resulted
  in a true belief, is that it was carried out in a skilful way, i.e.,
  the belief was justified.
\end{itemize}
\end{frame}

\begin{frame}{Plan for Today}
\protect\hypertarget{plan-for-today}{}
\begin{enumerate}[<+->]
\tightlist
\item
  Basics of Virtue Epistemology
\item
  Sosa on Virtue and Knowledge
\item
  Sosa on Scepticism
\end{enumerate}
\end{frame}

\hypertarget{virtue-epistemology}{%
\section{Virtue Epistemology}\label{virtue-epistemology}}

\begin{frame}{Ethics and Epistemology}
\protect\hypertarget{ethics-and-epistemology}{}
This is part of a big trend towards making ethics and epistemology more
entwined.

\begin{itemize}[<+->]
\tightlist
\item
  This goes in both directions: a lot of work in contemporary ethics is
  on the nature of moral reasons, and this work draws on what
  epistemologists have said about reasons and reasoning.
\end{itemize}
\end{frame}

\begin{frame}{Ethics and Epistemology}
\protect\hypertarget{ethics-and-epistemology-1}{}
At least five big trends.

\begin{enumerate}[<+->]
\tightlist
\item
  What kinds of things epistemic evaluations are.
\item
  What connections there are between knowledge and proper action.
\item
  Whether there is a proper/forgivable distinction in epistemology.
\item
  Whether epistemic consequentialism makes sense.
\item
  Whether epistemic virtues make sense.
\end{enumerate}
\end{frame}

\begin{frame}{Virtue Epistemology}
\protect\hypertarget{virtue-epistemology-1}{}
Traditional (i.e., pre-circa-1995) epistemology had a mistaken focus in
two ways.

\begin{enumerate}[<+->]
\tightlist
\item
  Too much attention to belief, i.e., the end of inquiry, as opposed to
  the process of inquiry.
\item
  Too much attention to very big picture evaluations, like rational,
  justified, knowledge, rather than more specific things like
  open-mindedness, curiosity, humility.
\end{enumerate}
\end{frame}

\begin{frame}{Vice Epistemology}
\protect\hypertarget{vice-epistemology}{}
In the last few years (i.e., since-circa-2015), there has been a small
uptick in thinking that we should start with the negative side, i.e.,
epistemic vices.

\begin{itemize}[<+->]
\tightlist
\item
  Stubbornness
\item
  Closed-mindedness
\end{itemize}
\end{frame}

\begin{frame}{Two Models of Virtue}
\protect\hypertarget{two-models-of-virtue}{}
\begin{enumerate}[<+->]
\tightlist
\item
  Reliabilist: virtues are things that reliably get you to truth (or
  some other desires end)
\item
  Responsible: virtues are things that responsible believers/inquirers
  do.
\end{enumerate}

\begin{itemize}[<+->]
\tightlist
\item
  First is clearly externalist; second could be either internalist or
  externalist.
\end{itemize}
\end{frame}

\begin{frame}{Virtues as Skills}
\protect\hypertarget{virtues-as-skills}{}
Some virtue epistemologists (not all, I think) equate epistemic virtues
with having \textbf{skills}.

\begin{itemize}[<+->]
\tightlist
\item
  This naturally goes with linking epistemic notions to notions of
  \textbf{achivement}.
\item
  Sosa in particular is fond of these equations, but he is hardly the
  only one.
\end{itemize}
\end{frame}

\hypertarget{sosa-on-virtue-and-knowledge}{%
\section{Sosa on Virtue and
Knowledge}\label{sosa-on-virtue-and-knowledge}}

\begin{frame}{Sosa}
\protect\hypertarget{sosa}{}
Three claims, in increasing order of strength.

\begin{itemize}[<+->]
\tightlist
\item
  Virtues are central to epistemology.
\item
  What makes something a virtue is that it reliably produces true
  beliefs.
\item
  Knowledge can be understood in terms of virtues.
\end{itemize}
\end{frame}

\begin{frame}{AAA model}
\protect\hypertarget{aaa-model}{}
\begin{itemize}[<+->]
\tightlist
\item
  A belief is \textbf{accurate} iff it is true.
\item
  A belief is \textbf{adroit} iff it is skilfully produced.
\item
  A belief is \textbf{apt} iff it is accurate because adroit.
\end{itemize}
\end{frame}

\begin{frame}{Belief and Achivements}
\protect\hypertarget{belief-and-achivements}{}
Can apply AAA model to any kind of goal-directed activity.

\begin{itemize}[<+->]
\tightlist
\item
  The account of accuracy will change, as will the theory of skill, but
  the structure will stay.
\end{itemize}
\end{frame}

\begin{frame}{Achivements}
\protect\hypertarget{achivements}{}
What is it for a result to be an achievement?
\end{frame}

\begin{frame}{Achievements}
\protect\hypertarget{achievements}{}
Toy theory. S's action is an achievement iff:

\begin{enumerate}[<+->]
\tightlist
\item
  It causes a good result.
\item
  It involves an exercise in the kind of skill that typically produces
  that good result.
\end{enumerate}
\end{frame}

\begin{frame}{Gettier Cases For Achievements}
\protect\hypertarget{gettier-cases-for-achievements}{}
Imagine a case where someone has a skill, and causes the outcome that
skill usually causes, but there is a lot of luck involved.

\begin{itemize}[<+->]
\tightlist
\item
  E.g., a skilled baker is baking a cake, and they make a rare slip and
  put in too much baking powder.
\item
  But the baking powder had gone somewhat stale, so it didn't have too
  much effect on the cake.
\end{itemize}
\end{frame}

\begin{frame}{Gettier Cases for Achievements}
\protect\hypertarget{gettier-cases-for-achievements-1}{}
Even if the cake works out well, it doesn't make the cake an
\textbf{achivement}.

\begin{itemize}[<+->]
\tightlist
\item
  And this tells us something interesting about the nature of
  achievement.
\item
  It requires not just success and skill, but something like success
  \textbf{because of} skill.
\end{itemize}
\end{frame}

\begin{frame}{Philosophical Significance}
\protect\hypertarget{philosophical-significance}{}
A lot of authors (Pasnau is sort of going to be an example) think that
the focus on Gettier type cases is overdone.

\begin{itemize}[<+->]
\tightlist
\item
  Who cares precisely how the English word `knows' gets applied?
\item
  Sosa is making an interesting reply here.
\item
  Gettier cases reveal something very general about the structure of
  achievements.
\end{itemize}
\end{frame}

\begin{frame}{Gettier Cases in Epistemology}
\protect\hypertarget{gettier-cases-in-epistemology}{}
\begin{itemize}[<+->]
\tightlist
\item
  The beliefs are both accurate and adroit, but these facts are
  coincidences.
\item
  The accuracy is not explained by the adroitness.
\end{itemize}
\end{frame}

\begin{frame}{Animal Knowledge and Reflective Knowledge}
\protect\hypertarget{animal-knowledge-and-reflective-knowledge}{}
\begin{itemize}[<+->]
\tightlist
\item
  These three conditions give one \textbf{animal} knoweldge.
\item
  As the name suggests, these are conditions that Sosa thinks even
  (non-human) animals can readily attain.
\item
  Actually a tricky question here about which animals have beliefs, and
  what it means to say an animal has a belief, but assuming they do,
  they can have AAA beliefs.
\end{itemize}
\end{frame}

\begin{frame}{Animal Knowledge and Reflective Knowledge}
\protect\hypertarget{animal-knowledge-and-reflective-knowledge-1}{}
Reflective knowledge is animal knowledge that one has animal knowledge.

\begin{itemize}[<+->]
\tightlist
\item
  It requires being able to reflect on one's own skills, and correctly
  judge that one has acquired true beliefs by the exercise of skill, in
  virtue of having skill at making this kind of judgment.
\item
  That's probably too much for non-humans!
\end{itemize}
\end{frame}

\hypertarget{sosa-on-scepticism}{%
\section{Sosa on Scepticism}\label{sosa-on-scepticism}}

\begin{frame}{Three Stages}
\protect\hypertarget{three-stages}{}
\begin{enumerate}[<+->]
\tightlist
\item
  Sensitivity
\item
  Safety
\item
  Aptness
\end{enumerate}
\end{frame}

\begin{frame}{Sensitivity}
\protect\hypertarget{sensitivity}{}
If \(p\) weren't true, S wouldn't believe it.

\begin{itemize}[<+->]
\tightlist
\item
  Leads to problems in things like the Potemkin village case.
\end{itemize}
\end{frame}

\begin{frame}{Safety}
\protect\hypertarget{safety}{}
Sosa's version: If S believed \(p\), it would be true.

\begin{itemize}[<+->]
\tightlist
\item
  I don't know really how to make sense of that, given that S does
  actually believe it.
\end{itemize}
\end{frame}

\begin{frame}{Safety}
\protect\hypertarget{safety-1}{}
Better version: There is no easy possibility where S falsely believes
\(p\).
\end{frame}

\begin{frame}{Problem}
\protect\hypertarget{problem}{}
Dreaming.

\begin{itemize}[<+->]
\tightlist
\item
  Evil demon, BIV possibilities are far-fetched. (Or so we non-sceptics
  say.)
\item
  But dreaming scenarios are not far-fetched; they happen all the time.
\end{itemize}
\end{frame}

\begin{frame}{About a Dream}
\protect\hypertarget{about-a-dream}{}
Two kinds of responses.

\begin{enumerate}[<+->]
\tightlist
\item
  We don't really form \textbf{beliefs} in dreams. This is what Sosa
  defends in chapter 1 of this book (which we didn't read).
\item
  Beliefs are so distinctive - they have a \textbf{dreamlike} quality -
  that we can tell when we're awake that we're not in them.
\end{enumerate}
\end{frame}

\begin{frame}{Sosa's Ch 2 Response}
\protect\hypertarget{sosas-ch-2-response}{}
Aptness is a matter of whether what S is doing works well in the
situation S is in.

\begin{itemize}[<+->]
\tightlist
\item
  It isn't the same as doing well in a range of cases.
\item
  A goalkeeper might be good at saving a very particular kind of shot,
  even if they are bad at saving other shots.
\item
  Skills in general can be very fine-grained.
\end{itemize}
\end{frame}

\begin{frame}{Sosa's Ch 2 Response}
\protect\hypertarget{sosas-ch-2-response-1}{}
Our skills at navigating the world \textbf{when awake} could be real
skills, even if they misfire when we're asleep.

\begin{itemize}[<+->]
\tightlist
\item
  What makes an action skillful is that it is the right kind of thing to
  do \textbf{in that situation}.
\item
  And the same for a belief forming process.
\end{itemize}
\end{frame}

\begin{frame}{For Next Time}
\protect\hypertarget{for-next-time}{}
\begin{itemize}[<+->]
\tightlist
\item
  We'll keep looking at scepticism.
\item
  Including looking at other arguments that we don't know we're not in
  sceptical scenarios.
\end{itemize}
\end{frame}



\end{document}
