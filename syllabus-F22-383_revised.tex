% Options for packages loaded elsewhere
\PassOptionsToPackage{unicode}{hyperref}
\PassOptionsToPackage{hyphens}{url}
\PassOptionsToPackage{dvipsnames,svgnames,x11names}{xcolor}
%
\documentclass[
]{article}

\usepackage{amsmath,amssymb}
\usepackage{lmodern}
\usepackage{iftex}
\ifPDFTeX
  \usepackage[T1]{fontenc}
  \usepackage[utf8]{inputenc}
  \usepackage{textcomp} % provide euro and other symbols
\else % if luatex or xetex
  \usepackage{unicode-math}
  \defaultfontfeatures{Scale=MatchLowercase}
  \defaultfontfeatures[\rmfamily]{Ligatures=TeX,Scale=1}
  \setmainfont[BoldFont = SF Pro Rounded Semibold,Scale =
MatchLowercase]{SF Pro Thin}
  \setmathfont[]{Fira Math}
\fi
% Use upquote if available, for straight quotes in verbatim environments
\IfFileExists{upquote.sty}{\usepackage{upquote}}{}
\IfFileExists{microtype.sty}{% use microtype if available
  \usepackage[]{microtype}
  \UseMicrotypeSet[protrusion]{basicmath} % disable protrusion for tt fonts
}{}
\makeatletter
\@ifundefined{KOMAClassName}{% if non-KOMA class
  \IfFileExists{parskip.sty}{%
    \usepackage{parskip}
  }{% else
    \setlength{\parindent}{0pt}
    \setlength{\parskip}{6pt plus 2pt minus 1pt}}
}{% if KOMA class
  \KOMAoptions{parskip=half}}
\makeatother
\usepackage{xcolor}
\usepackage[margin=1.5in]{geometry}
\setlength{\emergencystretch}{3em} % prevent overfull lines
\setcounter{secnumdepth}{-\maxdimen} % remove section numbering


\providecommand{\tightlist}{%
  \setlength{\itemsep}{0pt}\setlength{\parskip}{0pt}}\usepackage{longtable,booktabs,array}
\usepackage{calc} % for calculating minipage widths
% Correct order of tables after \paragraph or \subparagraph
\usepackage{etoolbox}
\makeatletter
\patchcmd\longtable{\par}{\if@noskipsec\mbox{}\fi\par}{}{}
\makeatother
% Allow footnotes in longtable head/foot
\IfFileExists{footnotehyper.sty}{\usepackage{footnotehyper}}{\usepackage{footnote}}
\makesavenoteenv{longtable}
\usepackage{graphicx}
\makeatletter
\def\maxwidth{\ifdim\Gin@nat@width>\linewidth\linewidth\else\Gin@nat@width\fi}
\def\maxheight{\ifdim\Gin@nat@height>\textheight\textheight\else\Gin@nat@height\fi}
\makeatother
% Scale images if necessary, so that they will not overflow the page
% margins by default, and it is still possible to overwrite the defaults
% using explicit options in \includegraphics[width, height, ...]{}
\setkeys{Gin}{width=\maxwidth,height=\maxheight,keepaspectratio}
% Set default figure placement to htbp
\makeatletter
\def\fps@figure{htbp}
\makeatother

%\usepackage{multicol}
%\setlength{\columnsep}{1cm}
%\linespread{1.05}

% \usepackage{titlesec}
% \usepackage{titling}
% \usepackage{fontspec}
% % Specify different font for section headings
% \newfontfamily\headingfont[]{SF Pro Thin}
% \titleformat{\section}
%   {\headingfont\Large\bfseries}{\thesection}{1em}{}
% \titleformat{\subsection}
%   {\headingfont\large\bfseries}{\thesubsection}{1em}{}
% \titleformat{\subsubsection}
%   {\headingfont\normalsize\bfseries}{\thesubsubsection}{1em}{}
% 
% \renewcommand{\maketitlehooka}{\headingfont}

%\setkomafont{disposition}{\normalcolor\bfseries}

%\setkomafont{descriptionlabel}{\normalfont\bfseries}

\setlength\heavyrulewidth{0pt}
\setlength\lightrulewidth{0pt}
\makeatletter
\makeatother
\makeatletter
\makeatother
\makeatletter
\@ifpackageloaded{caption}{}{\usepackage{caption}}
\AtBeginDocument{%
\ifdefined\contentsname
  \renewcommand*\contentsname{Table of contents}
\else
  \newcommand\contentsname{Table of contents}
\fi
\ifdefined\listfigurename
  \renewcommand*\listfigurename{List of Figures}
\else
  \newcommand\listfigurename{List of Figures}
\fi
\ifdefined\listtablename
  \renewcommand*\listtablename{List of Tables}
\else
  \newcommand\listtablename{List of Tables}
\fi
\ifdefined\figurename
  \renewcommand*\figurename{Figure}
\else
  \newcommand\figurename{Figure}
\fi
\ifdefined\tablename
  \renewcommand*\tablename{Table}
\else
  \newcommand\tablename{Table}
\fi
}
\@ifpackageloaded{float}{}{\usepackage{float}}
\floatstyle{ruled}
\@ifundefined{c@chapter}{\newfloat{codelisting}{h}{lop}}{\newfloat{codelisting}{h}{lop}[chapter]}
\floatname{codelisting}{Listing}
\newcommand*\listoflistings{\listof{codelisting}{List of Listings}}
\makeatother
\makeatletter
\@ifpackageloaded{caption}{}{\usepackage{caption}}
\@ifpackageloaded{subcaption}{}{\usepackage{subcaption}}
\makeatother
\makeatletter
\@ifpackageloaded{tcolorbox}{}{\usepackage[many]{tcolorbox}}
\makeatother
\makeatletter
\@ifundefined{shadecolor}{\definecolor{shadecolor}{rgb}{.97, .97, .97}}
\makeatother
\makeatletter
\makeatother
\ifLuaTeX
  \usepackage{selnolig}  % disable illegal ligatures
\fi
\IfFileExists{bookmark.sty}{\usepackage{bookmark}}{\usepackage{hyperref}}
\IfFileExists{xurl.sty}{\usepackage{xurl}}{} % add URL line breaks if available
\urlstyle{same} % disable monospaced font for URLs
\hypersetup{
  pdftitle={PHIL 383: Knowledge and Reality},
  pdfauthor={Brian Weatherson},
  colorlinks=true,
  linkcolor={blue},
  filecolor={Maroon},
  citecolor={Blue},
  urlcolor={Blue},
  pdfcreator={LaTeX via pandoc}}

\title{PHIL 383: Knowledge and Reality}
\author{Brian Weatherson}
\date{1/1/22}

\begin{document}
\maketitle
\ifdefined\Shaded\renewenvironment{Shaded}{\begin{tcolorbox}[borderline west={3pt}{0pt}{shadecolor}, sharp corners, interior hidden, enhanced, boxrule=0pt, breakable, frame hidden]}{\end{tcolorbox}}\fi

\textbf{Lead Instructor}: Brian Weatherson\\
\includegraphics[width=1em,height=1em]{figs/fa-icon-60569dd17e84ca89a2982f5f8ad92685.pdf}
weath@umich.edu\\
\includegraphics[width=1.25em,height=1em]{figs/fa-icon-0dbcc241ba7e369df73ab7fe566f4960.pdf}
canvas.umich.edu\\
\strut \\
\textbf{Office Hours}: Wednesday 10-11, Room 2207; Friday 10-11, Zoom.\\
\strut \\
\textbf{Discussion Section Leader}: Sumeet Patwardhan\\
\includegraphics[width=1em,height=1em]{figs/fa-icon-60569dd17e84ca89a2982f5f8ad92685.pdf}
sumeetcp@umich.edu

\hypertarget{course-description}{%
\section{Course Description}\label{course-description}}

This course discusses a number of topics in \emph{epistemology}, the
theory of knowledge. We will start with a very brief survey of some
recent debates, mainly to ensure everyone is up to speed with what is
assumed in the readings later in the course. Then the bulk of the course
will be devoted to working through two recent books: \emph{After
Certainty} by Robert Pasnau, and \emph{The Rationality of Perception} by
Susanna Siegel. These books are designed to complement one another.
Pasnau's book is a history of how European epistemology developed the
way it did, focussing on two key periods, the early 14th century and the
late 17th century. Siegel's book engages with contemporary empirical
work on perception and, via considerations of the social and political
significance of some features of perception, comes to a new view on the
role of perception in epistemology. So hopefully looking at some
contemporary work in light of historical work will give us fresh
insights into both.

\hypertarget{canvas}{%
\section{Canvas}\label{canvas}}

There is a Canvas site for this course, which can be accessed from
\url{https://canvas.umich.edu}. Course documents (syllabus, lecture
notes, assignments) will be available from this site. Please make sure
that you can access this site. Consult the site regularly for
announcements, including changes to the course schedule. And there are
many tools on the site to communicate with each other, and with me.

\newpage

\hypertarget{required-materials}{%
\section{Required Materials}\label{required-materials}}

There are two books for the course. Both of them are available through
the bookstore. You do not need physical versions of the books; I'll be
primarily teaching off the electronic versions. The books are:

\begin{itemize}
\tightlist
\item
  \emph{After Certainty: A History of Our Epistemic Ideals and
  Illusions}, by Robert Pasnau. Oxford University Press, 2017.
\item
  \emph{The Rationality of Perception}, by Susanna Siegel. Oxford
  University Press, 2017.
\end{itemize}

Note that in class we'll be working through the main text of Pasnau's
book, not the hundreds of pages of endnotes. These are not required
reading, but I do encourage you to dip into them if you're interested in
more detail on any of the points. The endnotes are basically blog posts,
sometimes short essays, expanding on the text, and often filling in
fascinating historical detail. But I'm not expecting everyone to read
all of them in the middle of term.

\hypertarget{course-requirements}{%
\section{Course Requirements}\label{course-requirements}}

\begin{enumerate}
\def\labelenumi{\arabic{enumi}.}
\item
  Participate in discussion section. This will require that you have
  read the relevant material closely, and thought about what to say
  about it. And it will require that you then participate in the
  discussions in section. This will help you learn, and it will help
  your classmates learn.
\item
  Do three online quizzes. At the end of each section of the course,
  there will be a short quiz to do on Canvas. The quizzes will be
  multiple choice.
\item
  Write three short essays. You will have to write a short paper, around
  1500 words, on each of the three sections of the course. Topics will
  be distributed well in advance.
\end{enumerate}

\hypertarget{grade-breakdown}{%
\section{Grade Breakdown}\label{grade-breakdown}}

\begin{itemize}
\tightlist
\item
  Discussion section participation: 10\%;
\item
  Three Quizzes: 10\% each, for a total of 30\%;
\item
  Three Essays: 20\% each, for a total of 60\%.
\end{itemize}

\hypertarget{due-dates}{%
\section{Due Dates}\label{due-dates}}

\begin{itemize}
\tightlist
\item
  Quiz One: Friday, September 30, 5pm.
\item
  Essay One: Friday, October 14, 5pm.
\item
  Quiz Two: Friday, October 28, 5pm.
\item
  Essay Two: Friday, November 04, 5pm.
\item
  Quiz Three: Friday, December 02, 5pm.
\item
  Essay Three: Friday, December 16, midday.
\end{itemize}

\newpage

The course is in three parts: Introduction, which is weeks 1-5, Pasnau,
which is weeks 6-8, and Siegel, which is weeks 9-12. We will do some
revision and general discussion of things that have come up in the
course after that. The readings are mostly linked, except for the
textbooks, and two pieces that are just through Canvas. Some of the
links require UM authorisation; it's easiest to simply be on campus when
you access them.

\hypertarget{part-one-introduction}{%
\section{Part One: Introduction}\label{part-one-introduction}}

\hypertarget{monday-august-29}{%
\subsection{Monday, August 29}\label{monday-august-29}}

\begin{description}
\tightlist
\item[Topic]
Introducing the course
\item[Required Reading]
This syllabus
\item[Suggested Reading]
None
\end{description}

\hypertarget{wednesday-august-31}{%
\subsection{Wednesday, August 31}\label{wednesday-august-31}}

\begin{description}
\tightlist
\item[Topic]
Classical Indian Philosophy
\item[Required Reading]
Stephen Phillips,
\href{https://plato.stanford.edu/entries/epistemology-india/}{\emph{Epistemology
in Classical Indian Philosophy}}.
\item[Suggested Reading]
None
\end{description}

\hypertarget{wednesday-september-07}{%
\subsection{Wednesday, September 07}\label{wednesday-september-07}}

\begin{description}
\tightlist
\item[Topic]
Testimony as a pramāṇa
\item[Required Reading]
Dhirendon Mohon Datta,
\href{https://www.jstor.org/stable/2249544}{\emph{Testimony as a Method
of Knowledge}}.
\item[Suggested Reading]
Nick Leonard,
\href{https://plato.stanford.edu/entries/testimony-episprob/}{\emph{Epistemological
Problems of Testimony}}, especially §§1 \& 3.
\end{description}

\hypertarget{monday-september-12}{%
\subsection{Monday, September 12}\label{monday-september-12}}

\begin{description}
\tightlist
\item[Topic]
Testimony and vigilance
\item[Required Reading]
Dan Sperber \emph{et al},
\href{https://doi.org/10.1111/j.1468-0017.2010.01394.x}{\emph{Epistemic
Vigilance}}.
\item[Suggested Reading]
Kourken Michaelian, \href{https://doi.org/10.1017/epi.2013.2}{\emph{The
Evolution of Testimony: Receiver Vigilance, Speaker Honesty and the
Reliability of Communication}}.
\end{description}

\hypertarget{wednesday-september-14}{%
\subsection{Wednesday, September 14}\label{wednesday-september-14}}

\begin{description}
\tightlist
\item[Topic]
Internalism and Externalism
\item[Required Reading]
George Pappas,
\href{https://plato.stanford.edu/entries/justep-intext/}{\emph{Internalist
vs.~Externalist Conceptions of Epistemic Justification}}, §§7 \& 10.

Alvin Goldman and Bob Beddor,
\href{https://plato.stanford.edu/entries/reliabilism/}{\emph{Reliabilist
Epistemology}}, §2.
\item[Suggested Reading]
The other sections of those articles.

James Pryor,
\href{https://www.jstor.org/stable/3541945}{\emph{Highlights of Recent
Epistemology}}, §3.
\end{description}

\hypertarget{monday-september-19}{%
\subsection{Monday, September 19}\label{monday-september-19}}

\begin{description}
\tightlist
\item[Topic]
Political arguments for externalism
\item[Required Reading]
Amia Srinivasan,
\href{https://doi.org/10.1215/00318108-8311261}{\emph{Radical
Externalism}}.
\item[Suggested Reading]
Zoë Johnson King,
\href{https://www.zoejohnsonking.com/s/Radical-Internalism-draft-46.pdf}{\emph{Radical
Internalism}}.
\end{description}

\newpage

\hypertarget{wednesday-september-21}{%
\subsection{Wednesday, September 21}\label{wednesday-september-21}}

\begin{description}
\tightlist
\item[Topic]
Analysis of Knowledge
\item[Required Reading]
Jonathan Jenkins Ichikawa and Matthias Steup,
\href{https://plato.stanford.edu/entries/knowledge-analysis/}{\emph{Anaysis
of Knowledge}}, §§1-7.
\item[Suggested Reading]
The rest of this entry.

Edmund Gettier, \href{https://doi.org/10.1093/analys/23.6.121}{\emph{Is
Justified True Belief Knowledge?}}.
\end{description}

\hypertarget{monday-september-26}{%
\subsection{Monday, September 26}\label{monday-september-26}}

\begin{description}
\tightlist
\item[Topic]
Virtue and knowledge
\item[Required Reading]
Ernest Sosa, \emph{A Virtue Epistemology}. (available on Canvas)
\item[Suggested Reading]
John Turri, Mark Alfano, and John Greco,
\href{https://plato.stanford.edu/entries/epistemology-virtue/}{\emph{Virtue
Epistemology}}.
\end{description}

\hypertarget{wednesday-september-28}{%
\subsection{Wednesday, September 28}\label{wednesday-september-28}}

\begin{description}
\tightlist
\item[Topic]
Descartes on Scepticism
\item[Required Reading]
René Descartes, \emph{Meditations on First Philosophy}, Meditations 1
and 2 (pages 6-13 of PDF available on Canvas).
\item[Suggested Reading]
The rest of the Meditations.
\end{description}

\hypertarget{monday-october-03}{%
\subsection{Monday, October 03}\label{monday-october-03}}

\begin{description}
\tightlist
\item[Topic]
Modern Sceptical Arguments
\item[Required Reading]
Alex Byrne,
\href{https://doi.org/10.1111/j.1468-0068.2004.00471.x}{\emph{How Hard
are the Sceptical Paradoxes?}}.
\item[Suggested Reading]
Brian Weatherson,
\href{http://brian.weatherson.org/html-papers/posts/2021-01-08-scepticism-rationalism-and-externalism/}{\emph{Scepticism,
Rationalism and Externalism}}
\end{description}

\hypertarget{part-two-robert-pasnau-on-history-of-epistemology}{%
\section{Part Two: Robert Pasnau on History of
Epistemology}\label{part-two-robert-pasnau-on-history-of-epistemology}}

Throughout this part, the best suggestions for extra readings will be to
either read the extensive endnotes, or to follow up some of the primary
sources that Pasnau discusses.

\hypertarget{wednesday-october-05}{%
\subsection{Wednesday, October 05}\label{wednesday-october-05}}

\begin{description}
\tightlist
\item[Topic]
Ideals
\item[Required Reading]
Pasnau, Chapter 1
\end{description}

\hypertarget{monday-october-10}{%
\subsection{Monday, October 10}\label{monday-october-10}}

\begin{description}
\tightlist
\item[Topic]
Certainty
\item[Required Reading]
Pasnau, Chapter 2
\item[Suggested Reading]
Hanti Lin,
\href{https://plato.stanford.edu/entries/epistemology-bayesian/}{\emph{Bayesian
Epistemology}}
\end{description}

\hypertarget{wednesday-october-12}{%
\subsection{Wednesday, October 12}\label{wednesday-october-12}}

\begin{description}
\tightlist
\item[Topic]
Perception
\item[Required Reading]
Pasnau, Chapter 3
\end{description}

\hypertarget{wednesday-october-19-and-monday-october-24}{%
\subsection{Wednesday, October 19 and Monday, October
24}\label{wednesday-october-19-and-monday-october-24}}

\begin{description}
\tightlist
\item[Topic]
Illusion
\item[Required Reading]
Pasnau, Chapter 4
\end{description}

\hypertarget{wednesday-october-26}{%
\subsection{Wednesday, October 26}\label{wednesday-october-26}}

\begin{description}
\tightlist
\item[Topic]
Proof
\item[Required Reading]
Pasnau, Chapter 5
\end{description}

\hypertarget{monday-october-31}{%
\subsection{Monday, October 31}\label{monday-october-31}}

\begin{description}
\tightlist
\item[Topic]
Scepticism
\item[Required Reading]
Pasnau, Chapter 6
\end{description}

\hypertarget{part-three-susanna-siegel-on-perception}{%
\section{Part Three: Susanna Siegel on
Perception}\label{part-three-susanna-siegel-on-perception}}

\hypertarget{wednesday-november-02}{%
\subsection{Wednesday, November 02}\label{wednesday-november-02}}

\begin{description}
\tightlist
\item[Topic]
Hijacked Experience
\item[Required Reading]
Siegel, Preface and Chapter 1
\end{description}

\hypertarget{monday-november-07}{%
\subsection{Monday, November 07}\label{monday-november-07}}

\begin{description}
\tightlist
\item[Topic]
Perceptions as Irrational
\item[Required Reading]
Siegel, Chapters 2 and 3
\end{description}

\hypertarget{wednesday-november-09}{%
\subsection{Wednesday, November 09}\label{wednesday-november-09}}

\begin{description}
\tightlist
\item[Topic]
Epistemic Downgrade
\item[Required Reading]
Siegel, Chapter 4
\end{description}

\hypertarget{monday-november-14}{%
\subsection{Monday, November 14}\label{monday-november-14}}

\begin{description}
\tightlist
\item[Topic]
Inference
\item[Required Reading]
Siegel, Chapter 5
\item[Suggested Reading]
Peter Railton,
\href{https://onlinelibrary.wiley.com/doi/10.1111/phpr.12735}{\emph{Comment
on Susanna Siegel, The Rationality of Perception}}
\end{description}

\hypertarget{wednesday-november-16}{%
\subsection{Wednesday, November 16}\label{wednesday-november-16}}

\begin{description}
\tightlist
\item[Topic]
Inference and Experience
\item[Required Reading]
Siegel, Chapters 6-7
\end{description}

\hypertarget{monday-november-21}{%
\subsection{Monday, November 21}\label{monday-november-21}}

\begin{description}
\tightlist
\item[Topic]
Perception and Selection
\item[Required Reading]
Siegel, Chapters 8-9
\end{description}

\hypertarget{monday-november-28}{%
\subsection{Monday, November 28}\label{monday-november-28}}

\begin{description}
\tightlist
\item[Topic]
Perception and Culture
\item[Required Reading]
Siegel, Chapter 10
\end{description}

\hypertarget{wednesday-november-30}{%
\subsection{Wednesday, November 30}\label{wednesday-november-30}}

\begin{description}
\tightlist
\item[Topic]
Critics of Siegel
\item[Required Reading]
Errol Lord,
\href{https://onlinelibrary.wiley.com/doi/10.1111/phpr.12734}{\emph{The
Vices of Perception}}

Adam Pautz,
\href{https://onlinelibrary.wiley.com/doi/10.1111/phpr.12733}{\emph{The
Arationality of Perception: Comments on Susanna Siegel}}
\item[Suggested Reading]
Susanna Siegel,
\href{https://onlinelibrary.wiley.com/doi/10.1111/phpr.12737}{\emph{Replies
to Lord, Railton and Pautz}}
\end{description}

\begin{itemize}
\tightlist
\item
  The last week will be for review, and for extra time for any topic we
  felt was too rushed through the course.
\end{itemize}

\newpage

\hypertarget{co-authorship}{%
\section{Co-Authorship}\label{co-authorship}}

You are allowed to co-author essays in this class. But there are some
rules.

\begin{enumerate}
\def\labelenumi{\arabic{enumi}.}
\tightlist
\item
  Only essays, not quizzes, can be co-authored.
\item
  Each paper can have at most two co-authors.
\item
  You can only co-author one paper with each co-author. (That is, you
  can co-author multiple papers, but you have to have different
  collaborators each time.)
\item
  Each person has to turn the paper in on Canvas, and it has to be
  clearly marked as a co-authored piece.
\end{enumerate}

\hypertarget{anonymous-grading}{%
\section{Anonymous Grading}\label{anonymous-grading}}

We aim to grade work anonymously as much as possible. So when you turn
in papers, do not include your name on the PDF or DOCX file that you
turn in. You should just include your UMID as a way of getting the paper
back to you if it gets misfiled. (And if it is co-authored, make the two
ID numbers prominent, so we don't think there was in class copying!)
Canvas can show us the paper without telling us who turned it in, so as
long as there are no identifying marks on the paper itself we can grade
the papers anonymously. We think this is a fairer way to grade papers,
and we appreciate your help in making this possible.

\hypertarget{plagiarism}{%
\section{Plagiarism}\label{plagiarism}}

Although team-work, and even co-authorship, is encouraged, plagiarism is
strictly prohibited. You are responsible for making sure that none of
your work is plagiarized. Be sure to cite work that you use, both direct
quotations and paraphrased ideas. Any citation method that is tolerably
clear is permitted, but if you'd like a good description of a citation
scheme that works well in philosophy, look at
\url{http://bit.ly/VDhRJ4}.

You are encouraged to discuss the course material, including
assignments, with your classmates, but all written work that you hand in
under your own name/ID number must be your own. If work is handed is as
the work of two people, you are affirming that each person did a fair
share of the work.

You should also be familiar with the academic integrity policies of the
College of Literature, Science \& the Arts at the University of
Michigan, which are available here:
\url{https://lsa.umich.edu/lsa/academics/academic-integrity.html}.
Violations of these policies will be reported to the Office of the
Assistant Dean for Student Academic Affairs, and sanctioned with a
course grade of F.

\hypertarget{disability}{%
\section{Disability}\label{disability}}

The University of Michigan abides by the Americans with Disabilities Act
of 1990, Section 504 of the Rehabilitation Act of 1973, and other
applicable federal and state laws that prohibit discrimination on the
basis of disability, which mandate that reasonable accommodations be
provided for qualified students with disabilities.

If you have a disability, and may require some type of instructional
and/or examination accommodation, please contact me early in the
semester. If you have not already done so, you will also need to
register with the Office of Services for Students with Disabilities. The
office is located at G664 Haven Hall.

For more information on disability services at the University of
Michigan, go to \url{http://ssd.umich.edu}.



\end{document}
